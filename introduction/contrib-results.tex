\subsection*{Results}

Since all our algorithms are provably correct, we only test the equivalence of
the produced edit distance between in order to lower the risk of programming
mistakes. Empirically, on a single-thread, we compare to existing optimal
aligners the scaling, runtime and memory usage of our implementations: the
semi-global aligner \astarix and the global aligner \astarpa.

\paragraph{Worst case vs related sequence}
Traditionally, the runtime and memory have been analysed for the worst case
asymptotic behavior of the algorithm. Since the near quadratic worst case
asymptotics is likely to be tight for both global~\citep{backurs2015edit} and
semi-global alignment, we target related sequences with limited edit distance,
and estimate their empiric runtime and memory scaling.

\paragraph{Setting}
We demontrate that the additional information we use from the whole sequence can
improve the scaling with query length, reference size and error rate,
substantially decrease the necessary computations (by more than 99.99\%), and
result in algorithms that are orders of magnitude faster than existing optimal
algorithms. Our empirical analysis of scaling is based on running the same
algorithm on increasingly more complex input data (longer sequences, higher
error rate), measuring its runtime and memory usage, and fitting a function to
explain the correlation between the input parameter and the measurement. This
approach is useful to estimate the scaling of the algorithm and its
implementation. Note that these scaling analyses are not asymptotical, include
noise, and we limit the fitting to only 1 parameter at a time.

\paragraph{Scaling with reference size}
In~\cref{ch:trie} we present the tool \astarix which applies the \A algorithm to
find optimal alignments, based on a simple heuristic and enhanced by multiple
algorithmic optimizations. We demonstrate that using a trie index we can achieve
sublinear runtime scaling with reference size, and that \A can scale
exponentially better than \dijkstra with increasing (but small) number of errors
in the reads. Moreover, for short reads, both \astarix and \dijkstra scale
better and outperform current state-of-the-art optimal aligners with increasing
genome graph size.

\paragraph{Scaling with query length}
In~\cref{ch:seed} we upgrade \astarix with a novel \sh which guides the \A
search by preferring crumbs on nodes that lead towards optimal alignments even
for long reads. This approach enables the near-linear scaling of semi-global
alignment with read length. On our linear and variation graph datasets, 99.99\%
of the states are skipped due to the accurate heuristic function. On both short
reads and long low-error reads from linear and variant graph references, \A with
\sh consistently outperforms all state-of-the-art optimal aligners
\graphaligner, \vargas, \pasgal by more than 60 times.

\paragraph{Scaling with error rate}
In~\cref{ch:global} we resolve the third major bottleneck---handling high error
rates. We presented an algorithm with an implementation in \astarpa solving
pairwise alignment between two sequences. The algorithm is based on \A with a
\sh, inexact matching, match chaining, and match pruning, which we proved to
find an exact solution according to edit distance. For random sequences with up
to $15\%$ uniform errors, the runtime of \astarpa scales near-linearly to very
long sequences ($10^7\bp$) and outperforms other exact aligners. We demonstrate
that on real ONT reads from a human genome, \astarpa is faster than other
aligners on only a limited portion of the reads.

\paragraph{Seed heuristic modes of operation}
The performance of \A with the seed heuristic follows one of two distinguishable
patterns based on the capability of the seed heuristic to compensate for all the
errors. If the heuristic potential is sufficiently higher than the resulting
edit distance between the sequences, the heuristic punishes suboptimal paths
very strongly, leading to a very directed \A exploration, which grows
linear-like with the sequence length. Otherwise, if the error rate is too high,
each unit of difference that the seed heuristic does not account for, is a unit
of extended search to the side, leading to a quadratic-like exploration of the
search space. Note that the total runtime may grow quadratically for other
reasons too: due to preprocessing of dense seed matches (which is very data
dependent), or due to slow heuristic computation (which has not been a
bottleneck for us). Interestingly, since the seed heuristic improves not from
having a match but from punishing lacks of matches, no matches are required for
the near-linear behavior of the seed heuristic.

% speculations
%Based on the empitical evaluations and the intuition behind the algorithms, we
%speculate that the sequence length until which our algorithms can scale
%near-linearly, can be exponentially increased by lowering the error rate. For
%the semi-global alignment, any unit of alignment cost that is not compensated by
%the potential of the seed heuristic, leads to a deeper exploration of the trie,
%which is grows exponentially until it saturate to quadratic.

%As it turns out, when the error rate is limited, our optimal solutions
%empirically scale near-linearly up to very long sequences. This translates to
%many orders of magnitude of runtime speedup compared to state-of-the-art optimal
%aligners.

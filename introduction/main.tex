\graphicspath{{\dir/}}

\chapter{Introduction} \label{ch:introduction}

An algorithm without an objective function may be wrong because they do not
solve the correct problem. or because they 

Algorithms that guarantee correctness can be wrong only by being given a wrong problem.

An approximate algorithm can be wrong either because it did not fullfill its
mathematical goal. reach its solve the problem because of either optimizing the
\emph{wrong} function.

Algorithm correctness is arguably a useful property which is often not simple to
guarantee. It can undoubtedly improve accuracy, especially in the case of
complex data, but still be wrong from biological point of view. This is because
of  This Nevertheless, since biology is a natural science, its  optimality
guarantees must have an additional impact on the development of the field. It
not only but allows to enjoy being wrong rather than vague. Moreove, often
problems in computational biology are ill-stated andalgorithms that approximate
algorithms that. But many algorithms in computational biology do not even have a
formal statement 

\section{Motivation}

Observation 1: A long query has more information that may hint towards the best
mapping place. Unlike the practical approximate algorithms, the existing optimal
algorithm do not exploit this information. As a result, all state-of-the-art
optimal algorithms take longer to map a longer query to a reference. 

Observation 2: The input is linear, the output is linear, the best case is
linear. On the other hand all optimal solutions are near-quadratic and the
theoretical limit is near-quadratic. To spead up beyond this near-quadratic
barrier, current practical algorithms break the optimality guarantee hoping that
the produced alignments are accurate enough. 

%But it is possible to be fast and accurate in the same time?
We take another approach: we preserve the optimality guarantee and use
substantially more information with the hope to be fast enough. As it turns out,
when the error rate is limited, our optimal solutions empirically scale
near-linearly up to very long sequences. This translates to many orders of
magnitude of runtime speedup compared to state-of-the-art optimal aligners.

\subsection{}

%\subsection{Knowledge gap}
%\dictum{%
%   Provably optimal and heuristically fast.}
%\vskip 1em

Sequence alignment is a class of combinatorial problems that is of primary
importance for analysis of genetic data. Algorithms and tools for alignment have
been thoroughly developed and routinely used for genome assembly, RNA
quantification, detecting splicing, oncology, multiple sequence alignment (MSA),
and evolutionary biology. Types of sequence alignment include global alignment,
semi-global alignment, mapping, local alignment, and others. For each type, a
common tradeoff that had to be done is between the alignment accuracy and the
performance to find it.

% graph reference
An additional difficulty is the fact that in the
upcoming pangenomic era, these algorithms must be also applicable to complex
graph structures. For more than 60 years, a linear sequence has been extremely
useful as a representation of a single genome. The affordability of sequencing
technologies enables not only to sequence genomes deeper but also to sequence
many genomes (e.g. of organisms or single cells), building a pangenome (an
abstracted genome that represents the genetic variation of a whole clade). The
shift towards population studies in the last decade motivates the adoption of
graph data structures which serve as compressed representations of collections
of related genomes (genomes are paths in the graph).

% \A context 
An optimal alignment can naturally be represented as a shortest path in an
alignment graph (equivalent to the DP table). In order to find such a shortest
path with minimal exploration, we instantiate the \A algorithm with a novel
heuristic function based on the unaligned parts of the sequences. This
additional information is a problem-specific heuristic function and it heavily
determines the efficiency of the search. For any explored state by \A, this
heuristic function should compute a lower bound on the remaining path length, or
more specifically, the minimal cost of edit operations needed to align the
remaining sequences.

\paragraph{Main contributions}
We have effectively applied the \A algorithm to optimal sequence alignment. We
demontrated that the additional information from the whole sequence can improve
the scaling with query length, reference size and error rate, substantially
decrease the necessary computations, and result in algorithms that are orders of
magnitude faster than existing optimal algorithms. We apply the \A approach to
two types of alignment: semi-global (mapping) and global.


% from seeds paper

% General: aligning, edit distance
Alignment of reads to a reference genome is an essential and early step in most
bioinformatics pipelines. While linear references have been used traditionally,
an increasing interest is directed towards graph references capable of
representing biological variation~\citep{garrison_variation_2018}.
%
Specifically, a \emph{sequence-to-graph} alignment is a base-to-base
correspondence between a given read and a walk in the graph. As sequencing
errors and biological variation result in inexact read alignments, edit distance
is the most common metric that alignment algorithms optimize in order to find
the most probable read origin in the reference.

% We note that in contrast to linear references, reference graphs capture
% genomic variation and therefore enable more accurate
% alignments~\citep{garrison_variation_2018}.

\paragraph{Suboptimal alignment}
%
In the last decades, approximate and alignment-free methods satisfied the demand
for faster algorithms which process huge volumes of genetic
data~\citep{kucherov2019evolution}. 
%
\emph{Seed-and-extend} is arguably the most popular paradigm in read
alignment~\citep{altschul_basic_1990,langmead_fast_2012,li_fast_2009}. First,
substrings (called \emph{seeds} or \emph{kmers}) of the read are extracted, then
aligned to the reference, and finally prospective matching locations are
\emph{extended} on both sides to align the full read.

While such a heuristic may produce acceptable alignments in many cases, it
fundamentally does not provide quality guarantees, resulting in suboptimal
alignment accuracy.
%
In contrast, here we demonstrate that seeds can benefit optimal alignment as
well.

\paragraph{Key challenges in optimal alignment}
%
Finding optimal alignments is desirable but expensive in the worst case,
requiring $\Oh(Nm)$ time~\citep{equi2019complexity}, for graph size $N$ and read
length $m$.
%
Unfortunately, most optimal sequence-to-graph aligners rely on dynamic
programming (DP) and always reach this worst-case asymptotic runtime. Such
aligners include \vargas~\citep{darby2020vargas},
\pasgal~\citep{jain_accelerating_2019},
\graphaligner~\citep{rautiainen_bitparallel_2019},
\hga~\citep{feng2021accelerating}, and \vg~\citep{garrison_variation_2018},
which use bit-level optimizations and parallelization to increase their
throughput.

In contrast, we follow the promising direction of using a heuristic to avoid
worst-case runtime on realistic data. To this end, \astarix rephrases the task
of alignment as a shortest-path problem in an \emph{alignment graph} extended by
a \emph{trie index}, and solves it using the \A~algorithm instantiated with a
problem-specific \prefixh. Importantly, its choice of heuristic only affects
performance, not optimality.
%
Unlike DP-based algorithms, this \prefixh allows scaling sublinearly with the
reference size, substantially increasing performance on large genomes. However,
it can only efficiently align reads of limited length.

%only optimal alignment in this thesis
%asymptotics analysis
%????? linear I/O but quadratic optimal

\section{Current state}

It may come as a surprise that another approach to such a basic problem as
alignment appears 60 years after the problem was first efficiently solved. Here
we speculate about the possible reasons for it: focus on new technology and
data, not believing that optimal solutions could be efficient.

%combining several existing ideas new (\A, trie, seeds)

Most of the techniques this thesis builds upon have been known for many decades
and have also been heavily motivated by applications in molecular biology. The
global alignment problem exists since
\citeyear{vintsyuk1968speech}~\cite{vintsyuk1968speech,needleman1970general}.

The problem of approximate/fuzzy string search (semi-global alignment of one
query) has been efficiently approached a decade later, in
\citeyear{sellers1980theory}~\cite{sellers1980theory,smith1981identification}.
more recent and additionally to the issues with handling errors, a new dimension
of complexity appears by the need to locate the position in the reference. The
problem of searching for multiple sequencing has been approached
since\citeyear{pearson1988improved}~\cite{pearson1988improved} and became
central to read alignment of high-throughput sequencing. The specifics with
semi-global alignment requires a trie-like index which is well-used in the
fiels, and known since
\citeyear{thue1912gegenseitige}~\cite{thue1912gegenseitige} and used in
informatics since~\citeyear{de1959file}~\cite{de1959file}.

An important factor for the development of the sequencing algorithms is the
technological advancement~\cite{alser2021technology}. One possible reason is the
combination of massive new data and advancement in computation hardware allowing
for various kinds of parallelization. Thus the advancements with bit-parallel
algorithms (CITE Myears, Mikko).

A major importance for our apporach is the usage of seeds to build good
heuristic functions that drive the \A search. Seeds are cousins of kmers which
are popular in sequence alignment since de Bruijn Graphs were applied for genome
assembly. Further kmers are cousins of ngrams which are popular in computational
linguistics since .

% parallelization
% sketches

\subsection{Practical feasibility of optimal alignment}

There is a fundamental trade-off between perforamance and optimality guarantees:
an algorithm which is allowed to be suboptimal may exploit the lesser
restrictions for greater performance. Especially given the worst-case analysis
requiring near-quadratic runtime even to compute edit distance exactly, it is
understandable why most scholars are skeptical about faster optimal algorithms.
With our \A approach  offers a to exploits another dimensions: average case or expected case analysis.

% asymptotics
%has been solved in linear time in \citeyear{morris1970linear}~\cite{morris1970linear}.

In the direction of global alignment, optimal algorithms are commonly used in
practice, despite of their quadratic scaling. The ongoing competition between
the optimal aligners employs both algorithmic advancements and implementation
optimizations on caching, bit-parallelization, GPU.~(\cref{ch:global}).

For semi-global alignment (read alignment), common believe is that optimal
algorithms are infeasible for read alignment, especially when reads are long. All
production read aligners following the approximate seed-extend
paradigm~\cite{alser2021technology}\footnote{This study examines 107 aligners.}.

Informed search
\section*{Problem statement}
\addcontentsline{toc}{section}{\protect\numberline{}{Problem statement}}

% alignment types
\begin{figure}[t]  %\begin{floatingfigure}[l]{0.5\textwidth}
    \includegraphics[width=\textwidth]{alignment-types-thesis.pdf}
	\caption[Main alignment types]{Names, alignment and paths of main alignment
    types. Synonims of the alignment names are shown in grey. The alignments are
    simplistically shows without edits. Paths corresponding to the alignments
    are shown as paths on an alignment graph. All paths start at green nodes and
    finish at red nodes. With green background are shown the two alignment types
    considered in this thesis.}
    \label{fig:alignment-types}
\end{figure}

\paragraph{Sequence alignment}
One of the simplest operations on sequences is to align one to another. We
consider the problem of pairwise sequence alignment in the context of genomic
data: given two DNA sequences, one can be aligned to the other in multiple
ways~(\cref{fig:alignment-types})~\citep{navarro2001guided}. In this thesis we
focus on global and semi-global alignment types only.

%If both sequences have to
%be aligned end-to-end, we look for a \emph{global alignment} whose minimal
%number of edits is known as \emph{edit distance}. If we instead search for an
%occurance of a query sequence within a reference sequence, we allow the
%alignment to start and end at any reference locations in a \emph{semi-global}
%alignment.

\paragraph{Minimizing edits}
It would have been relatively easy to find alignments if the sequences did not
differ from each other. Nevertheless, the real data may be subject to such
processes as biological evolution and sequencing technologies, which result in
differences between compared sequences. Commonly, the most probable explaination
of these differences is to be reconstructed. If we assume an \emph{error model}
which repeatedly applies a random single-letter edit (substitutions, insertions
and deletions), then the most probable sequence of edits would be one with a
minimal number of edits (possibly weighting different types of edits with
different costs). Throughout this thesis we consider this most widespread
objective -- it provides a tradeoff between expressivity and computability -- it
reasonably estimates the real world sequences while disregarding any memory in
the error model.

\paragraph{Global alignment}
The simplest type of alignment is to find a sequence of edits that convert one
whole sequence $A$ to another sequence $B$. Note that converting $A$ to $B$
corresponds to the reverse process of converting $B$ to $A$ (e.g. insertions
correspond to deletions). If we ignore the order of applying the edits, the
result can be equivalently represented simpler as an \emph{alignmnet}: each
letter from $A$ has either not been edited (so it corresponded to a matching
letter in $B$), or has been substituted for another letter (so it corresponded
to a mismatching letter in $B$), or has been deleted (which can be represented
as mismatching it with an inserted fictive \emph{gap symbol} in $B$).
Additionally, letters could have been added to $A$ (represented as a gap in $A$
mismatching a letter in $B$). Note that the number of edits is equal to the
number of mismatches in the corresponding alignment. The minimal number of edits
necessary to transform $A$ to $B$ is known as \emph{Levenshtein distance} when
each edit operation is equally costly, and more generally as \emph{edit
distance}, when the edit operations are weighted depending on their type.
Clearly, finding an alignment with a minimal number of edits implies computing
the edit distance. It has recently been shown that it is unlikely that there
exists an algorithm that computes edit distance in strictly subquadratic time in
the worst case~\cref{backurs2015edit}.

\paragraph{Semi-global alignment}
Semi-global sequence alignment (also called approximate/fuzzy string search
outside computational biology) is the alignment of a whole \emph{query} sequence
to a continuous region (subsequence) of a \emph{reference} sequence. This region
is unspecified, so semi-global alignment is a more complex problem than global
alignment. We target semi-global alignment by again minimizing edit distance.
Finding an alignment implies also finding a reference region where the query
aligns, which is known as \emph{mapping}.

\paragraph{Semi-global alignments on the same reference}
If multiple queries have to be aligned on the same reference, powerful
optimizations are applicable.

\paragraph{Sequence-to-graph alignment}
A set of reference sequences can be compactly represented as paths in a graph
(possbly cyclic). \emph{Sequence-to-graph} alignment is a semi-global alignment
between a query sequence and a walk in a reference
graph~\cite{jain_complexity_2019}.

\paragraph{Alignment as shortest path}
An alignment of two sequences is a letter-to-letter correspondence between the
two sequences (using gap letters for insertions and deletions). Considering the
corresponding letters consecutively, each pair of letters proceeds to a longer
prefix of $A$ (in case of an insertion), to a longer prefix of $B$ (in case of a
deletion), or to longer prefixes of both (in case of matching or substituted
letters). If we define an \emph{alignment graph} to contain all pairs of
prefixes as nodes, and all edits as edges, then any alignment will correspond to
a path. Moreover, if the edges are weighted by the edit costs, any alignment
with a minimal edit distance will correspond to a shortest path. Thus, we will
find best alignments as shortest paths in an alignment graph.
\section*{Contributions}
\addcontentsline{toc}{section}{\protect\numberline{}Main results}

\dictum{%
   Provably optimal and heuristically fast.}
\vskip 1em

\begin{figure}[h]
  \includegraphics[width=1.0\linewidth]{media/ownpubs-table.png}
  \caption{Overview of the publications.}
  \label{tab:ownpubs}
\end{figure}

We have effectively applied the \A algorithm to optimal sequence alignment. We
demontrated that the additional information from the whole sequence can improve
the scaling with query length, reference size and error rate, substantially
decrease the necessary computations, and result in algorithms that are orders of
magnitude faster than existing optimal algorithms. We apply the \A approach to
two types of alignment: semi-global (alignment) and global.

\subsection*{Principled approach by shortest paths}
shortest paths,

A*, admissibility

\subsection*{Optimality guarantees}
To ensure that our algorithms are practical, we introduce a number of
algorithmic optimizations which increase performance and decrease memory
footprint.

We also prove that all optimizations (greedy matching, ) are preserving the optimality.

\subsection*{Scaling with reference size using a trie index}

%\section{Reconceptualizing seeds for optimal alignment}
\section{Beyond \emph{seed-(chain)-extend} paradigm}

As we saw in \cref{ch:trie,ch:seed}, all optimal read aligners compute the whole
dynamic programming table, thus reaching the prohibitively slow quadratic
runtime. On the other side, all current production aligners rely on the
\emph{seed-extend} paradigm (and its \emph{seed-chain-extend} variants for long
reads).

This paradigm requires similar short \emph{seed} patches to be found
between the sequences (\eg by hashed kmers, minimiziers, maximum exact matching,
etc.), and then to \emph{extend} the alignment of the whole query around these
\emph{seeded} similar patches. This is a very intuitiv approach if the goal is
to find a \emph{good alignment}.

If we instead seek not good but provably \emph{best} alignments, we are required
to at least implicitly refute all the exponentially-many competing alignments.

Instead, to find optimal alignments, we do not need to choose the seeds to be
long and similar with the reference buare not required to be similar.

\begin{observation}[Seeds without matches]
    To efficiently find an optimal alignment using \A with the seed heuristic,
    seeds are not required to match (even on the resulting alignment).
\end{observation}

Nevertheless, each seed can penalize potential alignment by not more than its
\emph{potential} (\ie the number of plus $1$, for the case of exact matching
with unit costs). Any additional errors will require more states to be expanded.

This is an interesting observation was made by Ragnar while playing with the
seed heursitic. It looks Indeed, of finding a good alignment but to prove that all
alternative alignments are no better, the seed heuristic for \A search does not
really need matches to be efficient.

This novel usage of seeds carrie different problems and different possibilities.
\section{Scaling analysis}

Classically, the runtime and memory usage of algorithms has been analysed for a
worst case asymptotic behavior. Since the near quadratic worst case asymptotics
is likely to be tight, we target related sequences (with limited error rates),
and empyrically the runtime and memory scaling.

The seed heuristic, which is in the core of our \A algorithm, is admissible
(optimistic) but not consistent (monotone). As a consequence, it guarantees
finding a best alignment but the same state could be expanded more than once
(and this indeed happens). Theoretically, this also implies that the worst
case asymptotics in not anymore bounded by the quadratic number of states.

Nevertheless, in practice

we employ a best-fit estimation of the scaling which
is not asymptotic. Tools

Theory

exponential scaling

\subsection*{Scaling with query length using a seed heuristic}

\subsection*{Scaling with error rate using inexact matching and chaining}

Informed search: Two-stage algorithm, similar to Aho-Corasick, increasingly more information (length, prefix, seeds, chaining seeds, chaining seeds + gaps)

\subsection*{Aligning to general graph references}

\subsection*{Implementations}


\paragraph{Scaling with reference size}
In~\cref{ch:trie} we present the tool \astarix which applies the \A algorithm to
find optimal alignments, based on a domain-specific heuristic and enhanced by
multiple algorithmic optimizations. Importantly, our approach allows for both
cyclic and acyclic graphs including variation and de Bruijn graphs. We
demonstrate that using a trie index we can achieve sublinear scaling of aligning
runtime with reference size, and that \A can scale exponentially better than
\dijkstra with increasing (but small) number of errors in the reads. Moreover,
for short reads, both \astarix and \dijkstra scale better and outperform current
state-of-the-art optimal aligners with increasing genome graph size.
Nevertheless, scaling optimal alignment of long reads on big graphs remained an
open problem.

\paragraph{Scaling with reference size}
In~\cref{ch:seed} we upgrade \astarix with a novel \sh which guides the \A
search by preferring crumbs on nodes that lead towards optimal alignments even
for long reads. This approach enables the scaling of semi-global alignment with
read length.

In~\cref{ch:global} we resolve the third major bottleneck -- handling high error
rates. We presented an algorithm with an implementation in \astarpa solving
pairwise alignment between two sequences. The algorithm is based on \A with a
\sh, inexact matching, match chaining, and match pruning, which we proved to
find an exact solution according to edit distance. For random sequences with up
to $15\%$ uniform errors, the runtime of \astarpa scales near-linearly to very
long sequences ($10^7\bp$) and outperforms other exact aligners. We demonstrate
that on real ONT reads from a human genome, \astarpa is faster than other
aligners on only a limited portion of the reads.
\section{Preliminaries}

\subsection{Alignment as shortest path}

Alignment is equivalent to shortest path.

\paragraph{\dijkstra and \A}
\dijkstra's algorithm \citep{dijkstra1959note} finds a shortest path from $v_s$
to~$v_t$ by \emph{expanding} vertices in order of increasing distance $\g(u)$
from the start. The \A algorithm~\citep{hart1968formal,pearl1984heuristics},
instead, directs the search towards a target by expanding vertices in order of
increasing ${f(u) := g(u) + h(u)}$, where $h(u)$ is a heuristic function that
estimates the distance $\h(u)$ to the end and $g(u)$ is the shortest length of a
path from $v_s$ to $u$ found so far. A heuristic is \emph{admissible} if it is a
lower bound on the remaining distance, $h(u) \leq \h(u)$, which guarantees that
\A has found a shortest path as soon as it expands $v_t$. Heuristic $h_1$
\emph{dominates} another heuristic $h_2$ when $h_1(u) \ge h_2(u)$ for all vertices $u$.
A dominant heuristic will usually, but not always~\citep{holte2010common},
expand less vertices. Note that \dijkstra's algorithm is
equivalent to \A using a heuristic that is always $0$, and that both algorithms
require non-negative edge costs. Our variant of the \A algorithm is provided
in~\cref{GLOBALsec:astar}.

\subsection{\A algorithm and its heuristic function} \label{sec:astar}

% paper: trie
%\subsection{Background: General \A algorithm} \label{TRIEsubsec:general-astar}
Given a weighted graph $G=(V,E)$ with $E \subseteq V \times V \times
\mathbb{R}_{\geq 0}$, the \A algorithm (abbreviated as \A) searches for the
shortest path from sources $S \subseteq V$ to targets $T \subseteq V$. It is an
extension of \dijkstra's algorithm that additionally leverages a \emph{heuristic
function} $h \colon V \to \mathbb{R}_{\geq 0}$ to decide which paths to explore
first.
%
If $h(u) \equiv 0$, \A is equivalent to \dijkstra's algorithm.
%
You can refer to the \A and \dijkstra algorithms in \cref{alg:astar}, but do not
assume knowledge of either algorithm in the following.
%
At a high level, \A maintains the set of all \emph{explored} states, initialized
with the set of sources $S$. Then, \A iteratively \emph{expands} the explored
state with lowest estimated cost by exploring all its neighbors, until it finds
a target. Here, the cost for node $u$ is estimated by the distance from source, called $g(u)$, plus the estimate from the heuristic $h(u)$.

\paragraph{Heuristic Function}
The heuristic function $h(u)$ estimates the
cost $h^*(u)$ of a shortest path in $G$ from $u$ to a target $t \in T$. Intuitively, a
good heuristic correlates well with the distance from $u$ to $t$.

To ensure that \A indeed finds the shortest path, $h$ should be
\emph{admissible}:

\begin{definition}[Admissible heuristic] A heuristic $h$ is \emph{admissible}
    if it provides a lower bound on the distance to the closest target: $\forall
    u. h(u) \leq h^*(u)$.
\end{definition}

While any admissible $h$ ensures that \A finds optimal
alignments~\cite{dechter_generalized_1985}, the specific choice of $h$
is critical for performance. In particular, decreasing the error $\delta(u) =
h^*(u)-h(u)$ can only improve the performance of
\A~\cite{dechter_generalized_1985}. Thus, a key contribution of ours is
a domain-specific heuristic $h$.


\paragraph{\A algorithm}
We aim to guarantee optimal alignment while optimizing the average runtime
to not reach its worst-case complexity. While \dijkstra is an algorithm that
explores graph nodes in the order of their distance from the start, \A is a
generalization of \dijkstra that also accounts for their distance from the
target. \A prioritizes the exploration of nodes that seem to be closer to the
target nodes. This way, \A can sometimes dramatically improve on the performance
of \dijkstra while remaining optimal.

There has been one attempt to apply \A for optimal
alignment~\cite{dox2018efficient} which uses a heuristic function that accounts
only for the length of the remaining query sequence to be aligned. However, it
does not significantly outperform \dijkstra (in fact, it is equivalent for
a zero matching cost).
%
In contrast, the heuristic function we introduce is more informative and
consistently outperforms \dijkstra.

\cref{alg:astar} shows a generic implementation of the \A algorithm,
roughly following~\cite{dechter_generalized_1985}.
We do not implement the reconstruction of the best alignment in order to simplify the presentation.
The procedure \mbox{\textsc{BacktrackPath}} traces the best alignment back to the $source$, based on remembered edges used to optimize $f$ for each alignment state.
%
\cref{alg:astar} also shows a simple implementation of \dijkstra in terms of \A.
We omit the implementation of \textsc{BacktrackPath} for simplicity.

\begin{algorithm}[t]
	\caption{\A~algorithm} \label{alg:astar}
	\begin{algorithmic}[1]
		\Function{\A}{$G\colon \text{Graph}$,
			$S\colon \text{Sources}$,
			$T\colon \text{Targets}$,
			$h\colon \text{Heuristic function}$}
		\State $g \gets \mli{Map}\colon (\text{Nodes} \to \mathbb{R}_{\geq 0})$
		\Comment Shortest paths lengths to explored nodes

		\State $f \gets \mli{Map}\colon (\text{Nodes} \to \mathbb{R}_{\geq 0})$
		\Comment $f(u)=g(u)+h(u)$ 

		\State $Q \gets \mli{MinPriorityQueue}(\mli{priority}=f)$ 
		\Comment Priorities according to $f$
		\ForAll{$s \in S$}
			\State $g[s] \gets 0.0,\, f[s] \gets 0.0$
			\State $Q.\mli{push}(s)$
			\Comment Initially, explore all $s \in S$
		\EndFor
		\While{$Q \neq \emptyset$}
			\State $\mli{curr} \gets Q.\mli{pop}()$
			\Comment Get state with minimal $f$ to be expanded
			\If{$\mli{curr} \in T$}
				\State \Return \Call{BacktrackPath}{$\mli{curr}$}
				\Comment Reconstruct a walk to $\mli{curr}$
			\EndIf
			\ForAll{$(\mli{curr},\mli{next},\mli{cost}) \in G.\mli{outgoingEdges}(\mli{curr})$}
			\State $g_\mli{next} \gets g[\mli{curr}] + \mli{cost}$
			\State $\hat{f}_\mli{next} \gets g_\mli{next} + h(\mli{next})$
				\Comment Candidate value for $f[\mli{next}]$
				\If{$\hat{f}_\mli{next} < f[\mli{next}{}]$}
					\State $g[\mli{next}] \gets g_\mli{next}$		
					\State $f[\mli{next}] \gets \hat{f}_\mli{next}$		
					\State $Q.\mli{push}(\mli{next})$
					\Comment Explore state $\mli{next}$
				\EndIf
		\EndFor
		\EndWhile
		\State \textbf{assert} $\mli{False}$
		\Comment Cannot happen if $T$ is reachable from $S$
		\EndFunction

		\Statex

		\Function{\dijkstra}{$G\colon \mli{Graph}$,
			$S\colon \mli{Sources}$,
			$T\colon \mli{Targets}$}
			\State $h(v) \gets 0.0$
			\Comment Constant-zero function $h$
			\State $\Call{\A}{G,S,T,h}$
		\EndFunction
	\end{algorithmic}
\end{algorithm}

% paper: seeds

%\subsection{\A~algorithm for finding a shortest path} \label{SEEDsec:astar}
%
The \A~algorithm is a shortest path algorithm that generalizes \dijkstra's
algorithm by directing the search towards the target.
Given a weighted graph $G=(V,E)$, the \A~algorithm finds a shortest path from
sources $S \subseteq V$ to targets $T \subseteq V$.
%
To prioritize paths that lead to a target, it relies on an admissible heuristic
function $h \colon V \to \mathbb{R}_{\geq 0}$, where $h(v)$ estimates the
remaining length of the shortest path from a given node $v \in V$ to a target
$t \in T$.


\paragraph{Algorithm}
% 
In a nutshell, the \A~algorithm maintains a set of \emph{explored} nodes,
initialized by all possible starting nodes $S$. It then iteratively
\emph{expands} the explored state $v$ with lowest estimated total cost $f(v)$ by
exploring all its neighbors. Here, $f(v) := g(v) + h(v)$, where $g(v)$ is the
distance from $s \in S$ to $v$, and $h(v)$ is the estimated distance from $v$ to
$t \in T$.
%
When the \A~algorithm expands a target node $t \in T$, it reconstructs the path
leading to $t$ and returns it.
%
\paragraph{Admissibility}
%
The \A~algorithm is guaranteed to find a shortest path if its heuristic $h$
provides a lower bound on the distance to the closest target, often referred to
as $h$ being \emph{admissible} or optimistic.

Further, the performance of the \A~algorithm relies critically on the choice of
$h$. Specifically, it is crucial to have low estimates for the optimal paths but
also to have high estimates for suboptimal paths.

\paragraph{Discussion}
%
To summarize, we use the \A~algorithm to find a shortest path from $\st{u}{0}$
to $\st{v}{|q|}$ in $\AG$. To guarantee optimality, its heuristic function
$h\st{v}{i}$ must provide a lower bound on the shortest distance from state
$\st{v}{i}$ to a terminal state of the form $\st{w}{\lvert q \rvert}$.
%
Equivalently, $h\st{v}{i}$ should lower bound the minimal cost of aligning
$q[i{:}]$ to $\RG$ starting from $v$, where $q[i{:}]$ denotes the suffix of $q$
starting at position $i$ ($0$-indexed).
%
The key challenge is thus finding a heuristic that is not only admissible but
also yields favorable performance.

% paper:global
% Shortest paths, A* for MSA and semi-global alignment (AStarix), gaps
\paragraph{Shortest paths and \A}
A pairwise alignment that minimizes edit distance corresponds to a shortest path
in the \emph{alignment graph}~\citep{vintsyuk1968speech,ukkonen1985algorithms}.
Assuming non-negative edit costs, a shortest path can be found using \dijkstra's
algorithm~\citep{ukkonen1985algorithms} (\cref{GLOBALfig1-dij}) or
\A~\citep{spouge1989speeding}. \A is an informed search algorithm which uses a
task-specific heuristic function to direct its search. Depending on the
heuristic function, a shortest path may be found significantly faster than by an
uninformed search such as \dijkstra's algorithm.

% paper: global
\subsection{Seed heuristic}

\dictum{Cut off your nose to spite your face.}
\vskip 1em

Seed-and-extend is a commonly used paradigm for solving semi-global alignment
approximately~\citep{kucherov2019evolution}. Seeds are also used to define and
compute LCSk~\citep{benson2014longest}, a generalization of longest common
subsequence (LCS). In contrast, our \emph{\sh} speeds up finding an optimal
alignment by using seed matches to speed up the \A search. A limitation of the
existing \sh is the low tolerance to increasing error rates due to using only
long exact matches without accounting for their order.

Our seed heuristic uses seeds in a very different way than existing seeding
approaches: instead of searching for a good alignment around seed matches, it
punishes potential alignments by the lack of matches. This negation makes the
difference between finding a good alignment and proving that no other alignment
is better.

% paper-global
\paragraph{Paths, alignments, seeds and matches}
Any path from $\st ij$ to $\st{i'}{j'}$ in the alignment graph~$G$ represents a
\emph{pairwise alignment} (or just \emph{alignment}) of the substrings $A_{i
\dots i'}$ and $B_{j \dots j'}$. We denote with $d(u,v)$ the distance between
states $u$ and $v$. A shortest path $\pi^*$ corresponds to an optimal alignment,
thus $\cost(\pi^*) = d(v_s, v_t) = \ed(A, B)$. For a state $u$ we write $\g(u)
:= d(v_s, u)$ and $\h(u) := d(u, v_t)$ for the distance from the start to $u$
and from $u$ to the target $v_t$, respectively.

% paper-global
We outline algorithms for exact pairwise alignment and their fastest implementations for
biological sequences. Refer to~\citet{kucherov2019evolution} for approximate,
probabilistic, and non-edit distance algorithms and aligners.

\paragraph{Banding and bit-parallelization} When similar sequences are being
aligned, the whole DP table may not need to be computed. One such
output-sensitive algorithm is the \emph{banded} algorithm of
\citet{ukkonen1985algorithms} (\cref{GLOBALfig1-band}) which considers only states
near the diagonal within an exponentially increasing \emph{band}, and runs in
$\Oh(ns)$ time, where $s$ is the edit distance between the sequences. This
algorithm, combined with the \emph{bit-parallel optimization}
by~\citet{myers1999fast} is implemented by the \edlib
aligner~\citep{vsovsic2017edlib} that runs in $\Oh(ns/w)$ runtime, where $w$ is
the machine word size (nowadays 32 or 64).

\paragraph{Diagonal transition and WFA}
The $\Oh(ns)$ runtime complexity can be improved using the algorithm independently
discovered by \citet{ukkonen1985algorithms} and \citet{myers1986ano} that is
known as \emph{diagonal transition} \citep{navarro2001guided} (\cref{GLOBALfig1-wfa}).
It has an $\Oh(ns)$ runtime in the worst-case but only takes expected
$\Oh(n+s^2)$ time under the assumption that the input sequences are
random~\citep{myers1986ano}. This algorithm has been extended to linear and
affine costs in the \emph{wavefront alignment} (WFA)
algorithm~\citep{marco2021fast} in a way similar to~\citet{gotoh1982improved},
and has been improved to only require linear memory in
\wfa~\citep{marco2022optimal} by combining it with the \emph{divide and conquer}
approach of~\citet{hirschberg1975linear}, similar to \citet{myers1986ano}
algorithm for unit edit costs (\cref{GLOBALfig1-biwfa}).
% This new name, WFA, has since replaced the original \emph{diagonal transition}
% name and refers to both the unit cost and affine variants of the
% algorithm as well as the implementation.
Note that when each sequence letter has an error with a constant probability,
the total number of errors $s$ is proportional to $n$, so that even these
algorithms have a quadratic runtime.

\paragraph{Chains}
A state $u=\st ij\in V$ \emph{precedes} a state $v=\st {i'}{j'}\in V$, denoted
$u\preceq v$, when $i\leq i'$ and $j\leq j'$. Similarly, a match $m$ precedes a
match $m'$, denoted $m\preceq m'$, when the end of $m$ precedes the start of
$m'$. This makes the set of matches a partially ordered set.
A state $u$ precedes a match $m$ (denoted $u\preceq m$) when it precedes
the start of the match. A \emph{chain} of matches is a (possibly empty) sequence
of matches $m_1 \preceq \dots \preceq m_l$.

\paragraph{Contours}
To efficiently calculate maximal chains of matches, \emph{contours} are used.
Given a set of matches $\matches$, $\statescore(u)$ is the number of matches in
the longest chain $u\preceq m_0 \preceq \dots$, starting at $u$. The function
$\statescore\st ij$ is non-increasing in both $i$ and $j$. \emph{Contours} are
the boundaries between regions of states with $\statescore(u) = \ell$ and
$\statescore(u)<\ell$ (see~\cref{GLOBALfig:contours}). Note that contour $\ell$ is
completely determined by the set of matches $m\in \matches$ for which
${\statescore(\start(m)) = \ell}$~\citep{hirschberg1977algorithms}.

\citet{hunt1977fast} give an algorithm to efficiently compute $\statescore$ when
$\matches$ is the set of single-letter matches between $A$ and $B$, and
\citet{deorowicz2014efficient} give an algorithm when $\matches$ is the set of
$k$-mer exact matches.

% Output-sensitive overview
%\paragraph{Relaxations} Approximate
%algorithms~\citep{kucherov2019evolution}, an $\Oh(n \log n)$ algorithm that is
%exact with high probability assuming a limited error rate
%$e<3.485\%$~\citep[Proof of Lemma~25, p.~17]{ganesh2020near}. The
%asymptotically best know algorithm reaches
%$\Oh(n^2/\log^2n)$~\citep{masek1980faster} but is not practical.

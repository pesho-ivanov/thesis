\graphicspath{{\dir/}}

%keywords: Next-generation sequencing, Genome graph, Edit distance, Semi-global
%alignment, Global alignment, Long reads, Optimal alignment, Shortest path, \A
%algorithm, Seed heuristic

\chapter{Introduction} \label{ch:introduction}

\paragraph{Motivation}
Sequence alignment has been a core problem in computational biology for the last
half-century. It is an open problem whether exact pairwise alignment is possible
in linear time for related sequences~\citep{medvedev2022theoretical}.

An algorithm without an objective function may be wrong because they do not
solve the correct problem. or because they 

Algorithms that guarantee correctness can be wrong only by being given a wrong problem.

An approximate algorithm can be wrong either because it did not fullfill its
mathematical goal. reach its solve the problem because of either optimizing the
\emph{wrong} function.

Algorithm correctness is arguably a useful property which is often not simple to
guarantee. It can undoubtedly improve accuracy, especially in the case of
complex data, but still be wrong from biological point of view. This is because
of  This Nevertheless, since biology is a natural science, its  optimality
guarantees must have an additional impact on the development of the field. It
not only but allows to enjoy being wrong rather than vague. Moreove, often
problems in computational biology are ill-stated andalgorithms that approximate
algorithms that. But many algorithms in computational biology do not even have a
formal statement 

% Our scope
Formally, we consider the optimal \emph{sequence-to-graph alignment} problem,
the task of finding an optimal base-to-base correspondence between a query
sequence and a (possibly cyclic) walk in the graph. Related alignment problems
have already been formulated as graph shortest path
problems~\cite{jain_complexity_2019}.

\section{Motivation}

Observation 1: A long query has more information that may hint towards the best
mapping place. Unlike the practical approximate algorithms, the existing optimal
algorithm do not exploit this information. As a result, all state-of-the-art
optimal algorithms take longer to map a longer query to a reference. 

Observation 2: The input is linear, the output is linear, the best case is
linear. On the other hand all optimal solutions are near-quadratic and the
theoretical limit is near-quadratic. To spead up beyond this near-quadratic
barrier, current practical algorithms break the optimality guarantee hoping that
the produced alignments are accurate enough. 

%But it is possible to be fast and accurate in the same time?
We take another approach: we preserve the optimality guarantee and use
substantially more information with the hope to be fast enough. As it turns out,
when the error rate is limited, our optimal solutions empirically scale
near-linearly up to very long sequences. This translates to many orders of
magnitude of runtime speedup compared to state-of-the-art optimal aligners.

% Problem, applications
The problem of aligning one biological sequence to another has been formulated
over half a century ago~\citep{needleman1970general} and is known as
\emph{global pairwise alignment}~\citep{navarro2001guided}. Pairwise alignment
has numerous applications in computational biology, such as genome assembly,
read mapping, variant detection, multiple sequence alignment, and differential
expression~\citep{prjibelski2018sequence}. Despite the centrality and age of
pairwise alignment, ``a major open problem is to implement an algorithm with linear-like
empirical scaling on inputs where the edit distance is linear
in~$n$''~\citep{medvedev2022theoretical}.

% Near-quadratic worst case
Alignment accuracy affects the subsequent analyses, so a common goal
is to find a shortest sequence of edit operations (insertions, deletions, and
substitutions of single letters) that transforms one sequence into the other.
Finding such a sequence of operations is at least as hard as computing the \emph{edit
distance}, which has recently been proven to not be computable in strongly
subquadratic time, unless SETH is false~\citep{backurs2015edit}. Given that
the number of sequencing errors is proportional to the length, existing exact aligners are
limited by quadratic scaling not only in the worst case but also in practice.
This is a computational bottleneck given the growing amounts of biological data
and the increasing sequence lengths~\citep{kucherov2019evolution}.

%\subsection{Knowledge gap}
%\dictum{%
%   Provably optimal and heuristically fast.}
%\vskip 1em

Sequence alignment is a class of combinatorial problems that is of primary
importance for analysis of genetic data. Algorithms and tools for alignment have
been thoroughly developed and routinely used for genome assembly, RNA
quantification, detecting splicing, oncology, multiple sequence alignment (MSA),
and evolutionary biology. Types of sequence alignment include global alignment,
semi-global alignment, mapping, local alignment, and others. For each type, a
common tradeoff that had to be done is between the alignment accuracy and the
performance to find it.

% graph reference
An additional difficulty is the fact that in the
upcoming pangenomic era, these algorithms must be also applicable to complex
graph structures. For more than 60 years, a linear sequence has been extremely
useful as a representation of a single genome. The affordability of sequencing
technologies enables not only to sequence genomes deeper but also to sequence
many genomes (e.g. of organisms or single cells), building a pangenome (an
abstracted genome that represents the genetic variation of a whole clade). The
shift towards population studies in the last decade motivates the adoption of
graph data structures which serve as compressed representations of collections
of related genomes (genomes are paths in the graph).

% \A context 
An optimal alignment can naturally be represented as a shortest path in an
alignment graph (equivalent to the DP table). In order to find such a shortest
path with minimal exploration, we instantiate the \A algorithm with a novel
heuristic function based on the unaligned parts of the sequences. This
additional information is a problem-specific heuristic function and it heavily
determines the efficiency of the search. For any explored state by \A, this
heuristic function should compute a lower bound on the remaining path length, or
more specifically, the minimal cost of edit operations needed to align the
remaining sequences.

\paragraph{Main contributions}
We have effectively applied the \A algorithm to optimal sequence alignment. We
demontrated that the additional information from the whole sequence can improve
the scaling with query length, reference size and error rate, substantially
decrease the necessary computations, and result in algorithms that are orders of
magnitude faster than existing optimal algorithms. We apply the \A approach to
two types of alignment: semi-global (mapping) and global.


% from seeds paper

% General: aligning, edit distance
Alignment of reads to a reference genome is an essential and early step in most
bioinformatics pipelines. While linear references have been used traditionally,
an increasing interest is directed towards graph references capable of
representing biological variation~\citep{garrison_variation_2018}.
%
Specifically, a \emph{sequence-to-graph} alignment is a base-to-base
correspondence between a given read and a walk in the graph. As sequencing
errors and biological variation result in inexact read alignments, edit distance
is the most common metric that alignment algorithms optimize in order to find
the most probable read origin in the reference.

% We note that in contrast to linear references, reference graphs capture
% genomic variation and therefore enable more accurate
% alignments~\citep{garrison_variation_2018}.

\paragraph{Suboptimal alignment}
%
In the last decades, approximate and alignment-free methods satisfied the demand
for faster algorithms which process huge volumes of genetic
data~\citep{kucherov2019evolution}. 
%
\emph{Seed-and-extend} is arguably the most popular paradigm in read
alignment~\citep{altschul_basic_1990,langmead_fast_2012,li_fast_2009}. First,
substrings (called \emph{seeds} or \emph{kmers}) of the read are extracted, then
aligned to the reference, and finally prospective matching locations are
\emph{extended} on both sides to align the full read.

While such a heuristic may produce acceptable alignments in many cases, it
fundamentally does not provide quality guarantees, resulting in suboptimal
alignment accuracy.
%
In contrast, here we demonstrate that seeds can benefit optimal alignment as
well.

\paragraph{Key challenges in optimal alignment}
%
Finding optimal alignments is desirable but expensive in the worst case,
requiring $\Oh(Nm)$ time~\citep{equi2019complexity}, for graph size $N$ and read
length $m$.
%
Unfortunately, most optimal sequence-to-graph aligners rely on dynamic
programming (DP) and always reach this worst-case asymptotic runtime. Such
aligners include \vargas~\citep{darby2020vargas},
\pasgal~\citep{jain_accelerating_2019},
\graphaligner~\citep{rautiainen_bitparallel_2019},
\hga~\citep{feng2021accelerating}, and \vg~\citep{garrison_variation_2018},
which use bit-level optimizations and parallelization to increase their
throughput.

In contrast, we follow the promising direction of using a heuristic to avoid
worst-case runtime on realistic data. To this end, \astarix rephrases the task
of alignment as a shortest-path problem in an \emph{alignment graph} extended by
a \emph{trie index}, and solves it using the \A~algorithm instantiated with a
problem-specific \prefixh. Importantly, its choice of heuristic only affects
performance, not optimality.
%
Unlike DP-based algorithms, this \prefixh allows scaling sublinearly with the
reference size, substantially increasing performance on large genomes. However,
it can only efficiently align reads of limited length.

%only optimal alignment in this thesis
%asymptotics analysis
%????? linear I/O but quadratic optimal

\section{Feasibility of optimal alignment}

There is a fundamental trade-off between perforamance and optimality guarantees:
an algorithm which is allowed to be suboptimal may exploit the lesser
restrictions for greater performance. Especially given the worst-case analysis
requiring near-quadratic runtime even to compute edit distance exactly, it is
understandable why most scholars are skeptical about faster optimal algorithms.
With our \A approach  offers a to exploits another dimensions: average case or expected case analysis.

% asymptotics
%has been solved in linear time in \citeyear{morris1970linear}~\cite{morris1970linear}.

In the direction of global alignment, optimal algorithms are commonly used in
practice, despite of their quadratic scaling. The ongoing competition between
the optimal aligners employs both algorithmic advancements and implementation
optimizations on caching, bit-parallelization, GPU.~(\cref{ch:global}).

For semi-global alignment (read mapping), common believe is that optimal
algorithms are infeasible for read mapping, especially when reads are long. All
production read mappers following the approximate seed-extend
paradigm~\cite{alser2021technology}\footnote{This study examines 107 aligners.}.

\section{Preliminaries}

\subsection{Alignment as shortest path}

Alignment is equivalent to shortest path.

\paragraph{\dijkstra and \A}
\dijkstra's algorithm \citep{dijkstra1959note} finds a shortest path from $v_s$
to~$v_t$ by \emph{expanding} vertices in order of increasing distance $\g(u)$
from the start. The \A algorithm~\citep{hart1968formal,pearl1984heuristics},
instead, directs the search towards a target by expanding vertices in order of
increasing ${f(u) := g(u) + h(u)}$, where $h(u)$ is a heuristic function that
estimates the distance $\h(u)$ to the end and $g(u)$ is the shortest length of a
path from $v_s$ to $u$ found so far. A heuristic is \emph{admissible} if it is a
lower bound on the remaining distance, $h(u) \leq \h(u)$, which guarantees that
\A has found a shortest path as soon as it expands $v_t$. Heuristic $h_1$
\emph{dominates} another heuristic $h_2$ when $h_1(u) \ge h_2(u)$ for all vertices $u$.
A dominant heuristic will usually, but not always~\citep{holte2010common},
expand less vertices. Note that \dijkstra's algorithm is
equivalent to \A using a heuristic that is always $0$, and that both algorithms
require non-negative edge costs. Our variant of the \A algorithm is provided
in~\cref{GLOBALsec:astar}.

\subsection{\A algorithm and its heuristic function} \label{sec:astar}

% paper: trie
%\subsection{Background: General \A algorithm} \label{TRIEsubsec:general-astar}
Given a weighted graph $G=(V,E)$ with $E \subseteq V \times V \times
\mathbb{R}_{\geq 0}$, the \A algorithm (abbreviated as \A) searches for the
shortest path from sources $S \subseteq V$ to targets $T \subseteq V$. It is an
extension of \dijkstra's algorithm that additionally leverages a \emph{heuristic
function} $h \colon V \to \mathbb{R}_{\geq 0}$ to decide which paths to explore
first.
%
If $h(u) \equiv 0$, \A is equivalent to \dijkstra's algorithm.
%
You can refer to the \A and \dijkstra algorithms in \cref{alg:astar}, but do not
assume knowledge of either algorithm in the following.
%
At a high level, \A maintains the set of all \emph{explored} states, initialized
with the set of sources $S$. Then, \A iteratively \emph{expands} the explored
state with lowest estimated cost by exploring all its neighbors, until it finds
a target. Here, the cost for node $u$ is estimated by the distance from source, called $g(u)$, plus the estimate from the heuristic $h(u)$.

\paragraph{Heuristic Function}
The heuristic function $h(u)$ estimates the
cost $h^*(u)$ of a shortest path in $G$ from $u$ to a target $t \in T$. Intuitively, a
good heuristic correlates well with the distance from $u$ to $t$.

To ensure that \A indeed finds the shortest path, $h$ should be
\emph{admissible}:

\begin{definition}[Admissible heuristic] A heuristic $h$ is \emph{admissible}
    if it provides a lower bound on the distance to the closest target: $\forall
    u. h(u) \leq h^*(u)$.
\end{definition}

While any admissible $h$ ensures that \A finds optimal
alignments~\cite{dechter_generalized_1985}, the specific choice of $h$
is critical for performance. In particular, decreasing the error $\delta(u) =
h^*(u)-h(u)$ can only improve the performance of
\A~\cite{dechter_generalized_1985}. Thus, a key contribution of ours is
a domain-specific heuristic $h$.


\paragraph{\A algorithm}
We aim to guarantee optimal alignment while optimizing the average runtime
to not reach its worst-case complexity. While \dijkstra is an algorithm that
explores graph nodes in the order of their distance from the start, \A is a
generalization of \dijkstra that also accounts for their distance from the
target. \A prioritizes the exploration of nodes that seem to be closer to the
target nodes. This way, \A can sometimes dramatically improve on the performance
of \dijkstra while remaining optimal.

There has been one attempt to apply \A for optimal
alignment~\cite{dox2018efficient} which uses a heuristic function that accounts
only for the length of the remaining query sequence to be aligned. However, it
does not significantly outperform \dijkstra (in fact, it is equivalent for
a zero matching cost).
%
In contrast, the heuristic function we introduce is more informative and
consistently outperforms \dijkstra.

\cref{alg:astar} shows a generic implementation of the \A algorithm,
roughly following~\cite{dechter_generalized_1985}.
We do not implement the reconstruction of the best alignment in order to simplify the presentation.
The procedure \mbox{\textsc{BacktrackPath}} traces the best alignment back to the $source$, based on remembered edges used to optimize $f$ for each alignment state.
%
\cref{alg:astar} also shows a simple implementation of \dijkstra in terms of \A.
We omit the implementation of \textsc{BacktrackPath} for simplicity.

\begin{algorithm}[t]
	\caption{\A~algorithm} \label{alg:astar}
	\begin{algorithmic}[1]
		\Function{\A}{$G\colon \text{Graph}$,
			$S\colon \text{Sources}$,
			$T\colon \text{Targets}$,
			$h\colon \text{Heuristic function}$}
		\State $g \gets \mli{Map}\colon (\text{Nodes} \to \mathbb{R}_{\geq 0})$
		\Comment Shortest paths lengths to explored nodes

		\State $f \gets \mli{Map}\colon (\text{Nodes} \to \mathbb{R}_{\geq 0})$
		\Comment $f(u)=g(u)+h(u)$ 

		\State $Q \gets \mli{MinPriorityQueue}(\mli{priority}=f)$ 
		\Comment Priorities according to $f$
		\ForAll{$s \in S$}
			\State $g[s] \gets 0.0,\, f[s] \gets 0.0$
			\State $Q.\mli{push}(s)$
			\Comment Initially, explore all $s \in S$
		\EndFor
		\While{$Q \neq \emptyset$}
			\State $\mli{curr} \gets Q.\mli{pop}()$
			\Comment Get state with minimal $f$ to be expanded
			\If{$\mli{curr} \in T$}
				\State \Return \Call{BacktrackPath}{$\mli{curr}$}
				\Comment Reconstruct a walk to $\mli{curr}$
			\EndIf
			\ForAll{$(\mli{curr},\mli{next},\mli{cost}) \in G.\mli{outgoingEdges}(\mli{curr})$}
			\State $g_\mli{next} \gets g[\mli{curr}] + \mli{cost}$
			\State $\hat{f}_\mli{next} \gets g_\mli{next} + h(\mli{next})$
				\Comment Candidate value for $f[\mli{next}]$
				\If{$\hat{f}_\mli{next} < f[\mli{next}{}]$}
					\State $g[\mli{next}] \gets g_\mli{next}$		
					\State $f[\mli{next}] \gets \hat{f}_\mli{next}$		
					\State $Q.\mli{push}(\mli{next})$
					\Comment Explore state $\mli{next}$
				\EndIf
		\EndFor
		\EndWhile
		\State \textbf{assert} $\mli{False}$
		\Comment Cannot happen if $T$ is reachable from $S$
		\EndFunction

		\Statex

		\Function{\dijkstra}{$G\colon \mli{Graph}$,
			$S\colon \mli{Sources}$,
			$T\colon \mli{Targets}$}
			\State $h(v) \gets 0.0$
			\Comment Constant-zero function $h$
			\State $\Call{\A}{G,S,T,h}$
		\EndFunction
	\end{algorithmic}
\end{algorithm}

% paper: seeds

%\subsection{\A~algorithm for finding a shortest path} \label{SEEDsec:astar}
%
The \A~algorithm is a shortest path algorithm that generalizes \dijkstra's
algorithm by directing the search towards the target.
Given a weighted graph $G=(V,E)$, the \A~algorithm finds a shortest path from
sources $S \subseteq V$ to targets $T \subseteq V$.
%
To prioritize paths that lead to a target, it relies on an admissible heuristic
function $h \colon V \to \mathbb{R}_{\geq 0}$, where $h(v)$ estimates the
remaining length of the shortest path from a given node $v \in V$ to a target
$t \in T$.


\paragraph{Algorithm}
% 
In a nutshell, the \A~algorithm maintains a set of \emph{explored} nodes,
initialized by all possible starting nodes $S$. It then iteratively
\emph{expands} the explored state $v$ with lowest estimated total cost $f(v)$ by
exploring all its neighbors. Here, $f(v) := g(v) + h(v)$, where $g(v)$ is the
distance from $s \in S$ to $v$, and $h(v)$ is the estimated distance from $v$ to
$t \in T$.
%
When the \A~algorithm expands a target node $t \in T$, it reconstructs the path
leading to $t$ and returns it.
%
\paragraph{Admissibility}
%
The \A~algorithm is guaranteed to find a shortest path if its heuristic $h$
provides a lower bound on the distance to the closest target, often referred to
as $h$ being \emph{admissible} or optimistic.

Further, the performance of the \A~algorithm relies critically on the choice of
$h$. Specifically, it is crucial to have low estimates for the optimal paths but
also to have high estimates for suboptimal paths.

\paragraph{Discussion}
%
To summarize, we use the \A~algorithm to find a shortest path from $\st{u}{0}$
to $\st{v}{|q|}$ in $\AG$. To guarantee optimality, its heuristic function
$h\st{v}{i}$ must provide a lower bound on the shortest distance from state
$\st{v}{i}$ to a terminal state of the form $\st{w}{\lvert q \rvert}$.
%
Equivalently, $h\st{v}{i}$ should lower bound the minimal cost of aligning
$q[i{:}]$ to $\RG$ starting from $v$, where $q[i{:}]$ denotes the suffix of $q$
starting at position $i$ ($0$-indexed).
%
The key challenge is thus finding a heuristic that is not only admissible but
also yields favorable performance.

% paper:global
% Shortest paths, A* for MSA and semi-global alignment (AStarix), gaps
\paragraph{Shortest paths and \A}
A pairwise alignment that minimizes edit distance corresponds to a shortest path
in the \emph{alignment graph}~\citep{vintsyuk1968speech,ukkonen1985algorithms}.
Assuming non-negative edit costs, a shortest path can be found using \dijkstra's
algorithm~\citep{ukkonen1985algorithms} (\cref{GLOBALfig1-dij}) or
\A~\citep{spouge1989speeding}. \A is an informed search algorithm which uses a
task-specific heuristic function to direct its search. Depending on the
heuristic function, a shortest path may be found significantly faster than by an
uninformed search such as \dijkstra's algorithm.

% paper: global
\subsection{Seed heuristic}

\dictum{Cut off your nose to spite your face.}
\vskip 1em

Seed-and-extend is a commonly used paradigm for solving semi-global alignment
approximately~\citep{kucherov2019evolution}. Seeds are also used to define and
compute LCSk~\citep{benson2014longest}, a generalization of longest common
subsequence (LCS). In contrast, our \emph{\sh} speeds up finding an optimal
alignment by using seed matches to speed up the \A search. A limitation of the
existing \sh is the low tolerance to increasing error rates due to using only
long exact matches without accounting for their order.

Our seed heuristic uses seeds in a very different way than existing seeding
approaches: instead of searching for a good alignment around seed matches, it
punishes potential alignments by the lack of matches. This negation makes the
difference between finding a good alignment and proving that no other alignment
is better.
\section{Development of the field}

It may come as a surprise that another approach to such a basic problem as
alignment appears 60 years after the problem was first efficiently solved. Here
we speculate about the possible reasons for it: focus on new technology and
data, not believing that optimal solutions could be efficient.

%combining several existing ideas new (\A, trie, seeds)

Most of the techniques this thesis builds upon have been known for many decades
and have also been heavily motivated by applications in molecular biology. The
global alignment problem exists since
\citeyear{vintsyuk1968speech}~\cite{vintsyuk1968speech,needleman1970general}.

The problem of approximate/fuzzy string search (semi-global alignment of one
query) has been efficiently approached a decade later, in
\citeyear{sellers1980theory}~\cite{sellers1980theory,smith1981identification}.
more recent and additionally to the issues with handling errors, a new dimension
of complexity appears by the need to locate the position in the reference. The
problem of searching for multiple sequencing has been approached
since\citeyear{pearson1988improved}~\cite{pearson1988improved} and became
central to read mapping of high-throughput sequencing. The specifics with
semi-global alignment requires a trie-like index which is well-used in the
fiels, and known since
\citeyear{thue1912gegenseitige}~\cite{thue1912gegenseitige} and used in
informatics since~\citeyear{de1959file}~\cite{de1959file}.

An important factor for the development of the sequencing algorithms is the
technological advancement~\cite{alser2021technology}. One possible reason is the
combination of massive new data and advancement in computation hardware allowing
for various kinds of parallelization. Thus the advancements with bit-parallel
algorithms (CITE Myears, Mikko).

A major importance for our apporach is the usage of seeds to build good
heuristic functions that drive the \A search. Seeds are cousins of kmers which
are popular in sequence alignment since de Bruijn Graphs were applied for genome
assembly. Further kmers are cousins of ngrams which are popular in computational
linguistics since .

% parallelization
% sketches
\section{Alignment problem}

% Sequencing and variant calling
The analysis and understanding of genetic variation encoded in the genome of an
organism lies at the center of computational biology and medicine. Variation is
usually identified through matching sequences obtained from DNA/RNA-sequencing
back to a reference (genome) sequence in the process of \emph{variant calling},
making the alignment task a core problem in sequence bioinformatics.

% paper:seeds
\subsection{Problem statement: Alignment as shortest path} \label{SEEDsec:task}
%
In the following, we formalize the task of optimally aligning a read to a
reference graph in terms of finding a shortest path in an \emph{alignment
graph}. Our discussion closely follows~\citep{ivanov2020astarix} and is in line
with~\citep{rautiainen_aligning_2017}.

\para{Reference graph}
%
A reference graph $\RG=(\RGV,\RGE)$ encodes a collection of references to be
considered when aligning a read. Its directed edges $\RGE \subseteq \RGV \times
\RGV \times \Sigma$ are labeled by nucleotide letters from $\Sigma =
\{\texttt{A},\texttt{C},\texttt{G},\texttt{T}\}$, hence any walk
$\reference{\pi}$ in $\RG$ spells a string $\sigma(\reference{\pi}) \in
\Sigma^*$.

An alignment of a read $q \in \Sigma^*$ to a reference graph $\RG$ consists of
(i)~a walk $\reference{\pi}$ in $\RG$ and (ii)~a sequence of edits (matches,
substitutions, deletions, and insertions) that transform
$\sigma(\reference{\pi})$ to $q$. An alignment is \emph{optimal} if it minimizes
the sum of edit costs for a given real-valued cost model $\cedits = (\cmatch,
\csubst,\cdel, \cins)$.
%
We assume that edit costs are non-negative---a pre-requisite for the correctness
of \A. Further, we assume that $\cmatch \leq \csubst, \cins, \cdel$---a
prerequisite for the correctness of our heuristic.

We note that our approach naturally works for cyclic reference graphs.

\begin{figure}[t]
	\begin{alignat*}{20}
		(
			&\langle&& u &,& i   &\rangle&,
			&\langle&  v &,& i+1 &\rangle&,
			&&q[i],
			&&\cmatch
		&)&\in \AGE
		&& \quad \text{ if } (u,v,\ell) \in \RGE, \ell = q[i] & \qquad \text{(match)}\\
		%
		(
			&\langle&& u &,& i   &\rangle&,
			&\langle&  v &,& i+1 &\rangle&,
			&&q[i],
			&&\csubst
		&) &\in \AGE
		&& \quad \text{ if } (u,v,\ell) \in \RGE, \ell \neq q[i] & \qquad \text{(substitution)}\\
		%
		(
			&\langle&& u &,& i &\rangle&,
			&\langle&  v &,& i &\rangle&,
			&&\varepsilon,
			&&\cdel
		&) &\in \AGE
		&& \quad \text{ if } (u,v,\ell) \in \RGE & \qquad \text{(deletion)}\\
		%
		(
			&\langle&& u &,& i   &\rangle&,
			&\langle&  u &,& i+1 &\rangle&,
			&&q[i],
			&&\cins
		&) &\in \AGE
		&& \quad & \qquad \text{(insertion)},
	\end{alignat*}
	\caption[Formal definition of alignment graph]{Formal definition of
	alignment graph edges $\AGE \subseteq \AGV[q] \times \AGV[q] \times
	\Sigma_\varepsilon \times \mathbb{R}_{\geq 0}$. Here, $u,v \in \RGV$, $0
	\leq i < |q|$, $\ell \in \Sigma$, and $\varepsilon$ represents the empty
	string, indicating that letter $\ell$ was deleted.}
	\label{SEEDfig:graph-edges}
\end{figure}

\para{Alignment graph}
%
In order to formalize optimal alignment as a shortest path finding problem, we
rely on an \emph{alignment graph} $\AG[q]=(\AGV[q],\AGE[q])$.
%
Its nodes $\AGV[q]$ are \emph{states} of the form $\langle v, i \rangle$, where
$v \in \RGV$ is a node in the reference graph and $i \in \{0, \dots, |q|\}$
corresponds to a position in the read $q$.
%
Its edges $\AGE[q]$ are selected such that any path $\alignment{}{\pi}$ in
$\AG[q]$ from $\langle u, 0 \rangle$ to $\langle v, i \rangle$ corresponds to an
alignment of the first $i$ letters of $q$ to $\RG$.
%
Further, the edges are weighted, which allows us to define an \emph{optimal
alignment} of a read $q \in \Sigma^*$ as a shortest path $\alignment{}{\pi}$ in
$\AG[q]$ from $\langle u, 0 \rangle$ to $\langle v, |q| \rangle$, for any $u, v
\in \RGV$.
%
\cref{SEEDfig:graph-edges} formally defines the edges $\AGE$.

% paper:global
\paragraph{Sequences}
The input sequences $A = \overline{a_0a_1\dots a_i \dots a_{n-1}}$ and $B =
\overline{b_0b_1 \dots b_j \dots b_{m-1}}$ are over an alphabet $\Sigma$ with
$4$ letters. We refer to substrings $\overline{a_i \dots a_{i'-1}}$ as
$\substr Ai{i'}$, to prefixes $\overline{a_0 \dots a_{i-1}}$ as $A_{<i}$, and to
suffixes $\overline{a_i \dots a_{n-1}}$ as $A_{\geq i}$. The \emph{edit
distance} $\ed(A,B)$ is the minimum number of insertions, deletions, and
substitutions of single letters needed to convert $A$ into $B$.

\paragraph{Edit graph}
Let \emph{state} $\st{i}{j}$ denote the subtask of aligning the prefix $A_{<i}$
to the prefix $B_{<j}$. The \emph{edit graph} (also called \emph{alignment
graph}) $G(V,E)$ is a weighted directed graph with vertices $V = \{\st ij \vert
{0\leq i \leq n}, {0\leq j\leq m}\}$ corresponding to all states, and edges
connecting tasks to subtasks: edge ${\st ij \to \st{i{+}1}{j{+}1}}$ has cost
$0$ if ${a_i = b_j}$ (match) and $1$ otherwise (substitution), and edges ${\st ij
\to \st{i{+}1}{j}}$ (deletion) and ${\st ij \to\st{i}{j{+}1}}$ (insertion) have cost
$1$. We denote the root state $\st 00$ by $v_s$ and the target state $\st nm$ by
$v_t$.
For brevity we write $f(\st ij)$ as $f \st ij$.
The edit graph is a natural representation of the alignment problem that
provides a base for all alignment algorithms.
\section{Contributions}

\paragraph{Tools}

\subsection{Informed search}
Two-stage algorithm, similar to Aho-Corasick, increasingly more information (length, prefix, seeds, chaining seeds, chaining seeds + gaps)
 
% Accuracy and Metrics
The number of possible alignments grow exponentially with length. The usual
underlying question to finding ``correct'' alignments. Regarding the precision
of alignment, one is usually interested in base-to-base (aka letter-to-letter)
correspondence between the sequences, even though for some applications a less
detailed solution is sufficient: only the similarity between sequences or the
location where a read maps to a reference. Exact alignment is only useful for
very short sequences (often kmers), and for all other cases the optimized metric
may be hamming distance, edit distance (unit costs), Levenshtein distance,
affine costs, convex and concave costs, general costs and others. 

% Problem statement
Depending on the the number of aligned sequences, there is pairwise alignment
and multiple sequence alignment (MSA). Depending on the parts of the sequences
that are aligned to each other, we differentiate global, local and various
semi-global alignemnts. There are generalizations to sequence-to-sequence
alignment, including aligning to nonlinear structures, such as directed acyclic
graphs, DAGs, general graphs and others. These structures are nowadays becoming
more common as a compressed form of representing a set of references to which a
sequence can be aligned. Often, one best alignment is suefficient but finding
several best (top-K) alignments. In the context of read mapping, a set of reads
is aligned to the same reference sequence so an indexing procedure is often
useful for the performance.

We specifically consider the mapping of a set of reads to a general graph, and
the global pairwise alignment.

Existing optimal algorithms are based on dynamic programming (DP) and
run in quadratic time (assuming that the number of errors is proportional to the
length)

we employ the \A algorithm which is an \emph{informed search} algorithm.
TODO: a case for the informed algorithms

\subsection{Scalable optimal alignments}

% Heuristics for alignment
Both for sequence-to-sequence alignment and sequence-to-graph alignment,
heuristics are employed to keep alignment
tractable~\cite{altschul_basic_1990,langmead_fast_2012,garrison_variation_2018},
especially for large populations of human-sized genomes.
%
% Importance of optimal alignment
While such heuristics find the correct alignment for simple references, they
often perform poorly in regions of very high complexity, such as in the human
major histocompatibility complex (MHC)~\cite{dilthey_improved_2015}, in complex
but rare genotypes arising from somatic-subclones in tumor sequencing
data~\cite{harismendy_detection_2011}, or in the presence of frequent sequencing
errors~\cite{salmela_lordec_2014}.
%
Importantly, these cases can be of specific clinical or biological interest, and
incorrect alignment can cause severe biases for downstream analyses. For
instance, the combination of high variability of MHC sequences in humans and
small differences between alleles~\cite{buhler_hla_2011} leads to a risk of
misclassifications due to suboptimal alignment. Guaranteeing optimal alignment
against all variations represented in a graph is a major step towards
alleviating those biases.

% Optimal DP-based approaches
\paragraph{Optimal Alignment}
Current optimal alignment algorithms reach the impractical $\Oh(nm)$ runtime
that has been shown to be a lower bound for the worst-case edit distance
computation~\cite{backurs2015edit}. In this light, approaches for improving the
efficiency of optimal alignment have taken advantage of specialized features of
modern CPUs to improve the practical runtime of the Smith-Waterman dynamic
programming (DP) algorithm~\cite{smith_comparison_1981} considering all possible
starting nodes. These use modern SIMD instructions (\eg
\vg~\cite{garrison_variation_2018} and \pasgal~\cite{jain_accelerating_2019}) or
reformulations of edit distance computation to allow for bit-parallel
computations in \graphaligner \footnote{We refer as \bitparallel to to the
bit-parallel DP algorithm implemented in \graphaligner tool
\cite{rautiainen_bitparallel_2019}.}~\cite{rautiainen_bitparallel_2019}. Many of
these, however, are designed only for specific types of genome graphs, such as
{\itshape de Bruijn}
graphs~\cite{liu_debga_2016,limasset2019toward} and
variation graphs~\cite{garrison_variation_2018}. A compromise often made when
aligning sequences to cyclic graphs using algorithms reliant on directed acyclic
graphs involves the computationally expensive ``DAG-ification'' of graph
regions~\cite{kavya_sequence_2019,garrison_variation_2018}.

%\section{Optimal alignment}
Finding an optimal alignment requires a conceptually different approach than
finding an approximate alignment. Instead of finding \emph{one} good alignment,
finding an optimal alignment requires proving that \emph{all} other
exponentially-many alignments are not better.

Comparing one sequence to another is a basic combinatorial problem that has
several variations (shown on the right), each applicable in computational
biology. Needleman-Wunsch (1970)  and Smith-Waterman (1981) are dynamic
programming (DP) algorithms that serve as base solutions for global (or
computing edit distance of two strings) and semi-global alignments (or mapping
when a set of sequences is being aligned). Given that there is both biological
and technical variation in the data, a biologically plausible alignment is one
that minimizes the corresponding differences (e.g. insertions, deletions and
substitutions), so metrics based on edit distance are usually used. Backurs and
Indyk (2015) showed that even calculating the edit distance between two
sequences (without finding an alignment), is not generally solvable in
strongly-subquadratic time. Moreover, even for related sequences of lengths n
and m and edit distance s, the fastest optimal global (Marco-Sola et al., 2021;
Šošic and Šikic, 2017)) and semi-global aligners (Rautiainen et al., 2017) scale
quadratically when the edit distance increases with the length, which is the
case for sequencing errors and biological variation: O(s*min(n,m))=O(enm) and
O(nm), respectively, where e is the error rate (Navarro, 2001). In the age of
big data and long reads (e.g. PacBio, ONP), this quadratic scaling with length
is prohibitive, so the algorithms with practical usage (e.g. minimap2, bwt,
kallisto) do not guarantee optimality but run in subquadratic time (Kucherov,
2019). The gap between fast and optimal global alignment has been recognized but
no optimal algorithms are known that run subquadratically for related sequences
(Medvedev, 2022a). The interest towards genome graphs keeps increasing with the
first International Genome Graph Symposium being held this year in Ascona,
Switzerland (2022). The benefits of using graph references representing
biological variation has been demonstrated to increase the alignment quality
(Garrison et al., 2018). The transition towards graph references only aggravates
the computational issues owing to the potentially complex graph topology (Equi
et al., 2019). The optimal algorithms used in computational biology explore the
search space of possible alignments in an uninformed fashion: by aligning a
prefix of one sequence to a prefix of the other. This contrasts with the
informed search algorithms such as the algorithm by Hunt and Szymanski (1977)
solving the longest common subsequence (LCS) problem (a special case of the edit
distance alignment). Sequence alignment can naturally be formulated as a
shortest path problem solvable by Dijkstra’s algorithm (Ukkonen, 1985). \A is an
informed generalization of Dijkstra’s algorithm (Hart, 1968) but it has not been
successfully applied to sequence alignment. \A may be the missing piece in the
“a major open problem to implement an algorithm with linear-like empirical
scaling on inputs where the edit distance is linear in n” (Medvedev, 2022a).

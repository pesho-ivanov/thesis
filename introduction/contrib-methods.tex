\subsection*{Methods}

All our methods we combine and build upon textbook informatics algorithms and
data structures which have been known for many decades. These include the
formulation of sequence alignment as a shortest path problem, the \A algorithm,
the trie data structure, as well as trivial optimizations, such as hashing
substring, greedily matching equal letters, topological sorting.

\paragraph{\A for sequence alignment}
We consider the sequence alignment problem in its principled and powerful graph
formulation: an alignment minimizing edit distance is equivalent to a shortest
path in an \emph{alignment graph}. It allows to choose any shortest path
algorithm, and, assuming non-negative edit costs, we select the \A algorithm for
its ability to use any available information to quickly direct the search and
yet, to guarantee optimality. In order to efficiently apply \A to semi-global
and global alignment, we complement the reference with a trie index, design a
powerful \emph{seed heuristic}, and implement a number of algorithmic
optimizations. We formally prove the optimality of all algorithms, data
structures, and optimizations.

\paragraph{Implicit constructing of the alignment graph}
The alignment graph is defined as a Cartesian product of the reference and the
query. The structure of the alignment graph is thus regular, and we do not have
to build it explicitly but to only constuct it locally at the node we are
exploring. This optimization is crucial for the overall performance of \A which
spends time only at explored nodes before terminating at the target.

\paragraph{Sequence-to-graph alignment}
Unlike most dynamic programming solutions, we are not bound to acyclic graphs
due to using the general \A shortest path algorithm. Thus, our reference is not
limited to being a linear sequence but can as well be any genome graph. All
semi-global alignment algorithms in this thesis are applicable to general graphs
(possibly including cycles).

\paragraph{Scaling with reference size using a trie index}
In~\cref{ch:trie} we suggest how to exploit that many query sequences are
semi-globally aligned to the same reference. As a preprocessing step, we
complement the reference with a trie index (similar to a suffix tree) so that
any short substring from the reference appears as a path from the trie root to a
trie leaf, and then links to the reference. This way, any substring in the trie
is a spell of a path from the trie root. With an accurate heuristic function
(i.e. estimating the remaining edit distance well), this trie complement allows
to ignore most of the reference, setting the base for sublinear runtime scaling
with the reference size.

\paragraph{Scaling with query length using seed heuristic}
In~\cref{ch:seed} we introduce a powerful \emph{seed heuristic} for \A for
semi-global alignment. It estimates the remaining edit distance based on
information from the whole reference and query. We prove its admissibility, and
present an algorithm for its efficient computation even for the case of a trie
index. We borrow the existing concept of seeds but apply it in a novel way:
instead of searching for good alignments around seed matches, we negate this
logic in order to prove that no alignment is better than the one we find. For
each query, we do the following precomputation: split the query into
non-overlapping seeds of equal length, find all exact matches of each seed in
the reference, and mark all preceding reference nodes with \emph{crumbs} that
signify a following seed match. If \A explores a state, we can then compute the
heuristic as the number of remaining seeds to be aligned that we do not see a
crumb of -- this is, each seed that has to be aligned but cannot be matched,
will require at least 1 edit (assuming Levenshtein distance). Note that the lack
of a match does not require any work but helps us penalize all mismatching seeds
for all paths from all states. We call \emph{potential} of the seed heuristic
the maximal cost it can penalize (e.g. the number of seeds in case of
Levenshtein distance), and we note that the potential grows linearly with the
query length, providing the base for near-linear scaling with length. In order
to precompute the crumbs for the seed heuristic in the case of a trie index over
a potentially-cyclic graph reference, we introduce a linear algorithm based on
topological sort.

\paragraph{Scaling with error rate using inexact matching, chaining and gap costs}
In~\cref{ch:global} on the task of global alignment we demonstrate how to extend
the seed heuristic in order scale to high error rates. Our general
\emph{chaining seed heuristic} includes inexact matching, chaining, and gap
costs. Allowing a single error to match a seed leads to the guarantee of
requiring at least 2 edits to align that seed, which increases the seed
heuristic potential twice. Chaining the seed matches exploits the order of the
seeds in the query, which mitigates the effect of spurious matches. Accounting
for the gap between consecutive matches in a chain allows to penalize more or
longer insertions and deletions. The combination of these extensions enables the
efficient alignment of extremely long sequences with up to 30\% errors. 

\paragraph{Implementations}
All presented algorithms are available as free and open source tools.
\astarix\footnote{\url{https://github.com/eth-sri/astarix}} is our semi-global
sequence-to-graph aligner,
\astarpa\footnote{\url{https://github.com/RagnarGrootKoerkamp/astar-pairwise-aligner}}
is our global sequence-to-sequence aligner,
MinSH\footnote{\url{https://github.com/pesho-ivanov/minSH}} is a minimalistic
implementation of the seed heuristic for global alignment. Here is a working
5-line implementation of the seed heuristic that captures its basic idea:

\begin{samepage}
\begin{minted}[fontsize=\footnotesize, frame=lines]{python}
def build_seedh(A, B, k):
    seeds = [ A[i:i+k] for i in range(0, len(A)-k+1, k) ]
    kmers = { B[j:j+k] for j in range(len(B)-k+1) }
    is_seed_missing = [ s not in kmers for s in seeds ]
    suffix_sum = np.cumsum(is_seed_missing[::-1])[::-1]
    return lambda ij, k=k: suffix_sum[ ceildiv(ij[0], k) ]

h_seed = build_seed_heuristic(A, B, k=log(len(A)))
astar(A, B, h_seed)
\end{minted}
\end{samepage}
\subsection{Practical feasibility of optimal alignment}

There is a fundamental trade-off between perforamance and optimality guarantees:
an algorithm which is allowed to be suboptimal may exploit the lesser
restrictions for greater performance. Especially given the worst-case analysis
requiring near-quadratic runtime even to compute edit distance exactly, it is
understandable why most scholars are skeptical about faster optimal algorithms.
With our \A approach  offers a to exploits another dimensions: average case or expected case analysis.

% asymptotics
%has been solved in linear time in \citeyear{morris1970linear}~\cite{morris1970linear}.

In the direction of global alignment, optimal algorithms are commonly used in
practice, despite of their quadratic scaling. The ongoing competition between
the optimal aligners employs both algorithmic advancements and implementation
optimizations on caching, bit-parallelization, GPU.~(\cref{ch:global}).

For semi-global alignment (read alignment), common believe is that optimal
algorithms are infeasible for read alignment, especially when reads are long. All
production read aligners following the approximate seed-extend
paradigm~\cite{alser2021technology}\footnote{This study examines 107 aligners.}.

Informed search

An algorithm without an objective function may be wrong because they do not
solve the correct problem. or because they 

Algorithms that guarantee correctness can be wrong only by being given a wrong problem.

An approximate algorithm can be wrong either because it did not fullfill its
mathematical goal. reach its solve the problem because of either optimizing the
\emph{wrong} function.

Algorithm correctness is arguably a useful property which is often not simple to
guarantee. It can undoubtedly improve accuracy, especially in the case of
complex data, but still be wrong from biological point of view. This is because
of  This Nevertheless, since biology is a natural science, its  optimality
guarantees must have an additional impact on the development of the field. It
not only but allows to enjoy being wrong rather than vague. Moreove, often
problems in computational biology are ill-stated andalgorithms that approximate
algorithms that. But many algorithms in computational biology do not even have a
formal statement 

\paragraph{Suboptimal alignment}
%
In the last decades, approximate and alignment-free methods satisfied the demand
for faster algorithms which process huge volumes of genetic
data~\citep{kucherov2019evolution}. 
%
\emph{Seed-and-extend} is arguably the most popular paradigm in read
alignment~\citep{altschul_basic_1990,langmead_fast_2012,li_fast_2009}. First,
substrings (called \emph{seeds} or \emph{kmers}) of the read are extracted, then
aligned to the reference, and finally prospective matching locations are
\emph{extended} on both sides to align the full read.

While such a heuristic may produce acceptable alignments in many cases, it
fundamentally does not provide quality guarantees, resulting in suboptimal
alignment accuracy.
%
In contrast, here we demonstrate that seeds can benefit optimal alignment as
well.

\paragraph{Key challenges in optimal alignment}
%
Finding optimal alignments is desirable but expensive in the worst case,
requiring $\Oh(Nm)$ time~\citep{equi2019complexity}, for graph size $N$ and read
length $m$.
%
Unfortunately, most optimal sequence-to-graph aligners rely on dynamic
programming (DP) and always reach this worst-case asymptotic runtime. Such
aligners include \vargas~\citep{darby2020vargas},
\pasgal~\citep{jain_accelerating_2019},
\graphaligner~\citep{rautiainen_bitparallel_2019},
\hga~\citep{feng2021accelerating}, and \vg~\citep{garrison_variation_2018},
which use bit-level optimizations and parallelization to increase their
throughput.

In contrast, we follow the promising direction of using a heuristic to avoid
worst-case runtime on realistic data. To this end, \astarix rephrases the task
of alignment as a shortest-path problem in an \emph{alignment graph} extended by
a \emph{trie index}, and solves it using the \A~algorithm instantiated with a
problem-specific \prefixh. Importantly, its choice of heuristic only affects
performance, not optimality.
%
Unlike DP-based algorithms, this \prefixh allows scaling sublinearly with the
reference size, substantially increasing performance on large genomes. However,
it can only efficiently align reads of limited length.
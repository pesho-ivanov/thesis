\section{Motivation}

Observation 1: A long query has more information that may hint towards the best
alignment place. Unlike the practical approximate algorithms, the existing optimal
algorithm do not exploit this information. As a result, all state-of-the-art
optimal algorithms take longer to map a longer query to a reference. 

Observation 2: The input is linear, the output is linear, the best case is
linear. On the other hand all optimal solutions are near-quadratic and the
theoretical limit is near-quadratic. To spead up beyond this near-quadratic
barrier, current practical algorithms break the optimality guarantee hoping that
the produced alignments are accurate enough. 

%But it is possible to be fast and accurate in the same time?
We take another approach: we preserve the optimality guarantee and use
substantially more information with the hope to be fast enough. As it turns out,
when the error rate is limited, our optimal solutions empirically scale
near-linearly up to very long sequences. This translates to many orders of
magnitude of runtime speedup compared to state-of-the-art optimal aligners.

% Problem, applications
The problem of aligning one biological sequence to another has been formulated
over half a century ago~\citep{needleman1970general} and is known as
\emph{global pairwise alignment}~\citep{navarro2001guided}. Pairwise alignment
has numerous applications in computational biology, such as genome assembly,
read alignment, variant detection, multiple sequence alignment, and differential
expression~\citep{prjibelski2018sequence}. Despite the centrality and age of
pairwise alignment, ``a major open problem is to implement an algorithm with linear-like
empirical scaling on inputs where the edit distance is linear
in~$n$''~\citep{medvedev2022theoretical}.

% Near-quadratic worst case
Alignment accuracy affects the subsequent analyses, so a common goal
is to find a shortest sequence of edit operations (insertions, deletions, and
substitutions of single letters) that transforms one sequence into the other.
Finding such a sequence of operations is at least as hard as computing the \emph{edit
distance}, which has recently been proven to not be computable in strongly
subquadratic time, unless SETH is false~\citep{backurs2015edit}. Given that
the number of sequencing errors is proportional to the length, existing exact aligners are
limited by quadratic scaling not only in the worst case but also in practice.
This is a computational bottleneck given the growing amounts of biological data
and the increasing sequence lengths~\citep{kucherov2019evolution}.

Sequence alignment is a class of combinatorial problems that is of primary
importance for analysis of genetic data. Algorithms and tools for alignment have
been thoroughly developed and routinely used for genome assembly, RNA
quantification, detecting splicing, oncology, multiple sequence alignment (MSA),
and evolutionary biology. Types of sequence alignment include global alignment,
semi-global alignment, alignment, local alignment, and others. For each type, a
common tradeoff that had to be done is between the alignment accuracy and the
performance to find it.

% graph reference
An additional difficulty is the fact that in the
upcoming pangenomic era, these algorithms must be also applicable to complex
graph structures. For more than 60 years, a linear sequence has been extremely
useful as a representation of a single genome. The affordability of sequencing
technologies enables not only to sequence genomes deeper but also to sequence
many genomes (e.g. of organisms or single cells), building a pangenome (an
abstracted genome that represents the genetic variation of a whole clade). The
shift towards population studies in the last decade motivates the adoption of
graph data structures which serve as compressed representations of collections
of related genomes (genomes are paths in the graph).

% \A context 
An optimal alignment can naturally be represented as a shortest path in an
alignment graph (equivalent to the DP table). In order to find such a shortest
path with minimal exploration, we instantiate the \A algorithm with a novel
heuristic function based on the unaligned parts of the sequences. This
additional information is a problem-specific heuristic function and it heavily
determines the efficiency of the search. For any explored state by \A, this
heuristic function should compute a lower bound on the remaining path length, or
more specifically, the minimal cost of edit operations needed to align the
remaining sequences.

%only optimal alignment in this thesis
%asymptotics analysis
%????? linear I/O but quadratic optimal
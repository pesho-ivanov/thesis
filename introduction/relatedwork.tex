\section*{Related work}
\addcontentsline{toc}{section}{\protect\numberline{}{Related work}}

Over the last half-century, many algorithms have been developed for sequence
alignment~\citep{navarro2001guided}. With the increasing quantity and length of
genomic sequences, the need for fast and accurate algorithms is steadily
increasing~\cite{alser2021technology}. There are multiple directions in which
the sequence alignment algorithms are being developed: scalable algorithms,
generalizing the optimization metric, employing heuristics for approximate
solutions, adapting existing algorithms to novel types of data, exploiting
parallel computing, implementing fast and memory-efficient software. This thesis
focuses on algorithmic approaches to sequence alignment so here we outline the
leading existing optimal algorithms, the suboptimal seeding heuristics that we
will will use in an optimal way, and the \A shortest path algorithm which will
be the base for our provably optimal and heuristically fast approach.

\paragraph{Dynamic programming for global alignment}
The first efficient solution for global alignment, known as the Needleman-Wunsch
algorithm, got developed around~\citeyear{vintsyuk1968speech}~
\cite{vintsyuk1968speech,needleman1970general}. This algorithm is based on the
dynamic programming~(DP) technique~\cite{bellman1954theory} of splitting a task
into overlapping subtasks~(or \emph{states}) each of which can be solved once
and then being reused. Each node of the alignment graph (aligning a prefix of
the first sequence to a prefix of the second sequence) is a state in the DP.
Each subtask can be solved by reducing it to already solved alignments of
shorter prefixes, and extending them by a last operation of matching or
mismatching the last letters from the prefixes, inserting one, or inserting the
other (equivalent to deleting from the first sequence). Each subtask is solved
for $\Oh(1)$ so the quadratic number of prefix pairs is solved for quadratic
overall time, even though the number of possible alignments is exponential.
Independently, it was applied to compute edit distance for biological
sequences~\citep{needleman1970general,sankoff1972matching,sellers1974theory,wagner1974string}.
This well-known algorithm is implemented in modern aligners  like
\seqan~\citep{reinert2017seqan} and \parasail~\citep{daily2016parasail}.
Improving the performance of these quadratic algorithms has been a central goal
in later works.

\paragraph{Dynamic programming for semi-global alignment}
Semi-global alignment is a related problem to global alignment but in addition
to the need to align letter-to-letter, it also requires determining a starting
position in the reference. This task has also been efficiently solved using a
similar DP approach~\cite{sellers1980theory,smith1981identification} taking
$\Oh(nm)$ to align a single query of length $n$ to a reference of length $m$. It
is highly redundant to explore the whole reference for each alignment, so
various data structures were suggested that can be precomputed to index the
reference, and then reuse this index to quickly align multiple query sequences
~\citeyear{pearson1988improved}~\cite{pearson1988improved} and has become
central to read alignment of high-throughput sequencing.

% Optimal DP-based approaches
\paragraph{Optimal semi-global alignment}
Current optimal alignment algorithms reach the impractical $\Oh(nm)$ runtime
that has been shown to be a lower bound for the worst-case edit distance
computation~\cite{backurs2015edit}. In this light, approaches for improving the
efficiency of optimal alignment have taken advantage of specialized features of
modern CPUs to improve the practical runtime of the Smith-Waterman dynamic
programming (DP) algorithm~\cite{smith_comparison_1981} considering all possible
starting nodes. These use modern SIMD instructions (\AG
\vg~\cite{garrison_variation_2018} and \pasgal~\cite{jain_accelerating_2019}) or
reformulations of edit distance computation to allow for bit-parallel
computations in \graphaligner \footnote{We refer as \bitparallel to to the
bit-parallel DP algorithm implemented in \graphaligner tool
\cite{rautiainen_bitparallel_2019}.}~\cite{rautiainen_bitparallel_2019}. Many of
these, however, are designed only for specific types of genome graphs, such as
{\itshape de Bruijn}
graphs~\cite{liu_debga_2016,limasset2019toward} and
variation graphs~\cite{garrison_variation_2018}. A compromise often made when
aligning sequences to cyclic graphs using algorithms reliant on directed acyclic
graphs involves the computationally expensive ``DAG-ification'' of graph
regions~\cite{kavya_sequence_2019,garrison_variation_2018}.

\paragraph{Suboptimal heuristic algorithms}
All optimal solutions are near-quadratic and this is justified by the
theoretical lower bound. To spead up beyond this near-quadratic barrier, current
practical algorithms break the optimality guarantee with the hope that the
produced alignments are accurate enough. 

% Heuristics for alignment
Heuristics are employed for both sequence-to-sequence alignment and
sequence-to-graph alignment, to keep the computation
tractable~\cite{altschul_basic_1990,langmead_fast_2012,garrison_variation_2018},
especially for large populations of human-sized genomes. In the last decades,
approximate and alignment-free methods satisfied the demand for faster
algorithms which process huge volumes of genomic
data~\citep{kucherov2019evolution}. 

While heuristics may produce acceptable alignments in many cases, it
fundamentally does not provide quality guarantees, resulting in suboptimal
alignment accuracy.

\paragraph{Seed-and-extend}
Since optimal alignment is often intractable, many aligners use heuristics, most
commonly the \emph{seed-and-extend}
paradigm~\cite{altschul_basic_1990,langmead_fast_2012,li_fast_2009}. In this
approach, alignment initiation sites (\emph{seeds}) are determined, which are
then \emph{extended} to form the \emph{alignments} of the query sequence. The
fundamental issue with this approach, however, is that the seeding and extension
phases are mostly decoupled during alignment. Thus, an algorithm with a provably
optimal extension phase may not result in optimal alignments due to the
selection of a suboptimal seed in the first phase. In cases of high sequence
variability, the seeding phase may even fail to find an appropriate seed from
which to extend.

\paragraph{\A for sequence alignment}
\A is a an \emph{informed} shortest path algorithm that generalizes \dijkstra by
directing the search towards the target~\citep{hart1968formal}. To do this, it
relies on a huristic function that estimates the remaining distance from each
explored vertex to the target. If this heuristic never overestimates the actual
distance to the target, \A is guaranteed to find an optimal path. If the
heuristic is accurate -- similar to the actual remaining distance, then \A finds
the path with minimal exploration. If the heuristic can be computed efficiently,
the overall runtime stays low. Designing a heuristic function with these
properties has been challenging. There has been one attempt to apply \A for
optimal pairwise alignment~\cite{dox2018efficient} which used a highly in
heuristic function that accounts only for the length of the remaining query
sequence to be aligned. \A has also been considered for solving multiple
sequence alignmnet but has never been of practical interest due to its
exponential scaling with the number of
sequences~\citep{lermen2000practical,zhou2002multiple,mcnaughton2002memory}.

\paragraph{Graph references}
First attempts to include variation into the reference data structure were made
by augmenting the local alignment method to consider alternative walks during
the extend step~\cite{schneeberger_simultaneous_2009,palmapper}. This approach
has since been extended from the linear reference case to graph references. To
represent non-reference variation of multiple references during the seeding
stage, HISAT2 uses generalized compressed suffix
arrays~\cite{siren_indexing_2014} to index walks in an augmented reference
sequence, forming a local genome graph~\cite{kim_graphbased_2019}.
VG~\cite{garrison_variation_2018} uses a similar
technique~\cite{siren_indexing_2017} to index variation graphs representing a
population of references.

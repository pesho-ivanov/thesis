\section{Contributions}

\subsection{Knowledge gap}
\dictum{%
   Provably optimal and heuristically fast.}
\vskip 1em

\paragraph{Main contributions}
We have effectively applied the \A algorithm to optimal sequence alignment. We
demontrated that the additional information from the whole sequence can improve
the scaling with query length, reference size and error rate, substantially
decrease the necessary computations, and result in algorithms that are orders of
magnitude faster than existing optimal algorithms. We apply the \A approach to
two types of alignment: semi-global (alignment) and global.

\paragraph{Tools}

\subsection{Informed search}
Two-stage algorithm, similar to Aho-Corasick, increasingly more information (length, prefix, seeds, chaining seeds, chaining seeds + gaps)
 
% Accuracy and Metrics
The number of possible alignments grow exponentially with length. The usual
underlying question to finding ``correct'' alignments. Regarding the precision
of alignment, one is usually interested in base-to-base (aka letter-to-letter)
correspondence between the sequences, even though for some applications a less
detailed solution is sufficient: only the similarity between sequences or the
location where a read maps to a reference. Exact alignment is only useful for
very short sequences (often kmers), and for all other cases the optimized metric
may be hamming distance, edit distance (unit costs), Levenshtein distance,
affine costs, convex and concave costs, general costs and others. 

% Problem statement
Depending on the the number of aligned sequences, there is pairwise alignment
and multiple sequence alignment (MSA). Depending on the parts of the sequences
that are aligned to each other, we differentiate global, local and various
semi-global alignemnts. There are generalizations to sequence-to-sequence
alignment, including aligning to nonlinear structures, such as directed acyclic
graphs, DAGs, general graphs and others. These structures are nowadays becoming
more common as a compressed form of representing a set of references to which a
sequence can be aligned. Often, one best alignment is suefficient but finding
several best (top-K) alignments. In the context of read alignment, a set of reads
is aligned to the same reference sequence so an indexing procedure is often
useful for the performance.

We specifically consider the alignment of a set of reads to a general graph, and
the global pairwise alignment.

Existing optimal algorithms are based on dynamic programming (DP) and
run in quadratic time (assuming that the number of errors is proportional to the
length)

we employ the \A algorithm which is an \emph{informed search} algorithm.
TODO: a case for the informed algorithms

\subsection{Scalable optimal alignments}

% Heuristics for alignment
Both for sequence-to-sequence alignment and sequence-to-graph alignment,
heuristics are employed to keep alignment
tractable~\cite{altschul_basic_1990,langmead_fast_2012,garrison_variation_2018},
especially for large populations of human-sized genomes.
%
% Importance of optimal alignment
While such heuristics find the correct alignment for simple references, they
often perform poorly in regions of very high complexity, such as in the human
major histocompatibility complex (MHC)~\cite{dilthey_improved_2015}, in complex
but rare genotypes arising from somatic-subclones in tumor sequencing
data~\cite{harismendy_detection_2011}, or in the presence of frequent sequencing
errors~\cite{salmela_lordec_2014}.
%
Importantly, these cases can be of specific clinical or biological interest, and
incorrect alignment can cause severe biases for downstream analyses. For
instance, the combination of high variability of MHC sequences in humans and
small differences between alleles~\cite{buhler_hla_2011} leads to a risk of
misclassifications due to suboptimal alignment. Guaranteeing optimal alignment
against all variations represented in a graph is a major step towards
alleviating those biases.

% Optimal DP-based approaches
\paragraph{Optimal Alignment}
Current optimal alignment algorithms reach the impractical $\Oh(nm)$ runtime
that has been shown to be a lower bound for the worst-case edit distance
computation~\cite{backurs2015edit}. In this light, approaches for improving the
efficiency of optimal alignment have taken advantage of specialized features of
modern CPUs to improve the practical runtime of the Smith-Waterman dynamic
programming (DP) algorithm~\cite{smith_comparison_1981} considering all possible
starting nodes. These use modern SIMD instructions (\eg
\vg~\cite{garrison_variation_2018} and \pasgal~\cite{jain_accelerating_2019}) or
reformulations of edit distance computation to allow for bit-parallel
computations in \graphaligner \footnote{We refer as \bitparallel to to the
bit-parallel DP algorithm implemented in \graphaligner tool
\cite{rautiainen_bitparallel_2019}.}~\cite{rautiainen_bitparallel_2019}. Many of
these, however, are designed only for specific types of genome graphs, such as
{\itshape de Bruijn}
graphs~\cite{liu_debga_2016,limasset2019toward} and
variation graphs~\cite{garrison_variation_2018}. A compromise often made when
aligning sequences to cyclic graphs using algorithms reliant on directed acyclic
graphs involves the computationally expensive ``DAG-ification'' of graph
regions~\cite{kavya_sequence_2019,garrison_variation_2018}.

%\section{Optimal alignment}
Finding an optimal alignment requires a conceptually different approach than
finding an approximate alignment. Instead of finding \emph{one} good alignment,
finding an optimal alignment requires proving that \emph{all} other
exponentially-many alignments are not better.

Comparing one sequence to another is a basic combinatorial problem that has
several variations (shown on the right), each applicable in computational
biology. Needleman-Wunsch (1970)  and Smith-Waterman (1981) are dynamic
programming (DP) algorithms that serve as base solutions for global (or
computing edit distance of two strings) and semi-global alignments (or alignment
when a set of sequences is being aligned). Given that there is both biological
and technical variation in the data, a biologically plausible alignment is one
that minimizes the corresponding differences (e.g. insertions, deletions and
substitutions), so metrics based on edit distance are usually used. Backurs and
Indyk (2015) showed that even calculating the edit distance between two
sequences (without finding an alignment), is not generally solvable in
strongly-subquadratic time. Moreover, even for related sequences of lengths n
and m and edit distance s, the fastest optimal global (Marco-Sola et al., 2021;
Šošic and Šikic, 2017)) and semi-global aligners (Rautiainen et al., 2017) scale
quadratically when the edit distance increases with the length, which is the
case for sequencing errors and biological variation: O(s*min(n,m))=O(enm) and
O(nm), respectively, where e is the error rate (Navarro, 2001). In the age of
big data and long reads (e.g. PacBio, ONP), this quadratic scaling with length
is prohibitive, so the algorithms with practical usage (e.g. minimap2, bwt,
kallisto) do not guarantee optimality but run in subquadratic time (Kucherov,
2019). The gap between fast and optimal global alignment has been recognized but
no optimal algorithms are known that run subquadratically for related sequences
(Medvedev, 2022a). The interest towards genome graphs keeps increasing with the
first International Genome Graph Symposium being held this year in Ascona,
Switzerland (2022). The benefits of using graph references representing
biological variation has been demonstrated to increase the alignment quality
(Garrison et al., 2018). The transition towards graph references only aggravates
the computational issues owing to the potentially complex graph topology (Equi
et al., 2019). The optimal algorithms used in computational biology explore the
search space of possible alignments in an uninformed fashion: by aligning a
prefix of one sequence to a prefix of the other. This contrasts with the
informed search algorithms such as the algorithm by Hunt and Szymanski (1977)
solving the longest common subsequence (LCS) problem (a special case of the edit
distance alignment). Sequence alignment can naturally be formulated as a
shortest path problem solvable by Dijkstra’s algorithm (Ukkonen, 1985). \A is an
informed generalization of Dijkstra’s algorithm (Hart, 1968) but it has not been
successfully applied to sequence alignment. \A may be the missing piece in the
“a major open problem to implement an algorithm with linear-like empirical
scaling on inputs where the edit distance is linear in n” (Medvedev, 2022a).
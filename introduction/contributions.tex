\section*{Contributions}
\addcontentsline{toc}{section}{\protect\numberline{}Main results}

%combining several existing ideas new (\A, trie, seeds)

\begin{figure}[h]
  \includegraphics[width=1.0\linewidth]{media/ownpubs-table.png}
  \caption{Overview of the contributions by publication. \cref{ch:trie}
  corresponds to \citep{ivanov2020astarix}, \cref{ch:seed} to
  \citep{ivanov2022fast}, \cref{ch:global} to \citep{koerkamp2022exact}}
  \label{tab:ownpubs}
\end{figure}

% Everything in one paragraph
In this thesis we introduce a novel principled approach for provably optimally
and heuristically fast sequence alignment. Our basic idea is to direct the
search for a best alignment based not only on information from the aligned parts
of the sequences (as existing optimal algorithms do), but also from the
unaligned parts. We find the shortest path formulation of sequence alignment to
be more natural and powerful than the classic dynamic programming formulation
which already implies a specific solution. We choose the \A shortest path
algorithm since it can direct the search using a custom heuristic function that
estimates the remaining distance from any node to using the additional information from the whole sequences.
The \A framework allows for constructions We take an approach of preserving the
optimality guarantee while exploiting substantially more information for the
sake of performance.

We borrow the existing concept of seeds but apply it in a novel way: instead of
finding alignments around seed matches, we use the lack of matches to penalize
alignments by desigining the \emph{seed heuristic} that drive the \A search.
Seeds are cousins of kmers which are popular in sequence alignment since de
Bruijn Graphs were applied for genome assembly. Further kmers are cousins of
ngrams which are popular in computational linguistics since .

We have effectively applied the \A algorithm to optimal sequence alignment. We
demontrate that the additional information from the whole sequence can improve
the scaling with query length, reference size and error rate, substantially
decrease the necessary computations, and result in algorithms that are orders of
magnitude faster than existing optimal algorithms. We apply the \A approach to
two types of alignment: semi-global (alignment) and global.

\subsection*{Principled approach by shortest paths}
shortest paths,
A*, admissibility
Informed search: Two-stage algorithm, similar to Aho-Corasick, increasingly more information (length, prefix, seeds, chaining seeds, chaining seeds + gaps)

%\section{Reconceptualizing seeds for optimal alignment}
\section{Beyond \emph{seed-(chain)-extend} paradigm}

As we saw in \cref{ch:trie,ch:seed}, all optimal read aligners compute the whole
dynamic programming table, thus reaching the prohibitively slow quadratic
runtime. On the other side, all current production aligners rely on the
\emph{seed-extend} paradigm (and its \emph{seed-chain-extend} variants for long
reads).

This paradigm requires similar short \emph{seed} patches to be found
between the sequences (\eg by hashed kmers, minimiziers, maximum exact matching,
etc.), and then to \emph{extend} the alignment of the whole query around these
\emph{seeded} similar patches. This is a very intuitiv approach if the goal is
to find a \emph{good alignment}.

If we instead seek not good but provably \emph{best} alignments, we are required
to at least implicitly refute all the exponentially-many competing alignments.

Instead, to find optimal alignments, we do not need to choose the seeds to be
long and similar with the reference buare not required to be similar.

\begin{observation}[Seeds without matches]
    To efficiently find an optimal alignment using \A with the seed heuristic,
    seeds are not required to match (even on the resulting alignment).
\end{observation}

Nevertheless, each seed can penalize potential alignment by not more than its
\emph{potential} (\ie the number of plus $1$, for the case of exact matching
with unit costs). Any additional errors will require more states to be expanded.

This is an interesting observation was made by Ragnar while playing with the
seed heursitic. It looks Indeed, of finding a good alignment but to prove that all
alternative alignments are no better, the seed heuristic for \A search does not
really need matches to be efficient.

This novel usage of seeds carrie different problems and different possibilities.

\subsection*{Optimality guarantees}
To ensure that our algorithms are practical, we introduce a number of
algorithmic optimizations which increase performance and decrease memory
footprint.

We also prove that all optimizations (greedy matching, ) are preserving the optimality.

\section{Scaling analysis}

Classically, the runtime and memory usage of algorithms has been analysed for a
worst case asymptotic behavior. Since the near quadratic worst case asymptotics
is likely to be tight, we target related sequences (with limited error rates),
and empyrically the runtime and memory scaling.

The seed heuristic, which is in the core of our \A algorithm, is admissible
(optimistic) but not consistent (monotone). As a consequence, it guarantees
finding a best alignment but the same state could be expanded more than once
(and this indeed happens). Theoretically, this also implies that the worst
case asymptotics in not anymore bounded by the quadratic number of states.

Nevertheless, in practice

we employ a best-fit estimation of the scaling which
is not asymptotic. Tools

Theory

exponential scaling

\subsection*{Aligning to general graph references}

\subsection*{Implementations}

We empirically demonstrate the superior scaling and performance to other optimal
alignment algorithms.

\paragraph{Scaling with reference size}
In~\cref{ch:trie} we present the tool \astarix which applies the \A algorithm to
find optimal alignments, based on a domain-specific heuristic and enhanced by
multiple algorithmic optimizations. Importantly, our approach allows for both
cyclic and acyclic graphs including variation and de Bruijn graphs. We
demonstrate that using a trie index we can achieve sublinear scaling of aligning
runtime with reference size, and that \A can scale exponentially better than
\dijkstra with increasing (but small) number of errors in the reads. Moreover,
for short reads, both \astarix and \dijkstra scale better and outperform current
state-of-the-art optimal aligners with increasing genome graph size.
Nevertheless, scaling optimal alignment of long reads on big graphs remained an
open problem.

\paragraph{Scaling with query length}
In~\cref{ch:seed} we upgrade \astarix with a novel \sh which guides the \A
search by preferring crumbs on nodes that lead towards optimal alignments even
for long reads. This approach enables the scaling of semi-global alignment with
read length.

\paragraph{Scaling with error rate}
In~\cref{ch:global} we resolve the third major bottleneck -- handling high error
rates. We presented an algorithm with an implementation in \astarpa solving
pairwise alignment between two sequences. The algorithm is based on \A with a
\sh, inexact matching, match chaining, and match pruning, which we proved to
find an exact solution according to edit distance. For random sequences with up
to $15\%$ uniform errors, the runtime of \astarpa scales near-linearly to very
long sequences ($10^7\bp$) and outperforms other exact aligners. We demonstrate
that on real ONT reads from a human genome, \astarpa is faster than other
aligners on only a limited portion of the reads.

%%%%%%
As it turns out, when the error rate is limited, our optimal solutions
empirically scale near-linearly up to very long sequences. This translates to
many orders of magnitude of runtime speedup compared to state-of-the-art optimal
aligners.

% \A context 
An optimal alignment can naturally be represented as a shortest path in an
alignment graph (equivalent to the DP table). In order to find such a shortest
path with minimal exploration, we instantiate the \A algorithm with a novel
heuristic function based on the unaligned parts of the sequences. This
additional information is a problem-specific heuristic function and it heavily
determines the efficiency of the search. For any explored state by \A, this
heuristic function should compute a lower bound on the remaining path length, or
more specifically, the minimal cost of edit operations needed to align the
remaining sequences.

In contrast, here we demonstrate that seeds can benefit optimal alignment as
well.

In contrast, we follow the promising direction of using a heuristic to avoid
worst-case runtime on realistic data. To this end, \astarix rephrases the task
of alignment as a shortest-path problem in an \emph{alignment graph} extended by
a \emph{trie index}, and solves it using the \A~algorithm instantiated with a
problem-specific \prefixh. Importantly, its choice of heuristic only affects
performance, not optimality.
%
Unlike DP-based algorithms, this \prefixh allows scaling sublinearly with the
reference size, substantially increasing performance on large genomes. However,
it can only efficiently align reads of limited length.

Most of the techniques this thesis builds upon have been known for many decades
and have also been heavily motivated by applications in molecular biology. 

The specifics with semi-global alignment requires a trie-like index which is
well-used in the fields, and known since
\citeyear{thue1912gegenseitige}~\cite{thue1912gegenseitige} and used in
informatics since~\citeyear{de1959file}~\cite{de1959file}.
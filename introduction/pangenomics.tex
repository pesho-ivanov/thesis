\section{Pangenomes and reference graphs}

The shortest path approach naturally fits more complex references than linear.
In fact, any graph reference (incl. cycles) is fine.

% paper-trie; Accounting for variation
\para{Accounting for Variation}
First attempts to include variation into the reference data structure were made
by augmenting the local alignment method to consider alternative walks during the
extend step~\cite{schneeberger_simultaneous_2009,palmapper}. This approach has
since been extended from the linear reference case to graph references. To
represent non-reference variation of multiple references during the seeding
stage, HISAT2 uses generalized compressed suffix
arrays~\cite{siren_indexing_2014} to index walks in an augmented reference
sequence, forming a local genome graph~\cite{kim_graphbased_2019}.
VG~\cite{garrison_variation_2018} uses a similar
technique~\cite{siren_indexing_2017} to index variation graphs representing a
population of references.

% paper-trie; The benefit from genome graphs
Historically, a single linear reference sequence has been used to represent the
most common variants in a population. While providing a working abstraction for
most cases, rare or sub-population specific variation is especially hard to
model in this setting, creating a reference allele
bias~\cite{stevenson_sources_2013,brandt_mapping_2015}. Consequently, in the
last few years, the field has shifted first towards using sets of reference
sequences, and more recently to graph data structures (so-called {\em genome
graphs}), to represent many genomes or haplotypes
simultaneously~\cite{dilthey_improved_2015,paten_genome_2017,garrison_variation_2018}.
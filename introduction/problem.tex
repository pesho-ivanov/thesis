\section{Alignment problem}

% Sequencing and variant calling
The analysis and understanding of genetic variation encoded in the genome of an
organism lies at the center of computational biology and medicine. Variation is
usually identified through matching sequences obtained from DNA/RNA-sequencing
back to a reference (genome) sequence in the process of \emph{variant calling},
making the alignment task a core problem in sequence bioinformatics.

% paper:seeds
\subsection{Problem statement: Alignment as shortest path} \label{SEEDsec:task}
%
In the following, we formalize the task of optimally aligning a read to a
reference graph in terms of finding a shortest path in an \emph{alignment
graph}. Our discussion closely follows~\citep{ivanov2020astarix} and is in line
with~\citep{rautiainen_aligning_2017}.

\para{Reference graph}
%
A reference graph $\RG=(\RGV,\RGE)$ encodes a collection of references to be
considered when aligning a read. Its directed edges $\RGE \subseteq \RGV \times
\RGV \times \Sigma$ are labeled by nucleotide letters from $\Sigma =
\{\texttt{A},\texttt{C},\texttt{G},\texttt{T}\}$, hence any walk
$\reference{\pi}$ in $\RG$ spells a string $\sigma(\reference{\pi}) \in
\Sigma^*$.

An alignment of a read $q \in \Sigma^*$ to a reference graph $\RG$ consists of
(i)~a walk $\reference{\pi}$ in $\RG$ and (ii)~a sequence of edits (matches,
substitutions, deletions, and insertions) that transform
$\sigma(\reference{\pi})$ to $q$. An alignment is \emph{optimal} if it minimizes
the sum of edit costs for a given real-valued cost model $\cedits = (\cmatch,
\csubst,\cdel, \cins)$.
%
Throughout this work, we assume that edit costs are non-negative---a
pre-requisite for the correctness of \A. Further, we assume that $\cmatch \leq
\csubst, \cins, \cdel$---a prerequisite for the correctness of our heuristic.

We note that our approach naturally works for cyclic reference graphs.

\begin{figure}[t]
	\begin{alignat*}{20}
		(
			&\langle&& u &,& i   &\rangle&,
			&\langle&  v &,& i+1 &\rangle&,
			&&q[i],
			&&\cmatch
		&)&\in \AGE
		&& \quad \text{ if } (u,v,\ell) \in \RGE, \ell = q[i] & \qquad \text{(match)}\\
		%
		(
			&\langle&& u &,& i   &\rangle&,
			&\langle&  v &,& i+1 &\rangle&,
			&&q[i],
			&&\csubst
		&) &\in \AGE
		&& \quad \text{ if } (u,v,\ell) \in \RGE, \ell \neq q[i] & \qquad \text{(substitution)}\\
		%
		(
			&\langle&& u &,& i &\rangle&,
			&\langle&  v &,& i &\rangle&,
			&&\varepsilon,
			&&\cdel
		&) &\in \AGE
		&& \quad \text{ if } (u,v,\ell) \in \RGE & \qquad \text{(deletion)}\\
		%
		(
			&\langle&& u &,& i   &\rangle&,
			&\langle&  u &,& i+1 &\rangle&,
			&&q[i],
			&&\cins
		&) &\in \AGE
		&& \quad & \qquad \text{(insertion)},
	\end{alignat*}
	\caption[Formal definition of alignment graph]{Formal definition of
	alignment graph edges $\AGE \subseteq \AGV[q] \times \AGV[q] \times
	\Sigma_\varepsilon \times \mathbb{R}_{\geq 0}$. Here, $u,v \in \RGV$, $0
	\leq i < |q|$, $\ell \in \Sigma$, and $\varepsilon$ represents the empty
	string, indicating that letter $\ell$ was deleted.}
	\label{SEEDfig:graph-edges}
\end{figure}

\para{Alignment graph}
%
In order to formalize optimal alignment as a shortest path finding problem, we
rely on an \emph{alignment graph} $\AG[q]=(\AGV[q],\AGE[q])$.
%
Its nodes $\AGV[q]$ are \emph{states} of the form $\langle v, i \rangle$, where
$v \in \RGV$ is a node in the reference graph and $i \in \{0, \dots, |q|\}$
corresponds to a position in the read $q$.
%
Its edges $\AGE[q]$ are selected such that any path $\alignment{}{\pi}$ in
$\AG[q]$ from $\langle u, 0 \rangle$ to $\langle v, i \rangle$ corresponds to an
alignment of the first $i$ letters of $q$ to $\RG$.
%
Further, the edges are weighted, which allows us to define an \emph{optimal
alignment} of a read $q \in \Sigma^*$ as a shortest path $\alignment{}{\pi}$ in
$\AG[q]$ from $\langle u, 0 \rangle$ to $\langle v, |q| \rangle$, for any $u, v
\in \RGV$.
%
\cref{SEEDfig:graph-edges} formally defines the edges $\AGE$.

% paper:global
\paragraph{Sequences}
The input sequences $A = \overline{a_0a_1\dots a_i \dots a_{n-1}}$ and $B =
\overline{b_0b_1 \dots b_j \dots b_{m-1}}$ are over an alphabet $\Sigma$ with
$4$ letters. We refer to substrings $\overline{a_i \dots a_{i'-1}}$ as
$\substr Ai{i'}$, to prefixes $\overline{a_0 \dots a_{i-1}}$ as $A_{<i}$, and to
suffixes $\overline{a_i \dots a_{n-1}}$ as $A_{\geq i}$. The \emph{edit
distance} $\ed(A,B)$ is the minimum number of insertions, deletions, and
substitutions of single letters needed to convert $A$ into $B$.

\paragraph{Edit graph}
Let \emph{state} $\st{i}{j}$ denote the subtask of aligning the prefix $A_{<i}$
to the prefix $B_{<j}$. The \emph{edit graph} (also called \emph{alignment
graph}) $G(V,E)$ is a weighted directed graph with vertices $V = \{\st ij \vert
{0\leq i \leq n}, {0\leq j\leq m}\}$ corresponding to all states, and edges
connecting tasks to subtasks: edge ${\st ij \to \st{i{+}1}{j{+}1}}$ has cost
$0$ if ${a_i = b_j}$ (match) and $1$ otherwise (substitution), and edges ${\st ij
\to \st{i{+}1}{j}}$ (deletion) and ${\st ij \to\st{i}{j{+}1}}$ (insertion) have cost
$1$. We denote the root state $\st 00$ by $v_s$ and the target state $\st nm$ by
$v_t$.
For brevity we write $f(\st ij)$ as $f \st ij$.
The edit graph is a natural representation of the alignment problem that
provides a base for all alignment algorithms.
\section*{Genomic sequences}
\addcontentsline{toc}{section}{\protect\numberline{}{Relation to genomics}}

\paragraph{DNA structure}
The discovery of the structure and function of DNA\citep{watson1953structure}
seventy years ago started a revolution in our understanding of the living world.
The non-branching (linear or circular) DNA structure resembles similar the
linear structure of human text and speech. This property is massively exploited
by both \emph{sequencing} technologies used to digitalize genomic information,
and by algorithms used to analyze the data.

\paragraph{Sequence alignment}
Pairwise sequence alignment has been a core problem in computational biology for
more than half a century, exhibiting a multitude of applications, including
variation analysis, de novo assembly, read alignment, variant detection,
philogenetics, RNA quantification, multiple sequence alignment, differential
expression~\citep{prjibelski2018sequence}.

\paragraph{Biological and technical variation}
Even though our method should be applicable outside of DNA sequening and even
computational biology, we focus on genomic data for its immense practical
importance. More generally, the differences between the two sequences are
explained by biological variation (resulting from differences in the sequenced
organism) and technical variation (resulting from sequencing errors). Next, we
review the main sequencing technologies up to date and the data profile they
produce.

\paragraph{Read mapping}
Each sequence from a new biological sample can be compared to a reference genome
(if such is available) in order to identify mutations, quantify expression per
gene, and many other types of analysis. To figure out the location in the
reference each sequence most probably originates from, it should first be
semi-globally aligned to the reference. If the sequences of two proteins or
viruses have been reconstructed, aligning them globally reveals the most
probabale mutations between them.

\paragraph{First generation sequencing}
Genomic (DNA, RNA) sequencing is a technology that takes genetic material and
produces a data file with a set of sequences. Since the late 1970s, Sanger
sequencing is the first widely used technology that dominated the field for over
40 years. The produced \emph{reads} (each read is a sequence originating from a
continuous DNA fragment) are of length ${>}1000$ nucleotides and have very high
quality (${<}0.001\%$ error rate)~\citep{shendure2008next}.

\paragraph{Second generation (high-throughput / next-generation / shotgun) sequencing}
Initially, the sequenced data has been scarce and the sequence alignment was
done by hand. The development of high-throughput sequencing in the late 1990s
and 2000s lead to higher amounts of cheaper data due to sequencing millions and
billions of molecules in parallel. The amount and complexity of the available
genomic data bacame not only intractable by hand, but also challenging for
computer analysis.

\paragraph{Third generation (long-read) sequencing}
Since the end of the 2000s, novel technologies started to appear that are
capable of producing long sequences. Long sequences enabled massive applications
but the limiting factor was the high error rate (up to $20\%$) and the higher
price.

\paragraph{Pangenomics and genome graphs}
Traditionally, a single linear reference sequence has been used as a consensus
genome. Nevertheless, linear references are limited in representing biological
variation which the modern sequencing technologies are producing in great
abundance. Thus, in the 2010s, the field started shifting from linear references
towards sets of reference sequences, and their compact \emph{genome graph}
representations. A path in such a graph may represent a specie, an individual, a
cell, a haplotype~\cite{dilthey_improved_2015,paten_genome_2017}. Semi-global
alignment is more accurate on reference graphs than on linear references due to
capturing genomic variation~\citep{garrison_variation_2018}.

\paragraph{Practical alignment algorithms}
Practical alignment algorithms exploit similarities between the sequences to
speed up the computation but give up on optimality guarantees. Thus, a
satisfactory solution that combines optimality guanratees with heuristic
performance is yet to be developed~\citep{medvedev2022theoretical}.

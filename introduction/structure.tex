\section*{Thesis structure}
\addcontentsline{toc}{section}{\protect\numberline{}{Thesis structure}}

In~\cref{ch:trie} we present the tool \astarix which applies the \A algorithm to
find optimal alignments, based on a domain-specific heuristic and enhanced by
multiple algorithmic optimizations. Importantly, our approach allows for both
cyclic and acyclic graphs including variation and de Bruijn graphs. We
demonstrate that using a trie index we can achieve sublinear scaling of aligning
runtime with reference size, and that \A can scale exponentially better than
\dijkstra with increasing (but small) number of errors in the reads. Moreover,
for short reads, both \astarix and \dijkstra scale better and outperform current
state-of-the-art optimal aligners with increasing genome graph size.
Nevertheless, scaling optimal alignment of long reads on big graphs remained an
open problem.

In~\cref{ch:seed} we upgrade \astarix with a novel \seedh which guides the \A
search by preferring crumbs on nodes that lead towards optimal alignments even
for long reads. This approach enables the scaling of semi-global alignment with
read length.

In~\cref{ch:global} we resolve the third major bottleneck -- handling high error
rates. We presented an algorithm with an implementation in \astarpa solving
pairwise alignment between two sequences. The algorithm is based on \A with a
\sh, inexact matching, match chaining, and match pruning, which we proved to
find an exact solution according to edit distance. For random sequences with up
to $15\%$ uniform errors, the runtime of \astarpa scales near-linearly to very
long sequences ($10^7\bp$) and outperforms other exact aligners. We demonstrate
that on real ONT reads from a human genome, \astarpa is faster than other
aligners on only a limited portion of the reads.

\para{Alignment Graph}
To find an optimal alignment of $q$ to the edit graph $\EG$ using shortest path
finding algorithms, we must ensure that only paths spelling $q$ are considered.
To this end, we introduce an alternative but equivalent formulation of
alignments in terms of an \emph{alignment graph} $\AG=(\AGV,\AGE)$.

Here, each \emph{state} $\langle v,i \rangle \in \AGV$ consists of a vertex $v \in
\EGV$ and a query position $i \in \{0,\dots,|q|\}$ (equivalent
to~\cite{rautiainen_aligning_2017}). Traversing a state $\langle v,i \rangle \in
\AGV$ represents the alignment of the first $i$ query characters ending at node $v$.
%
In particular, query position $i=0$ indicates that we have not yet matched any
letters from the query.
%
We note that the alignment graph explicitly depends on the query $q$. In
particular, the example alignment graph $\AG[``\texttt{A}"]$ in
~\cref{TRIEfig:graph-constructions} lacks substitution edges from $\EG$, as their
labels ($\texttt{C}$, $\texttt{G}$, $\texttt{T}$) do not match the query
$q=``A"$.

We construct the alignment graph $\AG$ to guarantee that any walk from a source
$\langle u,0 \rangle$ to a state $\langle v,i \rangle$ corresponds to an
alignment of the first $i$ letters of query $q$ to $\RG$. As a consequence,
there is a one-to-one correspondence between alignments $\edit{\pi}$ of $q$ to
$\EG$ and paths $\alignment{\pi} \in \AG$ from sources $S:=\RGV \times \{0\}$ to
targets $T:=\RGV \times \{|q|\}$, with
$\cost{\reference{\pi}}=\cost{\alignment{\pi}}$. To find the best alignment in
$\EG$, only paths in $\AG$ (walks without repeating nodes) can be considered,
since repeating a node in $\AG$ cannot lead to a lower cost ($\cdel \geq 0$) for
the same state.

\begin{samepage}
The edges $\AGE \subseteq \AGV \times \AGV \times \Sigma_\epsilon \times
\mathbb{R}_{\geq 0}$ are built based on the edges in $\EGE$, except that the
former (i)~keep track of the position in the query $i$, and (ii)~only contain
empty edges or edges
whose label matches the next query letter:

\vspace{-0.7em}
{%
\small
\begin{alignat}{10}
	(u,v,\ell,w) &\in \EGE \implies (&\langle u, i \rangle, &\langle v, i+1
		&&\rangle,\ell,w) \in \AGE \quad \text{ for } 0 \leq i < |q| \text{ with }
		q[i]=\ell \label{TRIEeq:alignment-edges-nondeletions} \\
	(u,v,\epsilon,w) &\in\EGE \implies (&\langle u, i \rangle, &\langle v, i
		&&\rangle,\epsilon,w) \in \AGE \quad \text{ for } 0 \leq i < |q| \label{TRIEeq:alignment-edges-deletions}
\end{alignat}
}%
\end{samepage}
%
Here, assuming $0$-indexing, $q[i]$ is the next letter to be matched after
matching $i$ letters. Then, \cref{TRIEeq:alignment-edges-nondeletions} represents
matches, substitutions, and insertions (which advance the position in the query
by $1$), while \cref{TRIEeq:alignment-edges-deletions} represents deletions (which do
not advance the position in the query).

\para{Dynamic Construction}
As the size of the alignment graph is $\Oh(\lvert \RG \rvert \concat \lvert q
\rvert)$, it is expensive to build it fully for every new query.
Therefore, our implementation constructs the alignment graph $\AG$ on-the-fly:
the outgoing edges of a node are only generated on demand and are freed from
memory after alignment.

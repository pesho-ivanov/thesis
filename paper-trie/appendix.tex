% \section{Algorithms for Generative Model}

% In this section, we present runnable Python implementations of our generative model.

% \begin{figure*}
% 	\lstinputlisting[language=MyPython,caption={Full Python code for mutating a core genome. Utilities are shown in \cref{lst:utilities}.},label={lst:mutate-full}]{code/mutate_full.py}
% \end{figure*}

% \begin{figure*}
% 	\lstinputlisting[language=MyPython,caption={Full Python code for reading a segment. Utilities are shown in \cref{lst:utilities}.},label={lst:read-full}]{code/read_full.py}
% \end{figure*}

% \begin{figure*}
% 	\lstinputlisting[language=MyPython,caption={Utilities.},label={lst:utilities}]{code/utilities.py}
% \end{figure*}


\section{Equivalence of Core Genomes Model and Graph Model} \label{app:model-equivalence}
In this section, we discuss why the distributions induced by the Core Genome Model and the Graph Model are approximately equal.
To this end, we investigate why picking a random starting point in the core genome and only consider mutations from this starting point is approximately equivalent to picking a random starting point in the mutated genome (\cref{lem:ignore-early-mutations}).
The approximation is quite close; the behavior of the two processes is different in less than $1/|\genome^\gcore|$ of all cases.

\cref{lem:ignore-early-mutations} does not take into account that we must read $L$ letters from $\genome^\gmut$, starting from $\startInGenome^\gmut$.
Instead, it considers a process reading all letters starting from $\startInGenome^\gmut$.
However, it is straightforward to extend \cref{lem:ignore-early-mutations} (and its proof) to take this into account.
In this case, the two models (the Core Genome Model and the Graph Model) differ in less than $(L+1)/|\genome^\gcore|$ of all cases.

\para{Ignoring Mutations Before the Start of a Read}
The following lemma states that instead of mutating a whole core genome and picking a starting point in the mutated genome, it suffices to mutate only the part after a randomly chosen starting point in the core genome.
\begin{lemma}[] \label{lem:ignore-early-mutations}
	When starting from a fixed core genome $\genome^\gcore$, the following processes approximately induce the same output distributions:
	\begin{itemize}
		\item Sample $\genome^\gmut$ according to mutations $M$
		\item Uniformly sample starting point $\startInGenome^\gmut$
		\item Return $\genome^\gmut[\startInGenome^\gmut:]$ and $M[\startInGenome^\gmut:]$
	\end{itemize}
	and
	\begin{itemize}
		\item Uniformly sample starting point $\startInGenome'^\gcore$ in $\genome^\gcore$
		\item Sample $\seq'^\gmut$ according to mutations $M'$ on $\genome^\gcore[\startInGenome'^\gcore:]$, assuming $M'$ does not start with a deletion
		\item Return $\seq'^\gmut$ and $M'$
	\end{itemize}
	Here, $\genome^\gmut[\startInGenome^\gmut:]$ ignores all letters before the $\startInGenome^\gmut$-th letter and $M[\startInGenome^\gmut:]$ ignores all mutations that occurred before producing the $\startInGenome^\gmut$-th letter from $\genome^\gmut$.
\end{lemma}
\begin{proof}
	As stated in \cref{lem:uniform-position}, picking $\startInGenome^\gmut$ in $\genome^\gmut$ is equivalent to picking $\startInGenome'^\gcore$ in $\genome^\gcore$, unless $\startInGenome'^\gcore = |\genome^\gcore|+1$.
	As the latter happens very rarely (with a probability of less than $1/|\genome^\gcore|$), we can safely ignore this case.

	Therefore, there is a one-to-one correspondence between the mutations $M[\startInGenome^\gmut:]$ and $M'$ (here, it is crucial that $M'$ does not start with a deletion). Hence, their respective distributions are the same.
	As a consequence, the distribution of $\genome^\gmut[\startInGenome^\gmut:]$ and $\seq'^\gmut$ is also the same, as they are deterministically derived from their respective mutations.
\end{proof}

\para{Uniform Position in $\genome^\gcore$}
The following lemma states that picking a uniform point in the mutated genome $\genome^\gmut$ is equivalent to picking a uniform point in the core genome $\genome^\gcore$.

It assumes one-based indexing for $\startInGenome^\gcore$ and $\startInGenome^\gmut$. 
\begin{lemma}[] \label{lem:uniform-position}
	Picking a uniform point $\startInGenome^\gmut \uar \{1,\dots,|\genome^\gmut|\}$ and determining its origin $\startInGenome^\gcore$ yields a uniformly picked $\startInGenome^\gcore \uar \{1,\dots,|\genome^\gcore|\}$, given that $\startInGenome^\gcore \neq |\genome^\gcore|+1$.
\end{lemma}
\begin{proof}
	We prove this lemma by induction over the number of insertions ($n$) and deletions ($m$), always assuming that $\startInGenome^\gcore < |\genome^\gcore| + 1$
	Then, as the statement is true for any fixed number of insertions and deletions, it is also true for arbitrary numbers.
	We do not consider edits, as they are irrelevant for the selection of $\startInGenome^\gcore$ and $\startInGenome^\gmut$.

	\para{Base Case}
	The base case holds because for $n=0$ and $m=0$, picking $\startInGenome^\gmut$ is analogous to picking $\startInGenome^\gcore$.

	\para{Induction Step (insertions)}
	For the induction step where we add one insertion, recall that the mutations $M$ describe all operations when copying $\genome^\gcore$.

	We observe that the distribution of $M$ for $n+1$ insertions and $m$ deletions equals the distribution resulting from adding an insertion to the distribution of $M'$ for $n$ insertions and $m$ deletions, where we sample the position of the new insertion uniformly over all possible positions in $M'$.
	This is because due to symmetry, all locations for the new insertion are equally likely.

	Given $M$, assuming that we do not select the newly inserted position, $\startInGenome^\gcore$ is distributed uniformly, according to the induction hypothesis.
	Otherwise, as every position is equally likely to be selected for the insertion,$\startInGenome^\gcore$ is also uniformly distributed.
	Here, it is important that $\startInGenome^\gcore \neq |\genome^\gcore|+1$, as adding a new insertion to the end increases the probability that $\startInGenome^\gcore$ lies at the end (\ie that $\startInGenome^\gcore = |\genome^\gcore|+1$).

	\para{Induction Step (deletions)}
	For the induction step where we add one deletion, the distribution of $M$ for $n$ insertions and $m+1$ deletions equals the distribution resulting from replacing one copy transition by a deletion transition in $M'$ ($M'$ again consists of $n$ insertions and $m$ deletions).
	Since every copy transition has the same probability of being replaced by a deletion, the distribution of $\startInGenome^\gcore$ remains uniform.
\end{proof}
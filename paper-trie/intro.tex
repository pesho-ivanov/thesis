%%%%%%%%%%%%%%%%%%%%%%%%%%%%%%%%%
\section{Introduction}
%%%%%%%%%%%%%%%%%%%%%%%%%%%%%%%%%

% Our scope
Formally, we consider the optimal \textit{sequence-to-graph alignment} problem,
the task of finding an optimal base-to-base correspondence between a query
sequence and a (possibly cyclic) walk in the graph. Related alignment problems
have already been formulated as graph shortest path
problems~\cite{jain_complexity_2019}.

% from abstract
Motivation: all optimal algorithms have to iterate over the whole reference for
each query. We should be able to optimize this by precomputation.

We present an algorithm for the \emph{optimal alignment} of sequences to
\emph{genome graphs}. It works by phrasing the edit distance minimization
task as finding a shortest path on an implicit alignment graph. To find a
shortest path, we instantiate the \A paradigm with a novel domain-specific
heuristic function that accounts for the upcoming subsequence in the query
to be aligned, resulting in a provably optimal alignment algorithm called
\astarix.

\quad \quad Experimental evaluation of \astarix shows that it is 1--2
orders of magnitude faster than state-of-the-art optimal algorithms on the
task of aligning Illumina reads to reference genome graphs. Implementations
and evaluations are available at \mbox{\astarixurl}.

keywords: Next-generation sequencing, Optimal alignment, Genome graph,
Shortest path, \A algorithm

%%%%%%%%%%%%%%%%%%%%%%%%%%%%%%%%%
\subsection{Main Contributions}
We introduce a novel approach, called \astarix, for optimal sequence-to-graph
alignment based on \A. As with any \A instantiation, the core difficulty lies in
developing an accurate domain-specific heuristic which is fast to compute. We
design a heuristic that accounts for the content of the upcoming query letters
to be aligned, which more effectively guides the search. Our proposed heuristic
has two advantages: (i)~it is correctness-preserving, that is, it preserves the
fact that \astarix finds the best alignment, yet (ii)~it is practically
effective in that the algorithm performs a near-optimal number of steps.
Overall, this heuristic enables \astarix to compute the best alignment while
also scaling to larger reference graph sizes when compared to existing
state-of-the-art optimal aligners.

\begin{samepage}
Our main contributions include:
	
\begin{enumerate}
	\item \textbf{\astarix.} An algorithm for optimal sequence-to-graph
	alignment based on a novel instantiation of \A with an accurate
	domain-specific heuristic that accounts for the upcoming query letters to be
	aligned (\cref{TRIEsec:astarix-algo}).
	\item \textbf{Algorithmic optimizations.}
	To ensure that \astarix is practical, we introduce a number of algorithmic
	optimizations which increase performance and decrease memory footprint
	(\cref{TRIEsec:optimizations}). We also prove that all optimizations are
	correctness-preserving.
	\item \textbf{Thorough experimental evaluation of \astarix.}
	We demonstrate that \astarix is up to 2 orders of magnitude faster than
	other optimal aligners on various reference graphs (\cref{TRIEsec:eval}).
\end{enumerate}
\end{samepage}

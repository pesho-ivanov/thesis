\subsection{Setting}
All evaluations were executed singled-threaded on an Intel Core i7-6700 CPU running
at 3.40GHz.

\para{Reference Graphs and Reads}
We designed three experiments utilizing three different reference graphs (in
\cref{TRIEtab:results}). The first is a linear graph without variation based on the
\textit{E.~coli} reference genome (strain: K-12 substr. MG1655,
ASM584v2~\cite{howe2019ensembl}). The other two are variation graphs taken from
the \pasgal evaluations~\cite{jain_accelerating_2019}: they are based on the
Leukocyte Receptor Complex (LRC, with \numprint{1099856} nodes and
\numprint{1144498} edges), and the Major Histocompatibility Complex (MHC1, with
\numprint{5138362} nodes and \numprint{5318019} edges).
%
We note that we do not evaluate on de Brujin graphs, since \pasgal does not
support cyclic graphs.

%\para{Reads}
For the \textit{E.~coli} dataset we used the ART tool~\cite{huang_art_2012} to simulate an
Illumina single-end read set with \numprint{10000} reads of length 100. For the LCR and
MHC1 datasets, we sampled \numprint{20000} single-end reads of length 100 from the already
generated sets in~\cite{jain_accelerating_2019} using the
Mason2~\cite{holtgrewe_mason_2010} simulator.

For \dijkstra and \astarix, the runtime complexity depends not only on the data
size, but also on the data content, including edit costs. More accurate
heuristics lead to better \A performance~\cite{pearl_discovery_1983}, which is
why we evaluate \astarix with costs corresponding more closely to Illumina error
profiles: $\Delta=(0,1,5,5)$.

\para{Metrics}
As all aligners evaluated in this work are provably optimal, we are mostly
interested in their performance.
%
To study the end-to-end performance of the optimal aligners, we use the
Snakemake~\cite{koster_snakemakescalable_2012} pipeline framework to measure the
execution time of every aligner (including the time spent on reading and
indexing the reference graph input and outputting the resulting alignments). We
note that the alignment phase dominates for all tools and experiments.

To judge the potential of heuristic functions, we measure not only the runtime
but also the number of states explored by \astarix and \dijkstra. This number
reflects the quality of the heuristic function rather than the speed of
computation of the heuristic, the implementation and the system parameters.
%"Fig. 3" (it also discusses explored states)}.
%The number of explored states is a more direct indicator for the algorithm's performance
%than the number of expanded states since the algorithm has to generate and consider
%all the neighbors of the states it expands.
%\todo{Harun: this argument only really works
%if computation of the heuristic is not a bottleneck. maybe make that more clear?}
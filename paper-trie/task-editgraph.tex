\para{Edit Graph}
Instead of representing alignments as pairs of (i)~paths in the reference graph and
(ii)~sequences of edit operations on these paths, we introduce \textit{edit
graphs} whose paths intrinsically capture both. This way, we can
formally define an alignment more conveniently as a path in an edit graph.

Formally, an \emph{edit graph} $\EG:=(\EGV,\EGE)$ has directed, labeled edges
$\EGE \subseteq \EGV \times \EGV \times \Sigma_\epsilon \times \mathbb{R}_{\geq
0}$ with associated costs that account for edits. Here, $\Sigma_\epsilon :=
\Sigma \cup \{\epsilon\}$ extends the alphabet $\Sigma$ by $\epsilon$ to account
for deleted characters (see \cref{fig:graph-constructions}).
%
The edit and reference graphs consist
of the same vertices, \ie, $\EGV=\RGV$. However, $\EGE$ contains more edges
than $\RGE$ to account for edits.
%
Concretely, for each edge $(u,v,\ell) \in \RGE$, $\EGE$ contains edges to
account for (i)~matches, by an edge $(u,v,\ell,\cmatch)$, (ii)~substitutions, by
edges $(u,v,\ell',\csubst)$ for each $\ell' \in \Sigma \backslash \ell$,
(iii)~deletions, by an edge $(u,v,\epsilon,\cdel)$, and (iv)~insertions, by
edges $(u,u,\ell',\cins)$ for each $\ell' \in \Sigma$.
%
The spelling $\sigma(\pi) \in \Sigma^*$ of a path $\pi \in \EG$ is defined
analogously to reference graphs, except that deleted letters (represented by
$\epsilon$) are ignored. The cost $\cost{\pi}$ of a path $\pi \in \EG$ is the
sum of all its edge costs.

\para{Alignment on Edit Graph}
An \emph{alignment} of query $q$ to $\RG$ is a path $\pi \text{ in } \EG$
spelling $q$, \ie, $q=\sigma(\pi)$. An \emph{optimal alignment} is an alignment
of minimal cost.

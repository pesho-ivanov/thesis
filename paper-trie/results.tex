\section{Results}

TODO table: region (small and complex), real use case, realistic assumpitions, alignment specific measurements, feasible, strengths (graph, edits, phreds)

we will evaluate our tool in two phases, where the first phase is directly evaluating the alignment accuracy of our approach, and the second phase leverages our alignment to solve the high-level task of VDJ annotation (from a read, identifying which regions are from which variant of V, D, and J)

1. the first phase will be a validation of our approach with simulated data. for the graph, we will take a VDJ graph (possibly with transition probabilities taken from the second phase), and manually picked mutation and phred probabilities. we will then evaluate the precision of the alignment, against the ground truth (which we have because this is a simulation). as a baseline, we will also try edit distance. we may also turn mutations and phred values on\&off to see the effect of this
2. the second phase will be on the high-level application of VDJ annotation (from a read, identifying which regions are from which variant of V, D, and J). this has been done before (in iGOR and partis), and partis has a simulator we can leverage. based on this simulator, we construct a graph by (i) manually building a graph skeleton and (ii) inferring transition probabilities by relative frequencies of simulated reads. then, we evaluate by (i) simulating a read (from the existing simulator), (ii) adding noise (edits + phreds), (iii) aligning the resulting read (using the graph from before) and (iv) annotating the aligned read.

We follow the testing protocol of \cite{garrison2017sequence}.
They have asked different groups to generate variant graphs.
Then they simulated reads from these graphs, mapped them onto the graphs and measured the mapping accuracy.
We reuse the structure of these graphs and we additionally generate transition and edit probabilities for the graphs as well as phred probabilities for the reads.
For simplicity and generality, we use uniform distributions in chosen intervals instead of fine-tuned read simulators for specific technologies.

\subsection{Experiments}
To analyse the importance and behaviour of different mapping modes in our framework, we independently test all combinations of uncertainty sources.
According to \ref{fig:VennComparison} there are 7 modes (defined by the activated sources of undertainty):
\begin{enumerate}
	\item \textit{Variants} -- Mapping non-probabilistic reads to a graph reference without making edits (equivalent to the exact mapping in vg).
	\item \textit{Edits} -- Mapping non-probabilistic reads to a linear reference with edit allowed (equivalent to \cite{rautiainen2017aligning}).
	\item \textit{Phred} -- Mapping probabilistic reads to a linear reference without making edits (MAQ \cite{maq2008mapping}, \fxwarning{Check MAQ}).
	\item \textit{Variants + Edits} -- Mapping non-probabilistic reads to a graph reference with edits allowed (\fxwarning{Is this possible in vg?}).

	\item \textit{Variants + Phred} -- 
	\item \textit{Edits + Phred} -- 
	\item \textit{Variants + Edits + Phred} --
\end{enumerate}

First try: measure only the percentage of correct alignments
partis VDJ: goals -- 1) increase the D gene prediction accuracy (because of another construction; because of phreds; because of indels), 2) report sub-allele-level of annotation, 4),  3) ideally, report a mapping score

Evals scheme: generate TCR/BCER sequences using the simulator from IGoR'18\cite{marcou2018high}, (2) align using SW and \tool 

Cross-align: reads from one specie on graph build from another specie(es).

Plots correctly mapped reads vs incorrectly mapped reads\cite{frith2010parameters}.

sensitivity, p-value (\fxwarning{Mentioned in \cite{frith2010parameters}. What would be the null hypothesis?}), E-value

Variants, Mendelian accuracy\cite{eggertsson2017graphtyper}

For both graph construction and read sampling we assume that the genomes follow to a geometric distribution (similarly to \cite{danko2019minerva}).

Compare paths

\subsection{Datasets}
opensnp.org ?
Human genome?, MHC, HLA reference alleles were fetched from the IPD-IMGT/HLA database\cite{robinson2014ipd} (analoguous experimet to Graphtyper\cite{eggertsson2017graphtyper}), TCR/BCR

\subsection{Criteria}
``in the evaluation, a mapped read is considered to be correct if it overlaps the true mapping (we call it ‘Mapping accuracy’). This means that the detailed alignment between the mapped read and the reference genome, for example, ‘Aligned column accuracy’ or ‘Gap accuracy’, is not always correct even if the mapping is correct. It is possible that probabilistic alignment improves those accuracy measures.''\cite{hamada2011probabilistic}

We measure mapping correctness, alignment accuracy and performance.
We prove the output from all solutions approaches to be the same so we compare the DP, Dijkstra and A$^\star$ on performance.

\subsection{Results}
In terms of performance, we note that the Dijkstra approach is faster than the DP but still comparable whereas the A$^\star$ rulzz \fxnote{verify :\")}.

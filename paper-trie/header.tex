%\RequirePackage{amsmath}
%\documentclass[runningheads]{llncs}

%\usepackage{geometry}
%\geometry{
  %a4paper,         % or letterpaper
  %textwidth=15cm,  % llncs has 12.2cm
  %textheight=24cm, % llncs has 19.3cm
  %heightrounded,   % integer number of lines
  %hratio=1:1,      % horizontally centered
  %vratio=2:3,      % not vertically centered
  %margin=1.0in
%}

%\setlength{\textwidth}{15.2cm}

%\raggedbottom
%\setlength{\textfloatsep}{5pt}
%\setlength{\abovecaptionskip}{5pt plus 3pt minus 2pt} % Chosen fairly arbitrarily
\usepackage{graphicx}     % Used for displaying a sample figure. If possible,figure files should be included in EPS format.

\usepackage{wrapfig}
\usepackage[T1]{fontenc}  %% for small caps \textsc
\usepackage{libertine}
%\usepackage{hyperref}     %% TODO: remove
\usepackage{xspace}
\usepackage{amsmath}
\usepackage{amssymb}
\usepackage{amsfonts}
\usepackage{booktabs}
\newcommand{\ra}[1]{\renewcommand{\arraystretch}{#1}}

\usepackage{multirow}
\usepackage{xr}  % for importing external .aux (for the appendix)
\usepackage{filecontents}  % for embedding the .aux into .tex

\usepackage{numprint}
\npdecimalsign{.}

% add qed to all proofs
\let\oldProof\proof
\let\oldEndProof\endproof

\renewenvironment{proof}
{\oldProof}
{\qed \oldEndProof}

\usepackage{subcaption}  % for subfigures 
\captionsetup{compatibility=false}  % a workaround for "The `subcaption' package does not work correctly in compatibility mode."

%%%%%%%%%%%%%%%%%%%%%%%%%
% ALGORITHM DEFINITIONS %
%%%%%%%%%%%%%%%%%%%%%%%%%
\usepackage{algorithm}
%\usepackage{algorithmicx}
\usepackage[noend]{algpseudocode}  % layout for algorithmicx
\algdef{SE}[SUBALG]{Indent}{EndIndent}{}{\algorithmicend\ }%
\algtext*{Indent}
\algtext*{EndIndent}

%\usepackage{minted}

\usepackage[dvipsnames]{xcolor}
\usepackage{fontawesome}
\usepackage[clock]{ifsym}  % special symbols
%%\usepackage[linesnumbered,ruled,vlined,algo2e]{algorithm2e}

% temporary (for debugging)
\usepackage[]{todonotes}  % [disable]

\usepackage{standalone}
%\usepackage[hidelinks]{hyperref}

% Description: This file enables using \cref to references Figures, Sections,
% Equations, etc.
%
% Usage: \input{header/references}

%%%%%%%%%%%%%%
% REFERENCES %
%%%%%%%%%%%%%%

% Package documentation:
% http://ftp.math.purdue.edu/mirrors/ctan.org/macros/latex/contrib/cleveref/cleveref.pdf

% PACKAGE OPTIONS:
%
% capitalize: always capitalize cross-reference names, regardless of where they
% appear in the sentence, writing Theorem 1 and Equation 3 (as opposed to
% theorem 1 and equation 3)
%
% noabbrev:  avoid abbreviations (e.g. use Figure instead of Fig.). Note: To
% avoid all abbreviations, you must also check all manually defined reference
% names in this file.

%\usepackage[capitalize]{cleveref}

% OVERRIDE CREF FORMAT:
%
% Override the cref format, e.g., for sections to: §1.2
%
% #1: formatted version of the label counter
%
% #2, #3: beginning and end of the part of the cross-reference that forms the
% hyperlink
%
% Example: Override the cref format for equations to, e.g., Eq.~(1)
%
% \crefformat{equation}{Eq.~(#2#1#3)}

\crefformat{section}{\S#2#1#3}

% OVERRIDE CREF FORMAT FOR RANGES:
%
% override the cref format for ranges of sections, e.g.: §1.2 to §1.3
%
% #1, #2: formatted versions of the two label counters defining the reference
% range
%
% #3, #4: denote the beginning and end of the hyperlink for the first reference
%
% #5, #6: denote the beginning and end of the hyperlink for the second reference
\crefrangeformat{section}{\S#3#1#4\crefrangeconjunction\S#5#2#6}

% OVERRIDE CREF FORMAT FOR LISTS:
%
% override the cref format for multiple sections, e.g.,:
%
% Argument 1: the cross-reference type
%
% Argument 2: the format for the first cross-reference in a list
%
% Argument 3: the format for the second cross-reference in a list of two
%
% Argument 4: the format for the middle cross-references in a list of more than two
%
% Argument 5: the format for the last cross-reference in a list of more than two
\crefmultiformat{section}{\S#2#1#3}{\crefpairconjunction\S#2#1#3}{\crefmiddleconjunction\S#2#1#3}{\creflastconjunction\S#2#1#3}

% ADAPT CONJUNCTION
%
% Adapt the conjunction used in a reference range, to, e.g.: Figs. 1-2
\newcommand{\crefrangeconjunction}{--}

% CUSTOMIZE/ADD REFERENCE NAME
%
% Customize the cross-reference name for a given cross-reference type
%
% Argument 1: the cross-reference type
% 
% Argument 2: singular form of name
%
% Argument 3: plural form of name
% 
% Examples:
% 
% \crefname{section}{Sec.}{Sections}
% \crefname{theorem}{Thm.}{Thms.}
% \crefname{lstlisting}{Listing}{listings}

\crefname{listing}{Lst.}{listings}
%\crefname{line}{Lin.}{Lin.}
\crefname{appendix}{App.}{App.}

%%%%%%%%%%%%%%%%%%%%%%%%%%%
% OPTIONAL CUSTOMIZATIONS %
%%%%%%%%%%%%%%%%%%%%%%%%%%%

% Alias a counter to a different cross-reference type.
%
% Example: Write Fig.~5 for \cref{sec:abc}
%
% \crefalias{section}{figure}

%%%%%%%%%%%%
% APPENDIX %
%%%%%%%%%%%%

\newcommand{\app}[1]{%
	\ifbool{includeappendix}{\cref{#1}}{the appendix}%
}
\newcommand{\App}[1]{%
	\ifbool{includeappendix}{\cref{#1}}{The appendix}%
}


% If you use the hyperref package, please uncomment the following line
% to display URLs in blue roman font according to Springer's eBook style:
\renewcommand\UrlFont{\color{blue}\rmfamily}

%% custom names
%\newcommand{\astarix}[0]{\textsc{AStarix}\xspace}
\newcommand{\astarixurl}[0]{\url{https://github.com/eth-sri/astarix}\xspace}
\newcommand{\astarixurlwithbranch}[0]{\url{https://github.com/eth-sri/astarix/tree/recomb2020}\xspace}

%\newcommand{\dijkstra}[0]{\textsc{Dijkstra}\xspace}
\newcommand{\graphaligner}[0]{\textsc{GraphAligner}\xspace}
\newcommand{\bitparallel}[0]{\textsc{BitParallel}\xspace}
%\newcommand{\cellwise}[0]{\textsc{Cellwise}\xspace}  % from GraphAligner's paper
\newcommand{\brownie}[0]{\textsc{BrownieAligner}\xspace}
\newcommand{\pasgal}[0]{\textsc{PaSGAL}\xspace}
\newcommand{\vg}[0]{\textsc{VG}\xspace}
\newcommand{\valigntool}[0]{\textsc{V-ALIGN}\xspace}

\newcommand{\BF}[0]{BF$^\star$\xspace}
%\newcommand{\A}[0]{A$^\star$\xspace}
\newcommand{\K}[0]{K$^\star$\xspace}
\newcommand{\suf}[0]{\text{suff}}
\newcommand{\impl}[0]{\textbf{(implemented)}}

% reference graph
\newcommand{\reference}[1]{#1_\texttt{r}}
\newcommand{\RG}[0]{\reference{G}}
\newcommand{\RGV}[0]{\reference{V}}
\newcommand{\RGE}[0]{\reference{E}}
% trie
\newcommand{\trie}[1]{#1_\texttt{r}^\texttt{+}}
\newcommand{\TG}[0]{\trie{G}}
\newcommand{\TGV}[0]{\trie{V}}
\newcommand{\TGE}[0]{\trie{E}}
% edit graph
\newcommand{\edit}[1]{#1_\texttt{e}}
\newcommand{\EG}[0]{\edit{G}}
\newcommand{\EGV}[0]{\edit{V}}
\newcommand{\EGE}[0]{\edit{E}}
% alignment graph
\newcommand{\alignment}[2][q]{#2_\texttt{a}^{#1}}
\newcommand{\AG}[1][q]{\alignment[#1]{G}}
\newcommand{\AGV}[1][q]{\alignment[#1]{V}}
\newcommand{\AGE}[1][q]{\alignment[#1]{E}}

% stats
%\newcommand{\st}[2]{\langle #1, #2 \rangle}

% cost
%\newcommand{\cost}[1]{\text{cost}(#1)}

% PARAMETERS
% c
\newcommand{\costcap}[0]{c}

% edit distance
%\newcommand{\cedits}[0]{\Delta}
\newcommand{\cmatch}[0]{\Delta_\text{match}}
\newcommand{\csubst}[0]{\Delta_\text{subst}}
\newcommand{\cins}[0]{\Delta_\text{ins}}
\newcommand{\cdel}[0]{\Delta_\text{del}}
\newcommand{\dist}[0]{\mli{ED_\Delta}}

% paragraphs
\newcommand{\para}[1]{\vspace{0.8em}\noindent\textbf{#1.}}

% general math
\DeclareMathOperator*{\argmax}{arg\,max}
\DeclareMathOperator*{\argmin}{arg\,min}
\newcommand{\mli}[1]{\mathit{#1}}
%\newcommand{\Oh}[0]{\mathcal{O}}
\newcommand{\concat}[0]{{\cdot}}

% efficiency icons
%\newcommand{\muchfaster}[0]{\begingroup\color{green}\StopWatchEnd\endgroup\,}
%\newcommand{\muchslower}[0]{\begingroup\color{red}\StopWatchEnd\endgroup\,}
%\newcommand{\littlefaster}[0]{\begingroup\color{green}\textsuperscript{\StopWatchEnd}\endgroup\,}
%\newcommand{\littleslower}[0]{\begingroup\color{red}\textsuperscript{\StopWatchEnd}\endgroup\,}
%\newcommand{\muchsmaller}[0]{\begingroup\color{green}\faDatabase\endgroup\,}
%\newcommand{\muchbigger}[0]{\begingroup\color{red}\faDatabase\endgroup\,}
%\newcommand{\littlesmaller}[0]{\begingroup\color{green}\textsuperscript{\faDatabase}\endgroup\,}
%\newcommand{\littlebigger}[0]{\begingroup\color{red}\textsuperscript{\faDatabase}\endgroup\,}

%%%%%%%%%%
% COLORS %
%%%%%%%%%%

% blue
\definecolor{my-full-blue}{HTML}{1F77B4}
%\definecolor{blue}{RGB}{31,119,180} 

% orange
\definecolor{my-full-orange}{HTML}{FF7F0E}
%\definecolor{orange}{RGB}{255,127,14}

% green
\definecolor{my-full-green}{HTML}{2CA02C}
%\definecolor{green}{RGB}{44,160,44}

% red
\definecolor{my-full-red}{HTML}{d62728}
%\definecolor{red}{RGB}{214,39,40}

% purple
\definecolor{my-full-purple}{HTML}{9467bd}
%\definecolor{purple}{RGB}{148,103,189}

% lighter shades
\colorlet{my-blue}{my-full-blue!30}
\colorlet{my-orange}{my-full-orange!30}
\colorlet{my-green}{my-full-green!30}
\colorlet{my-red}{my-full-red!30}
\colorlet{my-purple}{my-full-purple!30}
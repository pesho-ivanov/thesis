%%%%%%%%%%%%%%%%%%%
% COMMON PACKAGES %
%%%%%%%%%%%%%%%%%%%

% \usepackage[utf8]{inputenc} % enable UTF-8

% Why `\usepackage[T1]{fontenc}'?
% https://tex.stackexchange.com/questions/664/why-should-i-use-usepackaget1fontenc
% If you don't use \usepackage[T1]{fontenc},
% - Words containing accented characters cannot be automatically hyphenated,
% - You cannot properly copy-and-paste such words from the output (DVI/PS/PDF),
% - Characters like the pipe sign, less than and greater sign give unexpected
%   results in text.
%
% In addition:
% - Code in lstlisting may not be displayed correctly
% \usepackage[T1]{fontenc}

%\usepackage[usenames, dvipsnames]{color} % for textcolor
\usepackage{xcolor} % for definecolor
\usepackage{xspace}
\usepackage{float}
\usepackage[hidelinks]{hyperref}
\usepackage{booktabs}
\usepackage{chapterbib}  % without adding this package the references disappear from the middle (before the supplementaries)

\usepackage{acro} % for \ac

\usepackage{mathtools}
%\usepackage{biblatex}  % for the second biblio
\usepackage{amsmath}
\usepackage{algorithm}
\usepackage[noend]{algpseudocode}
\usepackage{graphicx} % for resizebox
\usepackage{stmaryrd} % for llbracket, rrbracket

\usepackage{pifont}
\newcommand{\y}{\ding{51}}
%\newcommand{\n}{\ding{55}}
\newcommand{\n}{-\xspace}
\newcommand{\specialcell}[2][c]{% newline in table cells with \\
  \begin{tabular}[#1]{@{}c@{}}#2\end{tabular}}


\let\oldReturn\Return
\renewcommand{\Return}{\State\oldReturn}
%\algloopdefx{Continue}{\textbf{continue}}

%%%%%%%%%%%%%%%%%
% MORE PACKAGES %
%%%%%%%%%%%%%%%%%

%\usepackage{fixme}
\usepackage[draft,inline,nomargin]{fixme}
\fxsetup{theme=color}
\usepackage{bbding} % scissors
\usepackage{fontawesome}


%%%%%%%%%%%%%%%%%
%OVERFULL BOXES %
%%%%%%%%%%%%%%%%%

% ALLOW TO MARK OVERFULL BOXES
\ifdefined\isoverfull
	%\usepackage{showframe}
	\overfullrule=1cm
\else
	% no action necessary
\fi

%%%%%%%%
% TIKZ %
%%%%%%%%

\usepackage{tikz}
\usetikzlibrary{calc,decorations.pathmorphing,shapes}
\usetikzlibrary{positioning,shapes,fit,arrows}
\usetikzlibrary{decorations.markings}
\usetikzlibrary{shapes,shapes.geometric}
\usetikzlibrary{shadows,patterns}
\usetikzlibrary{backgrounds,decorations.pathreplacing,automata}
\usetikzlibrary{shapes.multipart}

\tikzset{arrow/.style={->}}

\tikzset{editarrow/.style={arrow,densely dotted,text=gray}}

% From Drawing Finite State Machines 2018: https://www3.nd.edu/~dchiang/teaching/theory/2018/www/tikz_tutorial.pdf
%\tikzset{node distance=2.5cm, % Minimum distance between two nodes. Change if necessary.
%	every state/.style={ % Sets the properties for each state
%		semithick,
%		fill=gray!10},
%	initial text={}, % No label on start arrow
%	double distance=2pt, % Adjust appearance of accept states
%	every edge/.style={ % Sets the properties for each transition
%		draw,
%		->,>=stealth’, % Makes edges directed with bold arrowheads
%		auto,
%	semithick}}

\tikzset{letter/.style={draw,circle}}

\newcommand*\coloredcircle[2][]{%
	\raisebox{-1.2pt}{
	\tikz[]{
		\node[shape=circle,draw,#2,fill=#2,inner sep=0pt,minimum size=1em] (char) {\small\textcolor{white}{#1}};
	}
	}%
}

%%%%%%%%%%%%
% COMMENTS %
%%%%%%%%%%%%

% TODO markers
\newcommand{\todo}[1]{
	\message{^^JTODO on page \thepage: #1^^J}
	\textcolor{red}{\fxerror{#1}}
}

\newenvironment{commentlist}
    {
	\color{gray}
	\begin{itemize}
    }
    { 
    \end{itemize}
	}

%%%%%%%%%%%%%%
% REFERENCES %
%%%%%%%%%%%%%%

\newcommand{\fig}[1]{Fig.~#1}
\newcommand{\sect}[1]{\S#1}
\newcommand{\thm}[1]{Thm.~#1}
\newcommand{\defn}[1]{Def.~#1}
\newcommand{\lst}[1]{Listing~#1}
\newcommand{\eq}[1]{Eq.~(#1)}
\newcommand{\app}[1]{App.~#1}
\newcommand{\lin}[1]{Line~#1}
\newcommand{\alg}[1]{Alg.~#1}
\newcommand{\tab}[1]{Tab.~#1}

%%%%%%%%%%%%%%%%%
% ABBREVIATIONS %
%%%%%%%%%%%%%%%%%

\newcommand{\eg}{e.g., }
\newcommand{\ie}{i.e., }
\newcommand{\etc}{etc.}
\newcommand{\cp}{cp.}
\newcommand{\nth}[1]{$#1^\text{th}$}
\newcommand{\wrt}{w.r.t. }

%%%%%%%%%%
% COLORS %
%%%%%%%%%%

% blue
\definecolor{my-full-blue}{HTML}{1F77B4}
%\definecolor{blue}{RGB}{31,119,180} 

% orange
\definecolor{my-full-orange}{HTML}{FF7F0E}
%\definecolor{orange}{RGB}{255,127,14}

% green
\definecolor{my-full-green}{HTML}{2CA02C}
%\definecolor{green}{RGB}{44,160,44}

% red
\definecolor{my-full-red}{HTML}{d62728}
%\definecolor{red}{RGB}{214,39,40}

% purple
\definecolor{my-full-purple}{HTML}{9467bd}
%\definecolor{purple}{RGB}{148,103,189}

% lighter shades
\colorlet{my-blue}{my-full-blue!30}
\colorlet{my-orange}{my-full-orange!30}
\colorlet{my-green}{my-full-green!30}
\colorlet{my-red}{my-full-red!30}
\colorlet{my-purple}{my-full-purple!30}

%%%%%%%%%%%%
% THEOREMS %
%%%%%%%%%%%%

\newtheorem{theorem}{Theorem}[section]
\newtheorem{hyp}{Hypothesis}
\newtheorem{lemma}[theorem]{Lemma}

%%%%%%%%
% MATH %
%%%%%%%%

\let\epsilon\varepsilon
\let\PrOnly\Pr
\renewcommand{\Pr}[2][]{\PrOnly_{#1}\left[#2\right]}
\newcommand{\E}[2][]{\mathbb{E}_{#1}\left[#2\right]}
\newcommand{\pocc}[1]{p\left(#1\right)}
\DeclareMathOperator*{\argmin}{arg\,min}
\DeclareMathOperator*{\argmax}{arg\,max}

% KL-divergence
%   \kld{p}{q}
\DeclarePairedDelimiterX{\infdivx}[2]{(}{)}{%
	#1\;\delimsize|\delimsize|\;#2%
}
\newcommand{\kld}[2]{\ensuremath{D_{KL}\infdivx{#1}{#2}}\xspace}
\newcommand{\vardist}[2]{\delta(#1,#2)}

\newcommand{\uar}[0]{\in_{\text{u.a.r}}}

\newcommand{\abs}[1]{\left| #1 \right|}

\newcommand{\shortcolon}[0]{{:}}

\newcommand{\pins}[0]{p_\text{ins}}
\newcommand{\pdel}[0]{p_\text{del}}
\newcommand{\ped}[0]{p_\text{ed}}
\newcommand{\pcopy}[0]{p_\text{copy}}

\newcommand{\tol}[1]{\xrightarrow{#1}}
\newcommand{\tcopy}[1]{\xrightarrow{\text{copy},#1}}
\newcommand{\tedit}[1]{\xrightarrow{\text{edit},#1}}
\newcommand{\tdelete}[0]{\xrightarrow{\text{del}}}
\newcommand{\tinsert}[1]{\xrightarrow{\text{ins},#1}}

%%%%%%%%%%%%
% ACRONYMS %
%%%%%%%%%%%%

\DeclareAcronym{map} {
    short = MAP,
    long = Maximum a Posteriori Probability,
    class = abbrev
}

%%%%%%%%%
% GRAPH %
%%%%%%%%%
\newcommand{\paths}[2]{\text{Paths}_{#1}(#2)}

%%%%%%%%%%%
% GENOMES %
%%%%%%%%%%%
\newcommand{\genomes}[0]{\mathcal{G}}
\newcommand{\genomeRV}[0]{G}
\newcommand{\genome}[0]{g}

\newcommand{\segments}[0]{\mathcal{S}}
\newcommand{\segmentRV}[0]{S}
\newcommand{\segment}[0]{s}

\newcommand{\alignments}[0]{\mathcal{A}}
\newcommand{\alignmentRV}[0]{A}
\newcommand{\alignment}[0]{a}

\newcommand{\startInGenomeRV}[0]{B}
\newcommand{\startInGenome}[0]{b}

\newcommand{\graph}[0]{\mathbb{G}}

\newcommand{\position}[0]{m}
\newcommand{\labels}[0]{\ell}
\newcommand{\trans}[0]{p}
\newcommand{\graphpath}[0]{\pi}
\newcommand{\startNode}[0]{b}


% picked from http://tug.ctan.org/info/symbols/comprehensive/symbols-a4.pdf
\newcommand{\gmut}[0]{\text{\faWrench}}
% \newcommand{\gchop}[0]{\text{\ScissorRight}}
\newcommand{\gread}[0]{{\hspace{0.4pt}\text{\faEye}}}
\newcommand{\gcore}[0]{{\hspace{0.4pt}\text{\faDotCircleO}}}

%\newcommand{\fmut}[1]{\text{mut}(#1)}
%\newcommand{\fchop}[1]{\text{chop}(#1)}
%\newcommand{\fread}[1]{\text{read}(#1)}

% outdated
\newcommand{\seq}[0]{S}

\newcommand{\pphred}[1]{\text{phred}_{#1}}

\newcommand{\vg}[1]{\text{VG}(#1)}

%%%%%%%%%%%
% TEXTUAL %
%%%%%%%%%%%

\newcommand{\tool}[0]{\textsc{pVG}\xspace}
\newcommand{\completem}[0]{Complete Genomes Model\xspace}
\newcommand{\corem}[0]{Core Genomes Model\xspace}
\newcommand{\graphm}[0]{Graph Model\xspace}
\newcommand{\para}[1]{\paragraph{\textbf{#1}.}}

\newcommand{\fullline}[0]{%
\vspace{0.3em}%
\rule{\linewidth}{0.4pt}%
\vspace{0.3em}%
}


%%%%%%%%
% CODE %
%%%%%%%%
% Description: This file defines a nice listing environment.
%
% Usage: You may need to modify the used packages below, depending on what other
% packages you include in your project


%%%%%%%%%
% SETUP %
%%%%%%%%%
% import relevant packages

\usepackage{listings}

% pro­vide many text sym­bols (such as baht, bul­let, copy­right,
% mu­si­cal­note, onequar­ter, sec­tion, and yen)
%
% needed for correct display of quotation marks (see setting `upquote=true`
% below)
\usepackage{textcomp}

% `xcolor` may already be loaded...
%
% We need the xcolor package to define the colors of different textual elements
\usepackage{xcolor}

% typically already loaded, needed for parsing UTF-8 code files
% \usepackage[utf8]{inputenc} % enable UTF-8

% typically already loaded, needed to correctly display beramono
% \usepackage[T1]{fontenc}

%%%%%%%%
% FONT %
%%%%%%%%
% import a nice typewriter font, improves readability of code
%
% List of options: http://www.tug.dk/FontCatalogue/typewriterfonts.html

% BERA
%
% Examples: http://www.tug.dk/FontCatalogue/beramono/
%
% Documentation: http://texdoc.net/texmf-dist/doc/fonts/bera/bera.txt
%
% "Bera" is a set of three PostScript Type1 font families:
% Bera Serif (a slab-serif Roman), Bera Sans (a "Frutiger
% descendant") and Bera Mono (monospaced/typewriter).
%
% - T1 and textcompanion encoding is selected
% - Bera Roman, Sans and Mono are loaded as the three 
%   main text font families (while the math fonts remain 
%   unchanged!)
% - the line spacing is enlarged by 5%, \ie,
%   \linespread{1.05}, with respect to the large x-height of
%   the Bera typefaces;
% - the definitions of the TeX and LaTeX logos \TeX and \LaTeX
%   are changed so as to suit BeraSerif.
% - `scaled=0.8`: scales down the letters to 80% of their "natural" size.
%
% WARNING:
% bera does not support all font encodings (requires T1)

\usepackage[scaled=0.8]{beramono}

% FIRA MONO
%
% Examples: http://www.tug.dk/FontCatalogue/firamono/
% 
% Documentation: https://ctan.org/tex-archive/fonts/fira?lang=en
%
% - activate Fira Mono as the monospaced text font
% - Options scaled=<number> or scale=<number> may be used to scale the fonts
% - Font encodings supported are OT1, T1, TS1, LY1 and LGR.
%

%\usepackage[scaled=0.8]{FiraMono}

%%%%%%%%%%
% COLORS %
%%%%%%%%%%
% define colors needed for syntax highlighting

\definecolor{ckeyword}{HTML}{7F0055}
\definecolor{ccomment}{HTML}{3F7F5F}
\definecolor{cstring}{HTML}{2A0099}

%%%%%%%%%%%%%%%%%%%
% DEFINE LANGUAGE %
%%%%%%%%%%%%%%%%%%%
% define a default language with standard, but nice, syntax highlighting
%
% Full documentation available at:
% http://texdoc.net/texmf-dist/doc/latex/listings/listings.pdf

% style for displaying line numbers
\lstdefinestyle{numbers}{
	% display line numbers on the left
	numbers=left,
	%
	% if code is framed, extend the frame to the left, to fit the line numbers
	framexleftmargin=20pt,
	%
	% determines the font and size of the numbers
	numberstyle=\tiny,
	%
	% `auto` lets the package choose the first number: a new listing starts with
	% number one, a named listing continues the most recent same-named listing
	% (named by `name=abc`), and a stand alone file begins with the number
	% corresponding to the first input line.
	firstnumber=auto,
	%
	% Distance between number and listing. Write line numbers closer to code
	numbersep=1em,
	%
	% Extra margin on left, aligns line number with text
	xleftmargin=2em
}

% style for general layouting of listings
\lstdefinestyle{layout}{
	% do not show frame
	frame=none,
	% put line on top and bottom
	%frame=tb,
	%
	% position the caption at the bottom
	captionpos=b,
}

\lstdefinestyle{comment-style}{
	% allow comments with // comment
	morecomment=[l]//,
	%
	% allow comments with /* comment */
	morecomment=[s]{/*}{*/},
	%
	% determines the style of comments
	commentstyle={\color{ccomment}\itshape},
}

\lstdefinestyle{string-style}{
	%
	% allow strings with "string"
	morestring=[b]",%
	%
	% allow strings with 'string'
	morestring=[b]',%
	%
	% determines the style of strings
	stringstyle={\color{cstring}},
	%
	% do not display black spaces in strings as ␣
	showstringspaces=false,%
}

\lstdefinestyle{keyword-style}{
	%
	% determines the style of keywords
	keywordstyle={\ttfamily\bfseries},
	%
	% add to keywords from keyword list
	morekeywords={
		function,
		constructor,
		int,
		bool,
		return,
		returns,
		uint
	},
	%
	% Add more keywords, with a special style
	morekeywords = [2]{},
	keywordstyle = [2]{\text},
	%
	% Introduce @ as a separator of keywords
	% otherkeywords={@},
	% morekeywords = [3]{@},
	% keywordstyle = [3]{},
	%
	% keywords are case sensitive
	sensitive=true,
}

\lstdefinestyle{input-encoding}{
	% determines the input encoding. The usage of this key requires the
	% `inputenc` package; nothing happens if it’s not loaded.
	inputencoding=utf8,
	%
	%
	% Allows extended characters in listings, that means (national) characters
	% of codes  128–255. If you use extended characters, you should load
	% `fontenc` and/or `inputenc`, for example
	extendedchars=true,
	%
	% replace strings in original listings
	%
	% {string to replace}{replacement text}{length of replacement text; number of characters}
	literate=
	{ℝ}{$\reals$}1%
	{→}{$\rightarrow$}1%
	{α}{$\alpha$}1%
	{β}{$\beta$}1%
	{λ}{$\lambda$}1%
	{θ}{$\theta$}1%
	{ϕ}{$\phi$}1%
}

\lstdefinestyle{escaping}{
	%
	% color everything marked by % in blue: %color this%
	moredelim={**[is][\color{blue}]{\%}{\%}},
	%
	% escapes the user to LATEX: all code between two such characters is
	% interpreted as LATEX code
	%
	% allow adding labels for line numbers
	escapechar=|,
	%
	% Activates special behavior of the dollar sign.  If activated a dollar sign
	% acts as TEX’s text math shift.
	%
	% This key is useful if you want to typeset formulas in listings
	mathescape=true
}

\lstdefinestyle{default-style}{
	%
	% Style selected at the beginning of each listing
	% ttfamily: selects a monospaced (typewriter) font family
	% fontencoding: selects T1 fontencoding (required for correct display in combination with the `beramono` package)
	% footnotesize: controls size of letters
	basicstyle=\fontencoding{T1}\ttfamily\footnotesize,
	%
	style=numbers,
	%
	style=layout,
	%
	style=comment-style,
	%
	style=string-style,
	%
	style=keyword-style,
	%
	style=input-encoding,
	%
	style=escaping,
	%
	%
	% Activates/deactivates automatic line breaking of long lines
	%breaklines=false,
	%
	% number of spaces to use for tabs
	tabsize=2,
	%
	% determines whether the left and right quote are printed ‘’ or `'. This key
	% requires the textcomp package if true. 
	upquote=true
}


\lstdefinelanguage{BASIC}{
	% Base language on C++
	language=C++,
	%
	style=default-style
}[keywords,comments,strings]%

% set default language
\lstset{language=BASIC}



%%%%%%%%
% CREF %
%%%%%%%%
\usepackage[capitalize]{cleveref}
\crefformat{section}{\S#2#1#3}
\crefname{listing}{Lst.}{listings}
\crefname{line}{Lin.}{Lin.}
\newcommand{\crefrangeconjunction}{--}
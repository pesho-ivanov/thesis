\subsection{Background: General \A algorithm} \label{subsec:general-astar}
Given a weighted graph $G=(V,E)$ with $E \subseteq V \times V \times
\mathbb{R}_{\geq 0}$, the \A algorithm (abbreviated as \A) searches for the
shortest path from sources $S \subseteq V$ to targets $T \subseteq V$. It is an
extension of Dijkstra's algorithm that additionally leverages a \emph{heuristic
function} $h \colon V \to \mathbb{R}_{\geq 0}$ to decide which paths to explore
first.
%
If $h(u) \equiv 0$, \A is equivalent to Dijkstra's algorithm.
%
We provide an implementation of \A and Dijkstra in \cref{app:astar}, but do not
assume knowledge of either algorithm in the following.
%
At a high level, \A maintains the set of all \emph{explored} states, initialized
with the set of sources $S$. Then, \A iteratively \emph{expands} the explored
state with lowest estimated cost by exploring all its neighbors, until it finds
a target. Here, the cost for node $u$ is estimated by the distance from source, called $g(u)$, plus the estimate from the heuristic $h(u)$.


\para{Heuristic Function}
The heuristic function $h(u)$ estimates the
cost $h^*(u)$ of a shortest path in $G$ from $u$ to a target $t \in T$. Intuitively, a
good heuristic correlates well with the distance from $u$ to $t$.

To ensure that \A indeed finds the shortest path, $h$ should be
\emph{optimistic}:

\begin{definition}[Optimistic heuristic] A heuristic $h$ is \textit{optimistic}
    if it provides a lower bound on the distance to the closest target: $\forall u. h(u) \leq h^*(u)$.
\end{definition}

While any optimistic $h$ ensures that \A finds optimal
alignments~\cite{dechter_generalized_1985}, the specific choice of $h$
is critical for performance. In particular, decreasing the error $\delta(u) =
h^*(u)-h(u)$ can only improve the performance of
\A~\cite{dechter_generalized_1985}. Thus, a key contribution of ours is
a domain-specific heuristic $h$.

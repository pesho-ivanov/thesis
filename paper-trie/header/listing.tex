% Description: This file defines a nice listing environment.
%
% Usage: You may need to modify the used packages below, depending on what other
% packages you include in your project


%%%%%%%%%
% SETUP %
%%%%%%%%%
% import relevant packages

% import listings package itself
\usepackage{listings}

% requires xcolor package for defining the colors of different textual elements
\usepackage{xcolor}

\usepackage{textcomp}

% typically already loaded, needed for parsing UTF-8 code files
% \usepackage[utf8]{inputenc} % enable UTF-8

% typically already loaded, needed for correct display of font
% \usepackage[T1]{fontenc}

%%%%%%%%
% FONT %
%%%%%%%%
% import a nice typewriter font, improves readability of code
%
% List of options: http://www.tug.dk/FontCatalogue/typewriterfonts.html

% BERA
%
% Examples: http://www.tug.dk/FontCatalogue/beramono/
%
% Documentation: http://texdoc.net/texmf-dist/doc/fonts/bera/bera.txt
%
% "Bera" is a set of three PostScript Type1 font families:
% Bera Serif (a slab-serif Roman), Bera Sans (a "Frutiger
% descendant") and Bera Mono (monospaced/typewriter).
%
% - T1 and textcompanion encoding is selected
%
% - Bera Roman, Sans and Mono are loaded as the three 
%   main text font families (while the math fonts remain 
%   unchanged!)
%
% - the line spacing is enlarged by 5%, i.e.,
%   \linespread{1.05}, with respect to the large x-height of
%   the Bera typefaces;
%
% - the definitions of the TeX and LaTeX logos \TeX and \LaTeX
%   are changed so as to suit BeraSerif.
%
% `scaled=0.8`: scales down the letters to 80% of their "natural" size.

% not using bera as it does not support all font encodings
%\usepackage[scaled=0.8]{beramono}

% FIRA MONO
%
% Examples: http://www.tug.dk/FontCatalogue/firamono/
% 
% Documentation: https://ctan.org/tex-archive/fonts/fira?lang=en
%
% - activate Fira Mono as the monospaced text font
% - Options scaled=<number> or scale=<number> may be used to scale the fonts
% - Font encodings supported are OT1, T1, TS1, LY1 and LGR.
\usepackage[scaled=0.8]{FiraMono}


%%%%%%%%%%
% COLORS %
%%%%%%%%%%
% define colors needed for syntax highlighting

\definecolor{ckeyword}{HTML}{7F0055}
\definecolor{ccomment}{HTML}{3F7F5F}
\definecolor{cstring}{HTML}{2A0099}

%%%%%%%%%%%%%%%%%%%
% DEFINE LANGUAGE %
%%%%%%%%%%%%%%%%%%%
% define a default language with standard, but nice, syntax highlighting
%
% Full documentation available at:
% http://texdoc.net/texmf-dist/doc/latex/listings/listings.pdf

% style for displaying line numbers
\lstdefinestyle{numbers}{
	% display line numbers on the left
	numbers=left,
	%
	% if code is framed, extend the frame to the left, to fit the line numbers
	framexleftmargin=20pt,
	%
	% determines the font and size of the numbers
	numberstyle=\tiny,
	%
	% `auto` lets the package choose the first number: a new listing starts with
	% number one, a named listing continues the most recent same-named listing
	% (named by `name=abc`), and a stand alone file begins with the number
	% corresponding to the first input line.
	firstnumber=auto,
	%
	% Distance between number and listing. Write line numbers closer to code
	numbersep=1em,
	%
	% Extra margin on left, aligns line number with text
	xleftmargin=2em
}

% style for general layouting of listings
\lstdefinestyle{layout}{
	% do not show frame
	frame=none,
	% put line on top and bottom
	%frame=tb,
	%
	% position the caption at the bottom
	captionpos=b,
}

\lstdefinestyle{comment-style}{
	% allow comments with // comment
	morecomment=[l]//,
	%
	% allow comments with /* comment */
	morecomment=[s]{/*}{*/},
	%
	% determines the style of comments
	commentstyle={\color{ccomment}\itshape},
}

\lstdefinestyle{string-style}{
	%
	% allow strings with "string"
	morestring=[b]",%
	%
	% allow strings with 'string'
	morestring=[b]',%
	%
	% determines the style of strings
	stringstyle={\color{cstring}},
	%
	% do not display black spaces in strings as ␣
	showstringspaces=false,%
}

\lstdefinestyle{keyword-style}{
	%
	% determines the style of keywords
	keywordstyle={\ttfamily\bfseries},
	%
	% add to keywords from keyword list
	morekeywords={
		function,
		constructor,
		int,
		bool,
		return,
		returns,
		uint
	},
	%
	% Add more keywords, with a special style
	morekeywords = [2]{},
	keywordstyle = [2]{\text},
	%
	% Introduce @ as a separator of keywords
	% otherkeywords={@},
	% morekeywords = [3]{@},
	% keywordstyle = [3]{},
	%
	% keywords are case sensitive
	sensitive=true,
}

\lstdefinestyle{input-encoding}{
	% determines the input encoding. The usage of this key requires the
	% `inputenc` package; nothing happens if it’s not loaded.
	inputencoding=utf8,
	%
	%
	% Allows extended characters in listings, that means (national) characters
	% of codes  128–255. If you use extended characters, you should load
	% `fontenc` and/or `inputenc`, for example
	extendedchars=true,
	%
	% replace strings in original listings
	%
	% {string to replace}{replacement text}{length of replacement text; number of characters}
	literate=
	{ℝ}{$\reals$}1%
	{→}{$\rightarrow$}1%
	{α}{$\alpha$}1%
	{β}{$\beta$}1%
	{λ}{$\lambda$}1%
	{θ}{$\theta$}1%
	{ϕ}{$\phi$}1%
	{⟦}{$\llbracket$}1%
	{⟧}{$\rrbracket$}1%
}

\lstdefinestyle{escaping}{
	%
	% color everything marked by % in blue: %color this%
	moredelim={**[is][\color{blue}]{\%}{\%}},
	%
	% escapes the user to LATEX: all code between two such characters is
	% interpreted as LATEX code
	%
	% allow adding labels for line numbers
	escapechar=@,
	%
	% Activates special behavior of the dollar sign.  If activated a dollar sign
	% acts as TEX’s text math shift.
	%
	% This key is useful if you want to typeset formulas in listings
	mathescape=true
}

\lstdefinestyle{default-style}{
	%
	% Style selected at the beginning of each listing
	% ttfamily: selects a monospaced (typewriter) font family
	% fontencoding: selects T1 fontencoding (required for correct display in combination with the `beramono` package)
	% footnotesize: controls size of letters
	basicstyle=\fontencoding{T1}\ttfamily\footnotesize,
	%
	style=numbers,
	%
	style=layout,
	%
	style=comment-style,
	%
	style=string-style,
	%
	style=keyword-style,
	%
	style=input-encoding,
	%
	style=escaping,
	%
	%
	% Activates/deactivates automatic line breaking of long lines
	breaklines=false,
	%
	% number of spaces to use for tabs
	tabsize=2,
	upquote=true
}


\lstdefinelanguage{BASIC}{
	% Base language on C++
	language=C++,
	%
	style=default-style
}[keywords,comments,strings]%

% set default language
\lstset{language=BASIC}


\lstdefinelanguage{MyPython}{
	language=Python,
	%
	style=default-style
}

% set default language
%\lstset{language=BASIC}

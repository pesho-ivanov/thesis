\subsection{Match pruning} \label{GLOBALsec:match-pruning}

In order to reduce the number of
states expanded by the \A algorithm, we apply the \emph{multiple-path pruning}
observation: once a shortest path to a vertex has been found, no other path to
this vertex could possibly improve the global shortest
path~\citep{poole2017artificial}. When \A expands the start of a
match we
\emph{prune} this match, so that the heuristic values of preceding states do no longer
benefit from it, thus getting deprioritized by the
\A. We define \emph{pruned} variants of the \sh and the \csh that ignore
pruned matches:

\begin{definition}[Match pruning]
   Let $E$ be the set of expanded states during the \A search, and let $\matches \backslash
   E$ be the set of \emph{unpruned matches}, \ie those matches not starting in an expanded
   vertex. Then the pruning \sh is $\hat h\ssh := h^{\matches\backslash E}\ssh$ and
   the pruning \csh is ${\hat h\scsh := h^{\matches\backslash E}\scsh}$.
\end{definition}

The hat ($\hat h$) denotes the implicit dependency on the progress of the \A. Even
though match pruning breaks the admissibility of our heuristics for some
vertices, we prove that \A is sill guaranteed to find a shortest path:

\begin{restatable}{thm}{thmprunedcorrect} \label{GLOBALthm:pruned-correct}
   \A with match pruning heuristic $\hh\ssh$ or $\hh\scsh$ finds a shortest path.
\end{restatable}

The proof can be found in~\cref{GLOBALsec:partial-admissible}. Since the heuristic may
increase over time, our algorithm ensures that $f$ is up-to-date when expanding a
state by \emph{reordering} nodes with outdated $f$ values (see \cref{GLOBALremark:monotone} in \cref{GLOBALsec:partial-admissible}).

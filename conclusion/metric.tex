\section{On the optimized metric}
%\dictum[Freeman Dyson]{%
%   It is better to be wrong than to be vague.}
%\vskip 1em

An algorithm without an objective function may be wrong because they do not
solve the correct problem. or because they 

In this thesis we were optimizing edit distance. 

Algorithms that guarantee correctness can be wrong only by being given a wrong problem.

An approximate algorithm can be wrong either because it did not fullfill its
mathematical goal. reach its solve the problem because of either optimizing the
\emph{wrong} function.

Algorithm correctness is arguably a useful property which is often not simple to
guarantee. It can undoubtedly improve accuracy, especially in the case of
complex data, but still be wrong from biological point of view. This is because
of  This Nevertheless, since biology is a natural science, its  optimality
guarantees must have an additional impact on the development of the field. It
not only but allows to enjoy being wrong rather than vague. Moreove, often
problems in computational biology are ill-stated andalgorithms that approximate
algorithms that. But many algorithms in computational biology do not even have a
formal statement 

Big jumps motivate affine and concave costs. Local alignment is
\cite{arslan2001new}.
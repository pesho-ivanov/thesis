\section{Limitations}

% +asymptotic analysis
Our presented method has several limitations:
\begin{enumerate}
  \item \emph{Complex regions trigger quadratic search.} Since it is unlikely
        that edit distance in general can be solved in strongly subquadratic
        time, it is inevitable that there are inputs for which our algorithm
        requires quadratic time.  In particular, regions with high error rate,
        long indels, and too many matches~(\cref{GLOBALsec:limitations}) are
        challenging and trigger quadratic exploration.
  \item \emph{High constant in runtime complexity.} Despite the near-linear
        scaling of the number of expanded states~(\cref{GLOBALsec:expanded}),
        \astarpa only outperforms \edlib and \wfa for sufficiently long sequences
        (~\cref{GLOBALfig:scaling-n}) due to the relatively high computational constant
        that the \A search induces.
  \item \emph{Complex parameter tuning.} The performance of our algorithm
        depends heavily on the sequences to be aligned and the corresponding choice of
        parameters (whether to use chaining, the seed length $k$, and whether to use
        inexact matches $\spot$). The parameter tuning (currently
        very simple~(\cref{GLOBALsec:evals-setup}) may require a more comprehensive
        framework when introducing additional optimizations.
  \item \emph{Real data.} The efficiency of the presented algorithm has high
        variability on real data~(\cref{GLOBALsec:evals-comparison-hg}) due to high
        error rates, long indels, and multiple repeats (demonstrated
        in~\cref{GLOBALfig:limitations}). Further optimizations are needed to align
        complex data.
\end{enumerate}
\chapter*{Conclusion} \label{ch:conclusion}
\addcontentsline{toc}{chapter}{\protect\numberline{}{Conclusion}}

We presented a novel \A approach to alignment which is provably optimal and
heuristically fast on related sequences. We incrementally extended the
applicability of our approach to several data dimensions: reference size,
query/sequence length and error rate. Using a trie, semi-global alignment
empirically scaled sublinearly with reference size. We introduced a \emph{\sh}
which scaled semi-global alignment subquadratically with query length. Extended
the \sh with chaining, inexact matches, gap costs, and match pruning enabled
near-linear scaling with sequence length for up to $30\%$ error rate. We proved
that all our heuristics and optimizations are guaranteed to find an alignment
with minimal edit distance.

% paper-prefix
\section{Scaling with reference size}

In \cref{ch:trie} we presented \astarix, a sequence-to-graph optimal alignment
tool based on the \A algorithm with a simple admissible heuristic and enhanced
by multiple algorithmic optimizations. We complemented the refernece with a trie
index to allow a shortest path from the trie root to be found for sublinear time
with the reference size. Our approach allows for general graph references that
may include cycles.

We demonstrated that \astarix scales exponentially better than \dijkstra with
increasing (but small) number of errors in the reads. Moreover, for short reads,
both \astarix and \dijkstra scale better and outperform current state-of-the-art
optimal aligners with increasing genome graph size.

% paper-seeds
\section{Scaling with sequence length}

In \cref{ch:seed} we upgraded our sequence-to-graph aligner \astarix with a
novel admissible \emph{\sh} which guides the search based on seed matches
between the query and the reference. In order to compute the \sh efficiently for
each explored node, for each query, we split it to seeds, precompute the seed
matches in the reference, and place \emph{crumbs} on the nodes before the match
in order to mark an upcoming match. In addition to the sublinear scaling with
reference size, the \sh alloed subquadratic scaling with the query length, which
enabled the optimal alignment of long reads. 

% paper-global
\section{Scaling with error rate}

In \cref{ch:global} we introduced \astarpa which solves global alignment. To
this end, we extended the existing \sh for \A with inexact matches, match
chaining, gap costs, match pruning, and diagonal-transition.

On random sequences with $d{\leq}8\%$ uniform divergence, the runtime of
\astarpa scales near-linearly to very long sequences ($n{\leq}10^7\bp$) and
outperforms other exact aligners. On long ONT reads of human data with $d{\leq}10\%$,
\astarpa is over $2\times$ faster than other aligners, and $1.4\times$ faster
when genetic variation is also present.
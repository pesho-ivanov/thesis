\section{Discussion}

\subsection{On the developments for sequence alignment}
A question I was asking myself and have been asked multiple times is why this
approach has not been used decades earlier.

Most of the techniques this thesis builds upon have been known for many decades
and have also been heavily motivated by applications in molecular biology. The
global alignment problem exists since
\citeyear{vintsyuk1968speech}~\cite{vintsyuk1968speech,needleman1970general}.

The problem of approximate/fuzzy string search (semi-global alignment of one
query) has been efficiently approached a decade later, in
\citeyear{sellers1980theory}~\cite{sellers1980theory,smith1981identification}.
more recent and additionally to the issues with handling errors, a new dimension
of complexity appears by the need to locate the position in the reference. The
problem of searching for multiple sequencing has been approached
since\citeyear{pearson1988improved}~\cite{pearson1988improved} and became
central to read mapping of high-throughput sequencing. The specifics with
semi-global alignment requires a trie-like index which is well-used in the
fiels, and known since
\citeyear{thue1912gegenseitige}~\cite{thue1912gegenseitige} and used in
informatics since~\citeyear{de1959file}~\cite{de1959file}.

An important factor for the development of the sequencing algorithms is the
technological advancement~\cite{alser2021technology}. One possible reason is the
combination of massive new data and advancement in computation hardware allowing
for various kinds of parallelization. Thus the advancements with bit-parallel
algorithms (CITE Myears, Mikko).

\subsection{Reconceptualizing seeds for optimal alignment}

All production read aligners rely on the \emph{seed-extend} paradigm (and its
\emph{seed-chain-extend} variants for long reads). This paradigm requires
similar short \emph{seed} patches to be found between the sequences (\eg by
hashed kmers, minimiziers, maximum exact matching, etc.), and then to
\emph{extend} the alignment of the whole query around these \emph{seeded}
similar patches. This is a very intuitiv approach if the goal is to find a
\emph{good alignment}.

If we instead seek not good but provably \emph{best} alignments, we are required
to at least implicitly refute all the exponentially-many competing alignments.

Instead, to find optimal alignments, we do not need to choose the seeds to be
long and similar with the reference buare not required to be similar.

\begin{observation}[Seeds without matches]
    To efficiently find an optimal alignment using \A with the seed heuristic,
    seeds are not required to match (even on the resulting alignment).
\end{observation}

Nevertheless, each seed can penalize potential alignment by not more than its
\emph{potential} (\ie the number of plus $1$, for the case of exact matching
with unit costs). Any additional errors will require more states to be expanded.

This is an interesting observation was made by Ragnar while playing with the
seed heursitic. It looks Indeed, of finding a good alignment but to prove that all
alternative alignments are no better, the seed heuristic for \A search does not
really need matches to be efficient.

This novel usage of seeds carrie different problems and different possibilities.

\subsection{On the feasibility of optimal alignment}
Many scholars consider read mapping to be infeasible for optimal algorithms,
especially when reads are long: all production read mappers following the
approximate seed-extend paradigm~\cite[including 107
aligners]{alser2021technology}.

\subsection{On the optimization metric}
\dictum[Freeman Dyson]{%
   It is better to be wrong than to be vague.}
\vskip 1em

\subsection{On pangenomics}
%has been solved in linear time in \citeyear{morris1970linear}~\cite{morris1970linear}.

\subsection{On scaling}
% asymptotics

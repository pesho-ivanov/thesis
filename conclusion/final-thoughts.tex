\section{Final thoughts}

The potential adoption of the \A approach to sequencing (incl. the usage of
\astarix aligner and the A*PA global aligner) may hugely alleviate the
decades-long conflict between alignment optimality and computational costs. This
basic task is an extremely common element in the upstream pipelines and other
high-level tools like assemblers. Possibly, many alignment tools can be
outcompeted (similar to the Parasail aligner) and become outdated. The general
impact of guaranteed optimality is increased trust in the tools by
bioinformaticians, ability to explain any alignment, reduced number of
misalignments, applicability to both linear and graph references. A risky
high-stake problem we plan to explore is the multiple sequence alignment (MSA)
since it has many resemblances to pairwise alignment but suffers from the
combinatorial explosion with the number of sequences. Even with relaxed
optimality-guarantees, a novel approach to this general problem may have a
direct impact on comparative genomics, including protein folding. Our experience
with presenting the \A approach convinced us that it is not a common knowledge
in the computation biology community, even though most conference attendants
with technical background are familiar with the name and the general usage. Our
conviction is that \A is a powerful instrument that has fallen out of focus in
our community but that may prove useful for other algorithmic biology problems
as well. Potentially, it could become a standard topic in the education in
computational biology.

% paper:global
Our graph-based approach to alignment differs considerably from dynamic
programming approaches, mainly because of the ability to use information from the
entire sequences. This additional information enables radically more focused
path-finding at the cost of more complex algorithms.
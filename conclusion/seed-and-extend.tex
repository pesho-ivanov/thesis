%\section{Reconceptualizing seeds for optimal alignment}
\section{\emph{Seed-and-extend} need not apply}

As we saw in \cref{ch:trie,ch:seed}, all optimal read mappers compute the whole
dynamic programming table, thus reaching the prohibitively slow quadratic
runtime. On the other side, all current production aligners rely on the
\emph{seed-extend} paradigm (and its \emph{seed-chain-extend} variants for long
reads).

This paradigm requires similar short \emph{seed} patches to be found
between the sequences (\eg by hashed kmers, minimiziers, maximum exact matching,
etc.), and then to \emph{extend} the alignment of the whole query around these
\emph{seeded} similar patches. This is a very intuitiv approach if the goal is
to find a \emph{good alignment}.

If we instead seek not good but provably \emph{best} alignments, we are required
to at least implicitly refute all the exponentially-many competing alignments.

Instead, to find optimal alignments, we do not need to choose the seeds to be
long and similar with the reference buare not required to be similar.

\begin{observation}[Seeds without matches]
    To efficiently find an optimal alignment using \A with the seed heuristic,
    seeds are not required to match (even on the resulting alignment).
\end{observation}

Nevertheless, each seed can penalize potential alignment by not more than its
\emph{potential} (\ie the number of plus $1$, for the case of exact matching
with unit costs). Any additional errors will require more states to be expanded.

This is an interesting observation was made by Ragnar while playing with the
seed heursitic. It looks Indeed, of finding a good alignment but to prove that all
alternative alignments are no better, the seed heuristic for \A search does not
really need matches to be efficient.

This novel usage of seeds carrie different problems and different possibilities.
\section*{Future work}
\addcontentsline{toc}{section}{\protect\numberline{}{Future work}}

% from the grand proposal
We initiated a new direction of sequence alignment based on the informed
shortest path algorithms \A that we have shown to be both provably optimal,
practically scalable and more performant than existing aligners in certain
cases. A general objective is to develop the \A for sequence alignment from
prototypical to practical, which includes the development of algorithms, formal
proofs and software development. Our specific aim is to use the already
constructed base for the development of \astarix as an industrial-scale aligner
handling human-scale graph references, and long and noisy reads, while being
competitive in runtime and memory even to suboptimal aligners. To fulfill this
aim, work in several directions will be needed: 1) rigorous development of the
seed heuristic theory, 2) theoretical analysis of the expected runtime, 3)
implementation optimized for runtime, memory, parallelization and use cases. We
foresee a multitude of extensions and optimizations that may lead to efficient
sequence aligning for production usage. Moreover, we suppose that due to
incorporating fuller information for its informed search, \A could become the
default framework for sequence alignment, useful for computational biology,
bioinformatics and general informatics.

\begin{enumerate}
    \item \emph{Performance.} The practical performance of our \A approach could
        be improved using multiple existing ideas from the alignment domain:
        diagonal transition method, variable seed lengths, overlapping seeds,
        combining heuristics with different seed lengths, gap costs between
        matches in a chain~\citep{ukkonen1985algorithms,wilbur1984context}, more
        aggressive pruning, and better parameter tuning. More efficient
        implementations may be possible by using computational
        domains~\citep{spouge1989speeding},
        bit-parallelization~\citep{myers1999fast}, and
        SIMD~\citep{marco2021fast}, lowering the \A constant. The efficiency of
        the presented algorithm has high variability on real
        data~(\cref{GLOBALsec:evals-comparison-hg}) due to high error rates,
        long indels, and multiple repeats. Further optimizations are needed to
        align complex data. More accurate heuristics lead to better \A
        performance~\cite{pearl_discovery_1983}, so machine learning may be
        useful for tuning seed heuristic parameters.
    \item \emph{Generalizations.} Our method can be generalized to more
        expressive cost models (non-unit costs, affine costs) and different alignment
        types (semi-global, ends-free, and possibly local alignment).
    \item \emph{Relaxations.} Abandoning the optimality guarantee
        enables various performance optimizations. Another relaxation
        of our algorithm would be to validate the optimality of a given alignment more
        efficiently than finding an optimal alignment from scratch.
    \item \emph{Analysis.} The near-linear scaling behavior requires a thorough
        theoretical analysis~\citep{medvedev2022limitations}. The fundamental
        question that remains to be answered is: \emph{What sequences and what
        errors can be tolerated while still scaling near-linearly with the
        sequence length?} We expect both theoretical and practical contributions
        to this question.
    \item \emph{Applicability.} More dimensions of data complexity could be
        explored: scaling to larger alphabet size, general language, compressed
        low-entropy text, and higher reference complexity (repeats, multiple best
        alignments, bubbles, cycles).
    \item \emph{Optimality verification} Verify the optimality of a given alignment.
\end{enumerate}

% paper-seed
%The memory usage is currently limiting the application of \astarix for bigger
%references due to the size of the trie index. A remaining challenge is designing
%a heuristic function able to handle not only long but also noisier reads, such
%as the uncorrected PacBio reads that may reach 20\% of mistakes. Possible
%improvements of the seed heuristic may include inexact matching of seeds,
%careful choice of seed positions, and accounting for the seeds' order.

\subsection*{Planned work packages}

\paragraph{Practical sequence-to-graph aligner}
Develop the \astarix sequence-to-graph aligner from prototype to production As
our previous work demonstrates, scaling optimal \A alignment to long sequences
and high error rates relies on various non-trivial algorithms, data structures
and optimizations. For \astarix to become of practical importance, it must be
comparable to not only optimal aligners but to approximate ones as well -- this
includes tolerating more errors, lowering the runtime and memory consumption,
and supporting features. We observe two qualitatively different modes in which
the \A explores the state space: linear and quadratic. The mode of exploration
is mainly determined by the ability of the seed heuristic to ``compensate'' for
the type and amount of errors. When the heuristic cannot cope with the errors,
\A has to continue exploring without the aid of the heuristic (similar to
Dijkstra): for long indels, the exploration becomes locally quadratic (around
the indel), and for too high error rate it becomes globally-quadratic (because
of the accumulating insufficiency of the seeds). By tolerating errors, we mean
the ability of the exploration to stay linear or near-linear. Currently,
\astarix supports 2-4\% errors on short reads and only 0.3\% on HiFi reads,
whereas our recent developments on global alignment demonstrated the potential
to tolerate error rates up to 25\% using match pruning, inexact matching, and
seed chaining. Developing these features in the semi-global setting is not
trivial because of the added complexity of crumbs and the trie index but we do
not envision any fundamental reasons against adopting them. A natural approach
to tolerating long indels, well known in the bioinformatics community, is to
generalize the edit distance metric to affine costs to match the error model
more closely. The main memory bottleneck is currently the trie index (which
would take >90\% of the memory given a standard compressed genome graph
representation) so our primary priority is to algorithmically reduce the number
of nodes in the trie and then implement it efficiently. Currently, it is used
with two separate functions: 1) to find all seed matches, and 2) to allow the
search \A to explore the whole reference only implicitly by abstracting together
the reference location reachable with the same prefix. The first function is
efficiently solvable by using the positional Burrows-Wheeler transform (gBWT) to
find exact seed matches and, in the case of inexact matches, to pre-generate all
kmers at distance 1 from the seeds and invoke gBWT separately for each. The
second function requires a trie but, if possible, not over the whole graph, only
pointing to those reference locations that hold at least one crumb: otherwise
the \A search will surely be quadratic in which case the fallback options are to
either refuse to align the read (which is still optimal but incomplete), or find
an alignment by explicitly considering all starting locations (which could be
done faster using another tool). We foresee that building a separate “slim” trie
per query will bring orders of magnitude of shrinkage of the trie, making it
proportional only to the query length, and not the reference size. Additionally,
a compressed implementation of the trie and the reference graph has the
potential to lower the current memory usage by an additional order of magnitude.
For a small error rate (0.3\%) and long reads (30 kbp), the bottleneck for the
runtime is the amount of crumbs. To effectively skip the work of placing most of
the crumbs, the placement of different seeds could be done together in one
backward traversal == this is expected to eliminate most of the work for placing
crumbs on the optimal path.  Another approach is to also ``jump over unitigs''
when placing crumbs in the backward traversal, and ``carry'' crumbs in the
forwards \A search to compensate for the skipped ones. A major optimization in
the case of a high error rate is to adapt the match pruning that we have
developed for the global alignment case. Features. By discussing \astarix on
conferences, we have figured out that extension alignment (very similar in the
approach to global alignment) is of great interest for its reusability in the
seed-and-extend paradigm that other aligners follow. The dataset for evaluation
ideally includes bigger graphs, noisy long reads (e.g. ONP, PacBio), and reads
with bigger indels. The comparison should be not only to optimal aligners but
also to approximate ones. The preliminary experiments demonstrate sublinear
scaling with the reference size but since this is an empirical result, it may
also deteriorate for bigger graphs. Some of the optimizations may not provide
the full expected effect or the bottleneck they would resolve may not yet be
current. Usage of \astarix by bioinformaticians is crucial for the adoption of
\astarix by the broader community.

\paragraph{Theoretical guarantees}
Two equal sequences can be aligned for linear time, whereas two general
sequences cannot be aligned strongly faster than
quadratically~\cite{backurs2015edit}. There is a knowledge gap for the
computational limitations of related sequences~\citep{medvedev2022limitations}
which may benefit various statistical approaches different from the worst-case
asymptotic analysis and the empirical
investigation~\citep{medvedev2022theoretical}. Our strong empirical results on
scaling with increasing sequence length and error rates, together with our
intuition from the internals of our \A algorithms, hint towards much stricter
theoretical bounds on the expected computational costs. Such theoretical results
may also be of general interest outside of the computational biology community.
Our approach to proving theoretical bounds is likely to be tightly related to
the term potential (of a seed heuristic function) which is a measure of the
ability of a specific seed heuristic instantiation (i.e. for fixed seed length,
etc.) to punish for future errors before exploring them. The analysis should
also account for the cost of computing the heuristic (including amortized
precomputation and querying) since infinite resources on it can result in an
oracle but also destroy the total performance. The risks of such analyses
include the potential inability to account for data and error models (random
reference sequence and uniform error model) that are realistic enough. Our
intuition on the scaling of the explored states includes several parameters: the
seed heuristic potential and the number of errors. Informally speaking, for each
error that is not compensated by the potential, the \A search will explore
another layer of states that are 1 more distant from the diagonal. This is also
theoretically and practically true for the special case of \A, Dijkstra, which
has a potential 0 and thus reaches O(ns) where s is the overall number of
errors. Our empirical results demonstrate that a near-linear runtime scaling is
practically possible up to very long sequences (100 Mbp) for error rates of
10\%. In addition to the computational costs for aligning a sequence to another
sequence, alignment has the burden of localization of the best alignment. Our
intuition behind his localization cost is related to to usage of \A on the trie:
for each error that the \A potential does not compensate for, the exploration
would get one lever deeper in the trie, thus exploring the expected 4x more
states. This is an interesting observation since it hints at an exponential
scaling with an error rate, which is not asymptotically possible because of the
quadratic worst case but may be the case near the best case. In such a case, a
type of sigmoid would possibly describe the number of explored states as a
function of the number of errors.

\paragraph{Alignment on a probabilistic graph references}
Probabilistic interpretations arise in different places among the alignment
problems. Sequencing machines report a standardized score per nucleotide that
estimates the probability of that nucleotide being wrong. This information is
currently largely ignored and very valuable for highly erroneous sequencing
technologies. Extending the edit distance metric to account for phred values
should be directly applicable and may even benefit the performance of the \A
algorithm due to the better correspondence between the scores and the error
model. A topic for current discussions in the pangenomics field is whether to
add more genomes to the reference graph or to keep the reference unbiased (by
unbalanced number of entries for different variations, or by adding genomes from
other populations). One reason for such discussions is the memory and runtime
limitations, but the accuracy of alignment is not less important. A weighting of
the graph may be useful as a soft tradeoff. Hidden Markov model (HMM) queries
have been useful in the context of microbial sequencing. Possibly, \A can be
extended to handle HMM queries instead of linear sequences. The current
solutions are prohibitively slow and \astarix is expected to greatly accelerate
the optimal alignment after being generalized to probabilistic queries. The
\textit{mapq} value~\citep{li2008mapping} may be calculated more precisely if
accounting for the probability estimates.

\paragraph{Apply \A to local alignment}
Explore the applicability of \A to local alignment Our previous work has been
focused on semi-global alignment (alignment) and global alignment (and
extension). The issue with extending the \A approach to local alignment is
two-fold: The search for the best local alignment is reset when the current
score becomes negative – in other words, negative scores are crucial for local
alignment but not directly applicable to Dijkstra and \A. A potential solution
may follow the approach of Jonson's algorithm~\citep{johnson1977efficient} which
reweights the graph to only non-negative edges as long as there are no negative
cycles. For semi-global and global alignments, the whole query is aligned, which
is a useful property for the admissibility of the seed heuristic, since each
seed is guaranteed to be aligned eventually. In the case of local alignment,
some seeds may not be aligned. Possible workarounds have to be explored. A
solution to local alignment may as well apply to the split alignment mode when
``teleportations'' are allowed. Another application of local alignment is to
align reads without trimming their ends from adapters first.

\paragraph{Machine learning for tuning the seed heuristic}
Explore the applications of machine learning to generate better admissible
heuristics Tuning parameters is generally hard and it has been problematic for
the seed heuristic as well. One potential place for runtime improvement is to
carefully choose the seed heuristic parameters: seed length and allowed number
of errors per match. Usually, machine learning methods do not provide correctness
guarantees about the produced results. Nevertheless, since the parametrization
of the seed heuristics influences only its performance and not its admissibility
and optimality of the results. Additionally, the seed heuristic parameters could
be chosen or modified dynamically, during the \A search. Reinforcement learning
may provide a reasonable method for tuning the parameters in real time.

\paragraph{K-best paths}
Explore the applications of finding K-best alignments which is trivially
achievable in \A framework. A natural extension of the shortest path algorithms
is to find not one shortest path, but K-shortest paths. One application of the
K-best alignments are the computation of the \textit{mapq} mapping
score~\citep{li2008mapping} which is based on the alignment costs of the K-best
alignments and provides an estimation of the certainty of the best alignment.
Potentially, there may be other applications.

\paragraph{Jointly align a set of reads}
Explore the joint alignment of a set of sequences to a graph. The reads to be
mapped to a reference genome are highly correlated since they usually come from
the same biological sample. On the other hand, they can be biased according to
the reference genome, due to biological variation. Ideally, a clever alignment
algorithm would reconstruct this biological variation and align the reads to the
corrected reference. This problem is ill-posed and needs careful consideration
to develop a theory/metric. Another complication is computational: aligning even
a single query to a graph that is allowed to change may require exponential time
with the query length.

\paragraph{Generalize the seed heuristic to multiple sequences}
Existing approaches to MSA using \A consider pairwise edit distances, and do not
considerably accelerate the alignment. A simultaneous seed match (anchor) within
several sequences may provide much bigger penalties for suboptimal paths.

\paragraph{Relax the \A optimality guarantees for performance gain}
Various relaxations of the heuristic admissibility, variants of \A, windowing
the search (dropping expanded states that are too far behind the exploration
head), have the potential to greatly increase the performance at the price of
alignment optimality. This may be useful for the performance and thus the
practical adoption of the algorithm but we prefer first gaining the performance
from the obvious promising optimality-preserving optimizations.

%\paragraph{Generalizing the distance metric}
%\dictum[Freeman Dyson]{%
%   It is better to be wrong than to be vague.}
%\vskip 1em
%
%In this thesis we were optimizing edit distance. Big gaps (indels) in biological
%sequences motivate affine and concave costs. Local alignment is
%\cite{arslan2001new}.

% theory An alternative theoretical analysis would consider the average case
%(expected) scaling of our algorithms under a data model. An imaginable result
%would look like a connection between the sequence lengths, error rate, and the
%algorithm steps (mostly dependent on the number of seed matches and the number
%of expanded states). To construct such a connection, a heuristic function will
%have to be chosen among a class of seed heuristics (with a certain seed length,
%allowed number of errors in seed matches, etc.).
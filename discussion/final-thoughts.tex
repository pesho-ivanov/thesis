\section*{Final thoughts}
\addcontentsline{toc}{section}{\protect\numberline{}{Final thoughts}}

It may be surprising that a novel approach to alignment appears 60 years after
the problem was first efficiently solved. Nevertheless, 

Here we speculate about the possible
reasons for it: focus on new technology and data, not believing that optimal
solutions could be efficient.

% paper:global
Our graph-based approach to alignment differs considerably from dynamic
programming approaches, mainly because of the ability to use information from the
entire sequences. This additional information enables radically more focused
path-finding at the cost of more complex algorithms.

%\section{Reconceptualizing seeds for optimal alignment}
\subsection*{Beyond \emph{seed-(chain)-extend} paradigm}
%\addcontentsline{toc}{subsection}{\protect\numberline{}{Beyond \emph{seed-(chain)-extend} paradigm}}

As we saw in \cref{ch:trie,ch:seed}, all optimal read aligners compute the whole
dynamic programming table, thus reaching the prohibitively slow quadratic
runtime. On the other side, all current production aligners rely on the
\emph{seed-extend} paradigm (and its \emph{seed-chain-extend} variants for long
reads).
%
This paradigm requires similar short \emph{seed} patches to be found
between the sequences (by hashed kmers, minimiziers, maximum exact matching,
etc.), and then to \emph{extend} the alignment of the whole query around these
\emph{seeded} similar patches. This is a very intuitiv approach if the goal is
to find a \emph{good alignment}.
%
If we instead seek not good but provably \emph{best} alignments, we are required
to at least implicitly refute all the exponentially-many competing alignments.
%
Instead, to find optimal alignments, we do not need to choose the seeds to be
long and similar with the reference buare not required to be similar.
%
\begin{observation}[Seeds without matches]
    To efficiently find an optimal alignment using \A with the seed heuristic,
    seeds are not required to match (even on the resulting alignment).
\end{observation}
%
Nevertheless, each seed can penalize potential alignment by not more than its
\emph{potential} (\ie the number of plus $1$, for the case of exact matching
with unit costs). Any additional errors will require more states to be expanded.
%
This is an interesting observation was made by Ragnar while playing with the
seed heursitic. It looks Indeed, of finding a good alignment but to prove that all
alternative alignments are no better, the seed heuristic for \A search does not
really need matches to be efficient.
%
This novel usage of seeds carrie different problems and different possibilities.

Most of the techniques this thesis builds upon have been known for many decades
and have also been heavily motivated by applications in molecular biology. 

In contrast, here we demonstrate that seeds can benefit optimal alignment as
well.

\subsection*{Pedagogy}
The potential adoption of the \A approach to sequencing (incl. the usage of
\astarix aligner and the A*PA global aligner) may hugely alleviate the
decades-long conflict between alignment optimality and computational costs. This
basic task is an extremely common element in the upstream pipelines and other
high-level tools like assemblers. Possibly, many alignment tools can be
outcompeted (similar to the Parasail aligner) and become outdated. The general
impact of guaranteed optimality is increased trust in the tools by
bioinformaticians, ability to explain any alignment, reduced number of
misalignments, applicability to both linear and graph references. A risky
high-stake problem we plan to explore is the multiple sequence alignment (MSA)
since it has many resemblances to pairwise alignment but suffers from the
combinatorial explosion with the number of sequences. Even with relaxed
optimality-guarantees, a novel approach to this general problem may have a
direct impact on comparative genomics, including protein folding. Our experience
with presenting the \A approach convinced us that it is not a common knowledge
in the computation biology community, even though most conference attendants
with technical background are familiar with the name and the general usage. Our
conviction is that \A is a powerful instrument that has fallen out of focus in
our community but that may prove useful for other algorithmic biology problems
as well. Potentially, it could become a standard topic in the education in
computational biology.
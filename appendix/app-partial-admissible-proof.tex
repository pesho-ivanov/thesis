\subsection{Partially admissible heuristic} \label{GLOBALsec:partial-admissible}

Here, we generalize the concept of an \emph{admissible} heuristic to that of a
\emph{partially admissible heuristic} that may change value as the \A
progresses, and even become inadmissible in some vertices.
In \cref{GLOBALthm:partial-admissible} we show that \A still finds a shortest path
when used with a partial admissible heuristic.

\begin{definition}[Fixed vertex]
  A \emph{fixed} vertex is a closed vertex $u$ to which \A has found a shortest path,
  that is, $g(u) = \g(u)$.
\end{definition}
A fixed vertex is never opened, and hence remains fixed.

\begin{definition}[Partial admissible]
  A heuristic $\hh$ is \emph{partially admissible} when there exists a shortest
  path $\pi^*$ from  $v_s$ to $v_t$ for which at any point during the \A and for
  any non-fixed vertex $u\in \pi^*$, the heuristic in $u$ is a lower bound on
  the remaining distance $\h(u)$: $\hh(u) \leq \h(u)$.
\end{definition}

We follow the structure and notation of \citet{hart1968formal} to prove that \A
finds a shortest path when used with a partially admissible heuristic.
\begin{lem}[Generalization of \citet{hart1968formal}, Lemma 1]\label{GLOBALlemma1}
  Assuming $v_t$ is not fixed,
  there exists an open vertex $n'$ on $\Path$ with $g(n') = \g(n')$ such that
  all vertices in $\pi^*$ following $n'$ are not fixed.
\end{lem}
\begin{proof}
  Let $n^*$ be the last fixed vertex in $\Path$, and let
  $n'$ be its successor on $\Path$. We have $g(n') = \g(n')$ since $\Path$ is a
  shortest path and $n^*$ was expanded, and $n'$ is open by our choice of $n^*$.
\end{proof}

\begin{cor}[Generalization of \citet{hart1968formal}, Corollary to Lemma 1]\label{GLOBALcor:cor}
  Suppose that $\hh$ is partially admissible, and suppose \A has not terminated.
  Then $n'$ as in \cref{GLOBALlemma1} has $f(n') \leq \g(v_t)$.
\end{cor}
\begin{proof}
  By definition of $f$ we have $f(n') = g(n') + \hh(n')$. Since $n'\in \Path$ and
  $\Path$ does not contain any fixed vertices after $n'$, the partial admissibility of $\hh$
  implies $\hh(n') \leq \h(n')$.
  Thus,
  $f(n') = g(n') + \hh(n') = \g(n') + \hh(n') \leq \g(n') + \h(n') = \g(v_t)$.
\end{proof}
\begin{thm}\label{GLOBALthm:partial-admissible}
  \A with a partially admissible heuristic finds a shortest path.
\end{thm}
\begin{proof}
  The proof of Theorem 1 in \citet{hart1968formal} applies, with the remark that we use the
  specific path $\Path$ and the specific vertex $n'$ from \cref{GLOBALcor:cor}, instead
  of an arbitrary shortest path $P$.
\end{proof}

\begin{remark}[Monotone heuristic]\label{GLOBALremark:monotone}
  We call $\hh(u)$ \emph{monotone} when it does not decrease as the \A
  progresses. To find the vertex $u$ with minimal $f(u)$, we add
  step \cref{GLOBALastar:pruning-check} to the \A algorithm in \cref{GLOBALsec:astar} to check
  whether the value of $f$ in the priority queue is still up to date.
  If $f=f(u)$, we know that for each vertex $(v, g_v, f_v)$ in the queue we have
  $f(u)\leq f_v\leq f(v)$ for
  all other open vertices $v$ in the queue with stored value $f_v$, ensuring that
  $u$ indeed has the minimal value of $f$ value of all open vertices.
\end{remark}

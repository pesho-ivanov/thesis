\paragraph{Modified \A algorithm for pruning} \label{GLOBALsec:astar}

%\renewcommand\theenumi{\algletter\arabic{enumi}}
%\renewcommand\labelenumi{{\rmfamily \algletter\arabic{enumi}.}}
%\setlength{\leftmargini}{2em}

We give our variant of the \A algorithm \citep{hart1968formal} that adds steps
\cref{GLOBALastar:pruning-check} and \cref{GLOBALastar:pruning} (marked as \textbf{bold}) to handle the
pruning of matches. All computed values of $g$ are stored in a map, and all
states in the front are stored in a priority queue of tuples $(v, g(v), f(v))$
ordered by increasing $f$.

\newcommand{\bolditem}{\refstepcounter{enumi}\item[\textbf{\theenumi.}]}
\newcommand{\boldsubitem}{\refstepcounter{enumii}\item[\textbf{\theenumii.}]}
\renewcommand{\algletter}{A}
\begin{enumerate}
  \item Set $g(v_s) = 0$ in the map and initialize the priority queue with
        $(v_s, g(v_s){=}0, f(v_s))$.
  \item Pop the tuple $(u, g, f)$ with minimal $f$.
  \item If $g > g(u)$, \ie the value of $g(u)$ in the map has decreased since pushing the
        tuple, go to step 2. This is impossible for a consistent heuristic.
  \item \label{GLOBALastar:pruning-check} \textbf{If $f \neq f(u)$, \emph{reorder} $u$: push $(u, g(u), f(u))$ and
        go to step 2. This can only happen when $f$ changes value, \ie after
        pruning.}
  \item If $u=v_t$, terminate and report a best path.
  \item Otherwise, \emph{expand} $u$:
  %      \vspace{-7pt}
  \begin{enumerate}
    \item \label{GLOBALastar:pruning} \textbf{Prune matches starting at $u$.}
    \item \label{GLOBALastar:open}
        For each successor $v$ of $u$, \emph{open} $v$ when either $g(v)$
        has not yet been computed or when $g(u) + d(u,v) < g(v)$.
        In this case, update the value of $g(v)$ in the map and insert $(v, g(v), f(v))$ in
        the priority queue. Go to step 2. This makes $u$ the \emph{parent} of $v$.
  \end{enumerate}
\end{enumerate}

Note that the heuristic is evaluated in steps \cref{GLOBALastar:pruning-check} and
\cref{GLOBALastar:open} for the computation of $f(u)$ and $f(v)$ respectively.

A node is called \emph{closed} after it has been expanded for the
first time, and called \emph{open} when it is present in the priority queue.

\subsection{General \csh}\label{sec:heuristic}

We introduce three heuristics for \A that estimate the edit distance between a
pair of suffixes. Each heuristic is an instance of a \emph{general \csh}. After
splitting the first sequence into seeds $\seeds$, and finding all matches
$\matches$ in the second sequence, any shortest path to the target can be
partitioned into a \emph{chain} of matches and connections between the matches.
Thus, the cost of a path is the sum of match costs~$\matchcost$ and
\emph{chaining costs}~$\gamma$. Our simplest \sh ignores the position in $B$
where seeds match and counts the number of seeds that were not matched
($\gamma{=}\seedcost$). To efficiently handle more errors, we allow seeds to be
matched inexactly, require the matches in a path to be ordered~(\CSH), and
include the gap-cost in the chaining cost ${\gamma {=} \max(\gapcost,
\seedcost)}$ to penalize indels between matches~(\GCH).

\paragraph{Inexact matches}
We generalize the notion of exact matches to \emph{inexact matches}. We fix a
threshold cost $r$ ($0{<}r{\leq}k$) called the \emph{seed potential} and define
the set of \emph{matches} $\matches_s$ as all alignments $m$ of seed $s$ with
\emph{match cost} ${\matchcost(m) < r}$. The inequality is strict so that
$\matches_s=\emptyset$ implies that aligning the seed will incur cost at least
$r$. Let $\matches = \bigcup_s \matches_s$ denote the set of all matches. With
$\spot{=}1$ we allow only \emph{exact} matches, while with $\spot{=}2$ we allow
both exact and \emph{inexact} matches with one edit. We do not consider higher
$\spot$ in this paper. For notational convenience, we define $m_\omega \not\in
\matches$ to be a match from $v_t$ to $v_t$ of cost~$0$.

\paragraph{Potential of a heuristic}
We call the maximal value the heuristic can take in a state its \emph{potential}
$\Pot$. The potential of our heuristics in state $\st ij$ is the sum of seed
potentials $r$ over all seeds after $i$: ${\Pot\st ij := r\cdot
\abs{\seeds_{\geq i}}}$.

\paragraph{Chaining matches}
Each heuristic depends on a \emph{partial order} on states that limits
how matches can be \emph{chained}. We write $u\preceq_p v$
for the partial order implied by a function $p$: ${p(u) \preceq p(v)}$. A
\emph{$\preceq_p$-chain} is a sequence of matches ${m_1 \preceq_p \dots
\preceq_p m_l}$ that precede each other: ${\matchend(m_i) \preceq_p
\matchstart(m_{i+1})}$ for $1\leq i < l$. To chain matches according
only to their $i$-coordinate, \SH is defined using $\preceq_i$-chains, while \CSH
and \GCH are defined using $\preceq$ that compares both $i$ and $j$.

\paragraph{Chaining cost}
The \emph{chaining cost} $\gamma$ is a lower bound on the path cost between two
consecutive matches: from the end state $u$ of a match, to the start $v$ of the
next match.

For \SH and \CSH, the \emph{seed cost} is $r$ for each non-matched seed
${\seedcost(u,v) {:=} r{\cdot} \abs{\substr\seeds uv}}$. When $u\preceq_i v$ and $v$
is not in the interior of a seed, then ${\seedcost(u, v) = \Pot(u) - \Pot(v)}$.

For~\GCH, we also include the gap cost $\gapcost(\st ij,
\st{i'}{j'}):=\abs{(i'{-}i) {-} (j'{-}j)}$ which is the minimal number of indels
needed to correct for the difference in length between the substrings $\substr
Ai{i'}$ and $\substr Bj{j'}$ between two consecutive
matches~(\cref{sec:preliminaries}). Combining the seed cost and the chaining
cost, we obtain the gap-seed cost ${\gscost=\max(\seedcost, \gapcost)}$, which
is capable of penalizing long indels and we use for \GCH.
Note that $\gamma{=}\seedcost{+}\gapcost$ would not give an admissible heuristic
since indels could be counted twice, in both $\seedcost$ and $\gapcost$.

For conciseness, we also define the $\gamma$, $\seedcost$, $\gapcost$, and $\gscost$ between
matches ${\gamma(m, m') := \gamma(\matchend(m), \matchstart(m'))}$, from
state to match ${\gamma(u, m') {:=} \gamma(u, \matchstart(m'))}$, and from
match to state ${\gamma(m, u) := \gamma(\matchend(m),u)}$.

\paragraph{General \csh}
We now define the general \csh that we use to instantiate \SH, \CSH and \GCH.

\begin{definition}[General \csh]\label{dfn:h}
  Given a set of matches $\matches$, partial order $\preceq_p$, and chaining cost~$\gamma$,
  the \emph{general \csh} $h^\matches_{p,\!\gamma}\!(u)$
  is the minimal sum of match costs and chaining costs over all $\preceq_p$-chains
  (indexing extends to $m_0 := u$ and $m_{l+1}:=m_\omega$):
  \begin{equation*}
    h^\matches_{p,\!\gamma}\!(u) :=
        \min_{
          \substack{
            u \preceq_p m_1 \preceq_p \dots \preceq_p m_l \preceq_p v_t\\
            m_i \in \matches}
        }
        \sum_{0\leq i \leq l}
          \big[ \gamma(m_i, m_{i+1}) + \matchcost(m_{i+1}) \big].
    \label{eq:general_sh}
  \end{equation*}
\end{definition}

\begin{table}[H]
	\centering
	\sffamily
    \begin{tabular}{@{}rlcl@{}}
    \toprule
    \multicolumn{2}{l}{\textbf{Heuristic}} & \textbf{Order} &  \textbf{Chaining cost $\gamma$} \\
      \midrule
    $\hsh(u)$ & \Sh (\SH) & $\preceq_i$ & $\phantom{\max(\gapcost,{}}\seedcost$ \\
    $\hcsh(u)$ & Chaining seed h. (\CSH) & $\preceq$   & $\phantom{\max(\gapcost,{}}\seedcost$ \\
    $\hgch(u)$ & Gap-chaining seed h. (\GCH)\!\! & $\preceq$   & $\max(\gapcost,\seedcost)$ \\
    \bottomrule
    \end{tabular}
	\caption{\textbf{Definitions of our heuristic functions.} \SH orders the matches by
    $i$ and uses only the seed cost. \CSH orders the matches by both $i$ and
    $j$. \GCH additionally exploits the gap cost.}
	\label{tab:heuristics}
\end{table}

We instantiate our heuristics according to~\cref{tab:heuristics}. Our
admissibility proofs~(\cref{sec:admissible-proof}) are based on $\matchcost$ and
$\gamma$ being lower bounds on disjoint parts of the remaining path. Since the
more complex $\hgch$ dominates the other heuristics it usually expand fewer
states.

\begin{restatable}{thm}{thmadmissible}\label{thm:admissible} The \sh $\hsh$\,,
  the \csh $\hcsh$\,, and the \gch $\hgch$\, are \emph{admissible}. Furthermore,
  $\hshM(u) \leq \hcshM(u) \leq \hgchM(u)$ for all states $u$.
\end{restatable}

We are now ready to instantiate \A with our admissible heuristics but we will
first improve them and show how to compute them efficiently.

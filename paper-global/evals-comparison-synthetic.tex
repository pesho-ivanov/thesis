\subsection{Comparison on synthetic data} \label{GLOBALsec:evals-comparison-synthetic}

Here we compare the runtime scaling and performance of the exact optimal
aligners on synthetic data.
\cref{GLOBALsec:techniques} compares the effects of pruning and inexact matches,
whereas~\cref{GLOBALsec:expanded} compares \SH and \CSH in terms of the number of
expanded states.

\paragraph{Scaling with sequence length}
% figs: better scaling => faster for long
First, we compare the aligners by runtime scaling with sequence length $n$ for
various error rates $e$ (\cref{GLOBALfig:scaling-n}). As expected from the theoretical
analysis, \edlib and \wfa scale approximately quadratically regardless of the
error rate. Unlike them, \SH and \CSH scale subquadratically which explains why
they are faster for long enough sequences.

% for low error rates
More specifically, for low error rates ($e{=}1\%$ and $e{=}5\%$), both \SH and
\CSH with exact matches scale near-linearly with sequence length (best
fit~$n^{1.08}$ for \mbox{$n{\le}10^7$}), and the benefit from chaining is
negligible.

% for high error rates
For high error rates ($e{\geq}10\%$) and large $n$, the need to match the seeds
inexactly causes a significant number of off-path matches. These off-path
matches lower the value of the heuristics and prevent the punishment of
suboptimal paths. This causes \SH to expand a super-linear number of states
(\cref{GLOBALsec:expanded}). Chaining the matches enforces them
to follow the order of the seeds, which greatly reduces the negative effects of
the off-path matches, leading to only linearly many expanded states.
Nevertheless, the runtime scaling of \CSH
is super-linear as a result of the increasing fraction of time (up to $93\%$ for
$n{=}10^7$) spent on reordering states because of outdated $f$ values after
pruning.

%(TODO: high variance; more seqs needed)
\paragraph{Performance}
\cref{GLOBALtab:evals} compares the runtime and memory of \astarpa with match
pruning to \edlib (based on banding, exponential search, bit-parallel) and \wfa
(based on diagonal transition and divide \& conquer) for aligning a single pair
of sequences of length $n{=}10^7$. \A with \CSH (\cshsymbol~\cshsymbolsq) is
more than $250$ times faster than \edlib and \wfa at $e{=}5\%$. For low error
rate ($e{\leq}5\%$), there is no significant performance difference between \SH
(\shsymbol~\shsymbolsq) and \CSH. The memory usage of \astarpa for $e{\leq}5\%$
is less than $\qty{600}{MB}$ for any $n$. At $e{=}15\%$, \CSH uses at most
$\qty{4.7}{GB}$ while \SH goes out of memory ($\geq\qty{30}{GB}$) because there
are too many expanded states to store.

   
%& \A, match-pruning, \sh                      
%& \A, match-pruning, \csh                     

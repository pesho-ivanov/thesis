\begin{figure}[t]%
  \captionsetup[subfloat]{justification=RaggedLeft,singlelinecheck=false}
    \centering
    \subfloat[\\Exponential band\\(\edlib)]{\includegraphics[width=0.3\linewidth]{imgs/fig1/1_ukkonen.png}\label{GLOBALfig1-band}}
    \hspace{-8em}
    \hspace{2.5em}
    \subfloat[\\\dijkstra] {\includegraphics[width=0.3\linewidth]{imgs/fig1/2_dijkstra.png}\label{GLOBALfig1-dij}}
    \hspace{-8em}
    \hspace{2.5em}
    \subfloat[\\DT\\(\oldwfa)]{\includegraphics[width=0.3\linewidth]{imgs/fig1/3_diagonal-transition.png}\label{GLOBALfig1-wfa}}
    \hspace{-8em}
    \hspace{2.5em}
    \subfloat[\\DT+D\&C\\(\wfa)]{\includegraphics[width=0.3\linewidth]{imgs/fig1/4_dt-divide-and-conquer.png}\label{GLOBALfig1-biwfa}}
    \hspace{-8em}
    \hspace{2.5em}
    \subfloat[\\\textbf{This
    work}\\\textbf{(\astarpa)}]{\includegraphics[width=0.3\linewidth]{imgs/fig1/5_astar-csh-pruning.png}\label{GLOBALfig1-astar}}
    \caption[Behavior of various global alignment algorithms]{%
      Demonstration of the computed states by various optimal alignment
algorithms and corresponding aligners that implement them on synthetic data
(length $n{=}500\bp$, error rate $e{=}20\%$). Blue-to-red coloring indicates the
order of computation. \protect\subref{GLOBALfig1-band} Exponential banding
algorithm (\edlib), \protect\subref{GLOBALfig1-dij} \dijkstra,
\protect\subref{GLOBALfig1-wfa} Diagonal transition/DT (\oldwfa),
\protect\subref{GLOBALfig1-biwfa} Diagonal transition with
divide-and-conquer/D\&C (\wfa), \protect\subref{GLOBALfig1-astar} \A with \csh
and match pruning (seed length $k{=}5$ and exact matches). This figure is
produced by Ragnar~Groot Koerkamp and Mykola~Akulov.}
    \label{GLOBALfig:comparison}
\end{figure}


\section{Overview}

Our algorithm exactly solves global pairwise alignment for edit distance costs,
also known as \emph{Levenshtein distance}~\citep{levenshtein1966binary}. It uses
the \A algorithm to find a shortest path in the alignment graph.

\paragraph{\Sh} In order to handle higher error rates, we extend the \emph{\sh}
to \emph{inexact matches}, allowing up to $1$ error in each match.  To handle
cases with a large number of seed matches we introduce \emph{match chaining},
constraining the order in which seed matches can be
linked~\citep{wilbur1984context,benson2016lcsk}. We prove that our \emph{\csh}
with inexact matches is admissible, which guarantees that \A finds a shortest
path.

\paragraph{Match pruning}
In order to reduce the number of states expanded by \A we apply the
\emph{multiple-path pruning} observation of \citet{poole2017artificial}: once a
shortest path to a vertex has been found, no other paths to this vertex can
improve the global shortest path. We prove that when a state at the start of a
match is expanded, a shortest path to this state has been found. Since no other
path to this state can be shorter, we show that we can \emph{prune} (remove) the
match, thus improving the \sh. This incremental heuristic search has some
similarities to Real-time Adaptive \A~\citep{koenig2006real}.

\paragraph{Implementation}
We efficiently implement our algorithm in the \astarpa aligner. In particular,
we use \emph{contours}
\citep{hirschberg1977algorithms,hunt1977fast,pavetic2017fast} to efficiently to
compute the \csh, and update them when pruning matches.

\paragraph{Scaling and performance}
We compare the scaling and performance of our algorithm to other exact aligners
on synthetic data, consisting of random genetic sequences with up to $15\%$
uniform errors and up to $10^7$ bases. We demonstrate that inexact matches and
match chaining enable scaling to higher error rates, while match pruning enables
near-linear scaling with length by reducing the number of expanded states to not
much more than the best path (\cref{GLOBALfig1-astar}). Our empirical results
show that for $e{=}5\%$ and $n{=}10^7\bp$, \astarpa outperforms the leading
aligners \edlib~\citep{vsovsic2017edlib} and \wfa~\citep{marco2022optimal} by
more than $250$ times.

We demonstrate a limited applicability of our algorithm to long Oxford
Nanopore (ONT) reads from human samples. \astarpa is the fastest exact aligner
on $58\%$ of the alignments on a dataset with only sequencing
errors, and on $17\%$ of the alignments on a dataset with biological
variation.

%\paragraph{Chaining matches} Given a set of matching substrings in the two
%sequences, the problem of connecting (\emph{chaining}) a maximal number of
%matches in increasing order in both sequences has been solved in
%$\Oh(M{\log}M)$~\citep{eppstein1992sparse,myers1992n2}.
%%\citet{myers1995chaining} Chaining of seed matches have been applied
%%by~\citep{pavetic2017fast}.

%\paragraph{Contours} \citet{hunt1977fast} developed an $\Oh((M+n) \log n)$
%algorithm using \emph{thresholds} by considering all $M$ pairs of indices of
%equal (\emph{matching}) elements in both sequences.  This is used in the
%\difftool tool \citep{hunt1976algorithm} where the sequence elements are lines
%in text files.  For nucleotide sequences where the alphabet size is small, $M$
%grows as $n^2$, leading to $\Oh(n^2 \log n)$ runtime even for similar sequences.
%%\citet{hirschberg1977algorithms} extends the concept of \emph{thresholds} to
%%\emph{dominant matches} and \emph{contours}.

%% LCSk++, Seed heuristic, possibly add myers2014efficient for seeds
%\paragraph{Seeds} \citet{benson2014longest} introduce an approach where only
%groups of $k$ consecutive letters can be matched.
%\citet{deorowicz2014efficient} extend the algorithm of \citet{hunt1977fast} to
%this setting. They first group the sequence letters into overlapping seeds of
%size $k$, effectively increasing the alphabet and hence reducing the number of
%matches $M$. Using an appropriate choise of $k$, they achieve $\Oh(n\log n)$
%expected runtime on random sequences. We use non-overlapping seeds in a
%\emph{seed heuristic} for \A to do exact semi-global alignment in the case of
%long and related sequences.
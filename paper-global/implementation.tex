\subsection{Implementation}\label{GLOBALsec:impl}

This section covers some implementation details which are necessary for a good
performance.

\paragraph{Bucket queue}
We use a hashmap to store all computed values of $g$ in the \A algorithm.
Since the edit costs are bounded integers, we implement the priority queue
using a \emph{bucket queue}~\citep{bertsekas1991linear}. Unlike
heaps, this data structure has amortized constant time push and pop operations
since the difference between the value of consecutive pop operations is bounded.

\paragraph{Greedy matching of letters}
From a state $\st ij$ where $a_i = b_j$, it is sufficient to only consider the
matching edge to $\st {i+1}{j+1}$~\citep{allison_lazy_1992}, and ignore the
insertion and deletion edges to $\st i{j+1}$ and $\st {i+1}j$. During alignment,
we greedily match as many letters as possible within the current seed before
inserting only the last opened state in the priority queue. We do not cross seed
boundaries in order to not interfere with match pruning. We include greedily
matched states in the reported number of expanded states.

\paragraph{Priority queue offset}
Pruning the last match in a layer may cause an increase of the heuristic in all
states preceding the start $u$ of the match. This invalidates $f$ values in the
priority queue and causes reordering (step \cref{GLOBALastar:pruning-check}).
We skip most of the update operations by keeping a global
offset to the $f$-values in the priority queue, which is updated it when all
states in the priority queue precede $u$.

\paragraph{Split vector for layers}
Pruning a match may lead to the removal of one or more layers of the array of
layer starts $LS$ or the array of layer matches $LM$, after which all higher
layers are shifted down. To support this efficiently, we use a \emph{split
vector}: a data structure that internally consists of two stacks, one containing
the layers before the last deletion, and one containing the layers after the
last deletion in reverse order. To delete a layer, we move layers from the top
of one of the stacks to the top of the other, until the layer to be deleted is
at the top of one of the stacks, and then remove it. When layers are removed in
decreasing order of $\ell$, this takes linear total time.

\paragraph{Binary search hints}
When we open $v$ from its parent $u$ in step \cref{GLOBALastar:open} of the \A
algorithm, the value of $h(v)$ is close to the value of $h(u)$. In particular,
$\shscore(v)$ (resp. $\cshscore(v)$) is close to and at most the score in $u$.
Instead of a binary search (step \cref{GLOBALstep:bin-search}), we do a
linear search starting at the value of $\ell=\shscore(u)$ (resp. $\ell=\cshscore(u)$).

We also speed up the binary search in the recomputation of $f(u)$ in the
reordering check (step \cref{GLOBALastar:pruning-check}) by storing the last computed
value of $hint(u):=\shscore(v_s) - \shscore(u)$ (resp.
$\cshscore(v_s) - \cshscore(u)$) in the hashmap. When evaluating $\shscore(u)$,
we start the linear search at $\shscore(v_s) - hint(u)$ with a fallback to
binary search in case the linear search needs more than $5$ iterations. Since
$hint(u)$ remains constant when matches starting after $u$ are pruned, it is a
good starting point for the search when only matches near the tip of the \A
search are pruned.

\paragraph{Code correctness}
Our implementation \astarpa is written in Rust, contains many assertions testing
\eg the correctness of our \A and layers data structure implementation,
and is tested for correctness and performance on randomly-generated sequences.
Correctness is checked against simpler algorithms (Needleman-Wunsch) and other
aligners (\edlib, \wfa).

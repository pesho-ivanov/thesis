\subsection{Contributions} \label{sec:contributions}

We solve global pairwise alignment exactly and efficiently using the \A shortest
path algorithm. In order to handle long and divergent sequences, we generalize
the existing seed heuristic to a novel \csh which includes chaining, inexact
matches, and gap costs. Then we improve the heuristic using a novel \emph{match
pruning} technique and optimize the \A search using the existing diagonal
transition technique. We prove that all our heuristics guarantee that \A finds a
shortest path.

\paragraph{\Csh} The \emph{\sh}~\citep{ivanov2022fast} penalizes missing seed
matches. Our \csh requires matches to be chained in order, which reduces the
negative effect of spurious (off-track)
matches~\citep{wilbur1984context,benson2016lcsk}, and improves performance for
highly divergent sequences. To handle long indels, we introduce the \emph{\gch}
which uses a \emph{gap cost} for indels between consecutive matches.

\paragraph{Inexact matches} We match seeds \emph{inexactly} by aligning each
seed with up to $1$ edit, enabling our heuristics to potentially double in value and
cope with up to twice higher divergence.

\paragraph{Match pruning}
In order to further improve our heuristics, we apply the \emph{multiple-path pruning}
observation of \citet{poole2017artificial}: once a shortest path to a vertex has
been found, no other paths through this vertex can improve the global shortest
path. When \A expands the start or end of a match, we remove (\emph{prune}) this
match from further consideration, thus improving (increasing) the heuristic for
the states preceeding the match. The incremental nature of our heuristic search is
remotely related to Real-Time Adaptive~\A~\citep{koenig2006real}. Match pruning
enables near-linear runtime scaling with length~(\cref{fig1-astar}).

\paragraph{Diagonal transition \A}
We speed up our algorithm by combining it with the diagonal transition
optimization that skips states where a farther state along the same diagonal
has already been reached with the same distance~(\cref{fig1-wfa}).

\paragraph{Implementation}
We implement our algorithm in \astarpa (\A Pairwise Aligner). The efficiency of
our implementation relies on contours, which enables near-constant time
evaluation of the heuristic.

\paragraph{Scaling and performance}
We compare the absolute runtime, memory, and runtime scaling of \astarpa to
other exact aligners on synthetic data, consisting of random genetic sequences
(length $n{\leq}10^7\bp$, uniform divergence $d{\leq}12\%$). For $d{=}4\%$ and
$n{=}10^7\bp$, \astarpa is up to $300\times$ faster than the fastest aligners
\edlib~\citep{vsovsic2017edlib} and \wfa~\citep{marco2022optimal}. Our real
dataset experiments involve ${>}500\kbp$ long Oxford Nanopore (ONT) reads with
$d{<}19.8\%$. When only sequencing errors are present, \astarpa is $2\times$
faster (in median) than the second fastest, and when genetic variation is also
present, it is $1.4\times$ faster.

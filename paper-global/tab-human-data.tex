\begin{table}[t]
  \centering
  \sffamily
  \small
  \tabcolsep=0.11cm
  \begin{tabular}{
    @{}
    l
    S[table-format=2]
    S[table-format=3]
    S[table-format=3]
    S[table-format=3]
    S[table-format=1.1]
    S[table-format=1.1]
    S[table-format=2.1]
    S[table-format=1.2]
    S[table-format=1.1]
    S[table-format=2]
    @{}
    }
    \toprule
    &
    & \multicolumn{3}{c}{\textbf{Length} [\kbp]}
    & \multicolumn{3}{c}{\textbf{Divergence} [$\%$]}
    & \multicolumn{3}{c}{\textbf{Max gap} [\kbp]}
    \\
    \cmidrule(lr){3-5} \cmidrule(lr){6-8} \cmidrule(lr){9-11}
    \textbf{Dataset}
    & {\!\!\textbf{Cnt}} &
    {\small min} & {\small \!mean} & {\small max} &
    {\small min} & {\small \!mean\!} & {\small max} &
    {\small min} & {\small \!mean\!} & {\small max} \\
    \midrule
    ONT & 50
     & 500 & 594 & 849 & 2.7 & 6.3 & 18.0 & 0.02 & 0.1 & 1 \\
    ONT+gen.var.\!\! & 48
     & 502 & 632 & 1053 & 4.4 & 7.4 & 19.8 & 0.05 & $\mathbf{1.9}$ & $\mathbf{42}$ \\
    \bottomrule
  \end{tabular}
  \caption{\textbf{Human datasets statistics.} ONT reads only include short
    gaps, while genetic variation also includes long gaps. \textbf{Cnt}: number of
    sequence pairs. \textbf{\mbox{Max gap}}: longest gap in the reconstructed
    alignment.}
  \label{tab:hg}
\end{table}
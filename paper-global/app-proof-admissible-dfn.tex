\paragraph{\A with a \wah finds a shortest path}
\label{sec:weak-admissible-dfn}

\begin{definition}[Fixed node]
  A node $u$ is \emph{fixed} if it is expanded and \A has found a shortest path
  to it, that is, $g(u) = \g(u)$.
\end{definition}
A fixed node cannot be opened again (\cref{alg:astar}, line~$\ref{alg:astar:open}$),
and hence remains fixed.

\begin{definition}[\Wa]\label{dfn:weak-admissible}%
  A heuristic $\hh$ is \emph{\wa} if at any moment during the \A search there
  exists a shortest path~$\pi^*$ from~$v_s$~to~$v_t$ in which all nodes $u
  \in \pi^*$ after its last fixed node $n^*$ satisfy $\hh(u) \leq \h(u)$.
\end{definition}

To prove that \A finds a shortest path when used with a \wah, we follow the
structure of \citet{hart1968formal}. First we restate their Lemma 1 in our
notation with a slightly stronger conclusion that follows directly from
their proof.

\begin{lem}[Lemma 1 of \citet{hart1968formal}]\label{lemma1}
  For any unfixed node $n$ and for any shortest path $\Path$ from $v_s$ to $n$,
  there exists an open node $n'$ on $\Path$ with $g(n') = \g(n')$ such that
  $\Path$ does not contain fixed nodes after $n'$.
\end{lem}

Next, we prove that in each step the \A algorithm can proceed along a shortest path
to the target:

\begin{cor}[Generalization of Corollary to Lemma 1 of \citet{hart1968formal}]\label{cor:cor}
  Suppose that $\hh$ is \wa, and that \A has not terminated. Then, there exists
  an open node $n'$ on a shortest path from $v_s$ to $v_t$ with $f(n') \leq
  \g(v_t)$.
\end{cor}
\begin{proof}
  Let $\Path$ be the shortest path from $v_s$ to $v_t$ given by the \way of $\hh$ (\cref{dfn:weak-admissible}).
  Since \A has not terminated, $v_t$ is not fixed.
  Substitute ${n=v_t}$ in \cref{lemma1} to derive that
  there exists an open node $n'$ on $\Path$ with $g(n') = \g(n')$. By
  definition of $f$ we have $f(n') = g(n') + \hh(n')$. Since $\Path$ does not
  contain any fixed nodes after $n'$, the \way of $\hh$ implies $\hh(n')
  \leq \h(n')$. Thus, ${f(n') = g(n') + \hh(n') \leq \g(n') + \h(n') = \g(v_t)}$.
\end{proof}

\thmpartialadmissibledfn*
\begin{proof}
  The proof of Theorem~1 in \citet{hart1968formal} applies, with the remark that
  instead of an arbitrary shortest path, we use the specific path $\Path$
  given by the \way and the specific node $n'$ given by \cref{cor:cor}.
\end{proof}

\subsection{Overview} \label{GLOBALsec:intuition}

% Shortest path task, \A
In order to find a minimal cost alignment of $A$ and $B$, we use the \A
algorithm to find a shortest path from the starting state $v_s{=}\st 00$ to the
target state $v_t{=}\st nm$ in the alignment graph. We present two heuristic functions,
the \sh and the \csh, and prove that they are admissible, so that \A always finds a
shortest path.

% SH
To define the \sh $\hsh$\,, we split $A$ into short, non-overlapping substrings
(\emph{seeds}) of fixed length $k$. Since the whole of sequence $A$ has to be
aligned, each of the seeds also has to be aligned somewhere. If
a seed cannot be \emph{exactly} matched in $B$ without mistakes, then at least
one edit has to be made to align it. We first compute all positions in
$B$ where each seed from $A$ matches exactly. Then, a
lower bound on the edit distance between the remaining suffixes $A_{\geq i}$ and
$B_{\geq j}$ is given by the number of seeds entirely contained in $A_{\geq i}$
that do not match anywhere in $B$. An example is shown in \cref{GLOBALfig:sh}.

% CSH
We improve the \sh by enforcing that the seed matches occur in the same order as
their seeds occur in $A$, \ie, they form a \emph{chain}. Now, the number of upcoming
errors is at least the minimal number of remaining seeds that cannot be aligned
on a single path to the target. The \csh $\hcsh$ equals the number of remaining
seeds minus the maximum length of a chain of matches, see \cref{GLOBALfig:csh}.

% inexact matches
To scale to larger error rates, we generalize both heuristics to use
\emph{inexact matches}. For each seed from $A$, our algorithm finds all its
inexact matches in $B$ with cost at most $1$. Then, not using any match of a
seed will require at least $r{=}2$ edits for the seed to be eventually aligned.

% pruning
Finally, we use \emph{match pruning}: once we find a shortest path to a state
$v$, no shorter path to that state is possible. Since we are only looking for a
single shortest path, we may ignore any other path going to $v$. In particular,
when we are evaluating the heuristic $h$ in some state $u$ preceding $v$, and
$v$ is at the start of a match, all paths containing the match are excluded, and
thus we may simply ignore (\emph{prune}) this match for future computations of
the heuristic. This increases the value of the heuristic, as shown in \cref{GLOBALfig:pruning}, and leads to a
significant reduction of the number of states being expanded by the \A.

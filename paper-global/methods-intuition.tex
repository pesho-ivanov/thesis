\subsection{Intuition} \label{sec:intuition}

% Shortest path task, A*
To align sequences~$A$ and~$B$ with minimal cost, we use the \A shortest path
algorithm from the start to the end of the alignment
graph~(\cref{app:comparison}). \A uses a heuristic function that estimates the
shortest distance from the current vertex to the target. To ensure that \A finds a
shortest path, the heuristic must be \emph{admissible}, i.e. never overestimate
the distance. A good heuristic efficiently computes an accurate estimate of
the remaining distance.

% SH
\paragraph{\Sh (\SH)} To define the \emph{\sh} $\hsh$\,, we split $A$ into short,
non-overlapping substrings~(\emph{seeds}) of fixed length $k$. Since the whole
sequence~$A$ has to be aligned, each of the seeds also has to be aligned
somewhere in~$B$. If a seed does not match anywhere in~$B$ without mistakes,
then at least $1$ edit has to be made to align it. To this end, we precompute
for each seed from~$A$, all positions in~$B$ where it matches exactly. Then, the
\SH is the number of remaining seeds~(contained in~$A_{\geq i}$) that do not
match anywhere in~$B$~(\cref{fig:sh}), and is a lower bound on the distance
between the remaining suffixes $A_{\geq i}$ and~$B_{\geq j}$. Next, we improve
the \sh with chaining, inexact matching, and gap costs.

% CSH
\paragraph{Chaining (\CSH)} We improve on the \SH by enforcing that the matches
occur in the same order in $B$ as their corresponding seeds occur in $A$, i.e.,
the matches form a \emph{chain} going down and to the right. Now, the number of
upcoming errors is at least the minimal number of remaining seeds that cannot be
aligned on a single chain to the target. To compute this number efficiently, we
instead count the maximal number of matches in a chain from the current state,
and subtract that from the number of remaining seeds~(\cref{fig:csh}). This
improves the \sh when there are many spurious matches (i.e. outside the optimal
alignment).

% GCH
\paragraph{Gap costs (\GCH)} In order to penalize long indels, the \gch $\hgch$
(\cref{fig:gch}) extends \csh by also incurring a \emph{gap cost} if two
consecutive matches in a chain do not lie on the same diagonal. In such cases,
the transition to the next match requires at least as many gaps as the
difference in length of the two substrings.

% inexact matches
\paragraph{Inexact matches} To scale to higher divergence, we use
\emph{inexact matches}. For each seed in $A$, our algorithm finds all its inexact
matches in $B$ with cost at most $1$. Then, not using any match of a seed will
require at least $r{=}2$ edits for the seed to be eventually aligned. Thus, the
\emph{potential} of our heuristics to penalize errors roughly doubles.

% pruning
\paragraph{Pruning} We introduce the \emph{match pruning}
improvement~(\cref{fig:pruning}) for all our heuristics: once we find a shortest
path to a state $v$, no shorter path to that state is possible. Since we are
only looking for a single shortest path, we can ignore all alternative paths to
$v$. If $v$ is the start or end of a match and $v$ has been expanded, we are not
interested in alternative paths using this match, and hence we simply
ignore~(\emph{prune}) this match when evaluating a heuristic at states preceding
$v$. Pruning a match increases the heuristic in states before the match, thus
penalizing states preceding the ``tip'' of the \A search. Pruning significantly
reduces the number of expanded states, and leads to near-linear scaling with the
sequence length.

% DT
\paragraph{Diagonal transition (\DT)}
The DT optimization~(\cref{sec:dt}) only computes a subset of \emph{farthest
reaching} states. This speeds up the \A in cases where the search becomes
quadratic~(\cref{fig1-wfa}).

Finally, we prove correctness of all heuristics and optimizations, and
in \cref{sec:computation} we show how to compute the heuristics efficiently.

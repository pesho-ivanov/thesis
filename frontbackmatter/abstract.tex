%*******************************************************
% Abstract
%*******************************************************
%\renewcommand{\abstractname}{Abstract}
\pdfbookmark[1]{Abstract}{Abstract}
\begingroup
\let\clearpage\relax
\let\cleardoublepage\relax
\let\cleardoublepage\relax

\chapter*{Abstract}

% Context
Sequence alignment is the process of detecting similarities between biological
sequences, such as DNA, RNA and proteins. For the last half a century,
variations of this problem have been of central importance for molecular
biology. Alignment is usually a building block for a wide range of applications,
include evolutional biology, genome assembly, read mapping, variation detection
aComputational methods are crucial for the correctness of the analyses and the
practical applicability to the vast amounts of biological data.

% Problem
Biological sequences do not generally align perfectly due to biological
differences and technical errors. Instead, the best imperfect alignment is
desired. a similarity metric, such as \emph{edit distance}, is usually being
optimized. Generally, an alignment algorithm is desired to compute
\textcircled{1} accurate alignments with \textcircled{2} good performance, and
\textcircled{3} applicability to various data. We consider a principled shortest
path formulation of alignment and argue that the A* algorithm enables the
desired properties. In particular, we demonstrate \textcircled{1} optimality
guarantees according to edit distance, \textcircled{2} scaling to
\textcircled{3} satisfies these basic desires by providing optimality guarantees
according to edit distance metric and scaling near-linear for similar sequences.

%(approximate vs probabilistic vs exact/optimal algorithm), Runtime, memory
%usage, scaling.
% Data variation
%A good algorithm should be optimal, complete, performant (fast and low on
%memory) on short and long sequences, various error rates and error models,
%and reference complexity (topology and enthropy).

% Algorithms and Asymptotics
The classical dynamic programming (DP) algorithms run in quadratic time, whereas
the resulting alignment is only linear. This performance gap has motivated a
numerous algorithms which are faster in the expense of accuracy. Moreover,
theoretical analyses suggest that it is not possible to solve alignment in
strongly subquadratic time in the worst case. In this thesis we argue that on
real data an informed search algorithm, such as A*, can radically improve the scaling
without sacrificing optimality guarantees. In some cases, the scaling may even
be linear.

% Contribution overview
In this thesis, we consider the alignment as a shortest path problem and explore
its solution using the A* algorithm. Most importantly, this approach enables us
to design heuristic functions which guarantee finding an optimal alignment while
achieving polynomial speed ups on real data. This approach allows to find a
base-to-base alignment that minimizes general edit distance. Our A* approach is
applicable to various settings of the alignment problem.

% A*
Informed search. Admissibility. Seed heuristic.

% Tools and Performance
Many tools do not have a well-stated problem they optimize.

% Future work and limitations
Probabilistic approach.
Focus on the metric.
Extend to MSA, local, affine.
Prorotype implementations

\endgroup

\cleardoublepage%

\begingroup
\let\clearpage\relax
\let\cleardoublepage\relax
\let\cleardoublepage\relax

\begin{otherlanguage}{ngerman}
\pdfbookmark[1]{Zusammenfassung}{Zusammenfassung}
\chapter*{Zusammenfassung}

Deutsche Zusammenfassung hier.

\end{otherlanguage}

\endgroup

\vfill
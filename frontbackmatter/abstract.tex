%*******************************************************
% Abstract
%*******************************************************
%\renewcommand{\abstractname}{Abstract}
\pdfbookmark[1]{Abstract}{Abstract}
\begingroup
\let\clearpage\relax
\let\cleardoublepage\relax
\let\cleardoublepage\relax

\chapter*{Abstract}

% History and motivation 
Sequence alignment is the process of detecting similarities between sequences.
Since genetic sequences were first sequenced half a century ago, sequence
alignment is a basic task in molecular biolog, with applications in evolutionary
biology, genome assembly, variation detection, and others. An ongoing transition
from using genomes to using pangenomes motivates a rethinking of the classic
alignment algorithms. The variety of applications combined with the growing
amount of genetic data motivate the development of fast and accurate alignment
algorithms. 

% Goals
Existing alignment algorithms are either optimal but quadratic or fast but
approximate. This thesis proses an elegent approach to alignment based on the \A
algorithm, which is both heuristically fast and provably optimal. It has been
shown that alignment is likely not solvable in strongly subquadratic time in the
general case. The goal we pursue throughout this thesis is to apply the \A
approach to as many types of data as possible, while remaining fast and optimal.

% Tasks and algorithmic contribution
We consider two types of alignment: \emph{semi-global}, for mapping a set of DNA
sequences to a pangenome reference; and \emph{global}, for calculating the edit
distance distance between two sequences. In order to handle various data
dimensions, we propose several techniques and empyrically study their runtime
scaling: a trie index enables sublinear scaling with the reference size,
\emph{seed heuristic} enables near-linear scaling with sequence length, and
inexact seed matching and match chaining enable scaling to high error rates.
Owing to the superior scaling of the \A approach, our prototypical
implementations run orders of magnitude faster than existing optimal approaches
even on long erroneous sequences.

% Future work and limitations
We foresee a multitude of future directions for advancing \A for sequence
alignment, including other types of alignment, generalizing the edit distance
metric, relaxing the optimality guarantee, theoretical analyses of the
performance, and more efficient implementations for production use.

\endgroup

\cleardoublepage%

\begingroup
\let\clearpage\relax
\let\cleardoublepage\relax
\let\cleardoublepage\relax

\begin{otherlanguage}{ngerman}
\pdfbookmark[1]{Zusammenfassung}{Zusammenfassung}
\chapter*{Zusammenfassung}

Deutsche Zusammenfassung hier.

\end{otherlanguage}

\endgroup

\vfill

% Problem
%Biological sequences do not generally
%align perfectly due to biological differences and technical errors. Given two
%sequences, the desired alignment is a position-to-position correspondence
%between two sequences which minimizes the edit costs (substitutions, insertions
%or deletions).
%This task is closely related to calculating \emph{edit distance}.

% Existing algorithms
% by using heuristic
%information to speed up the alignment without sacrificing correctness.
%to to making it both heuristically fast and provably optimal. 

%Can we use the A* algorithm to find provably optimal alignment heuristically
%fast?

%Practical alignment algorithms are desired to \textcircled{1} find accurate
%alignments, \textcircled{2} apply to a wide range of data, and \textcircled{3}
%use little time and memory.

% Shortest path formulation
%We consider a principled alignment formulation based on shortest paths and
%demonstrate that the A* shortest path algorithm can be used to outperform
%current methods. Unlike existing methods, A* enables an \emph{informed search}
%based on information from the unaligned sequence suffixes, thus radically

%Runtime, memory usage, scaling. Data variation A good algorithm should be
%optimal, complete, performant (fast and low on memory)

%polynomial speed ups on real data. It \textcircled{1} provides optimality
%guarantees according to edit distance, \textcircled{2} scales to long and noisy
%sequences, and \textcircled{3} scales subquadratically with sequence length.

%To scale to large reference sequences, we extend the graph with a trie index. To
%scale to long queries, we introduce design an admissible \emph{seed heuristic},
%which is provably-optimal also efficient to compute. To scale to high error
%rates, we design  

%improving the empyrical runtime scaling (up to linear) in the average case while
%providing optimality guarantees.

%mapping on pangenomes, graph references, 
%*******************************************************
% Abstract
%*******************************************************
%\renewcommand{\abstractname}{Abstract}
\pdfbookmark[1]{Abstract}{Abstract}
\begingroup
\let\clearpage\relax
\let\cleardoublepage\relax
\let\cleardoublepage\relax

\chapter*{Abstract}

% History and motivation 
Sequence alignment is the process of detecting similarities between sequences.
Since DNA, RNA and proteins were first sequenced half a century ago, the
alignment task has been a central building block in molecular biology. Its
applications include evolutionary biology, genome assembly, variation detection,
and others. The variety of applications and the growing amount of genetic data
require accuracy and speed from the alignment algorithms. Moreover, the ongoing
transition from using single genome references to using pangenomes motivates a
rethinking of the classic alignment algorithms.

% Goals
Existing alignment algorithms are either optimal but quadratic, or fast but
approximate. This thesis proposes an elegant approach to alignment based on the
\A algorithm for finding shortest paths in graphs. \A uses a problem-specific
heuristic function that informs it about the not-yet-aligned parts of the
sequences to make it both heuristically fast and provably optimal. In the
general case, the alignment problem is likely not solvable in strongly
subquadratic time. Our theme throughout this thesis is to apply the \A approach
to as many types of data as possible, while remaining fast and optimal.

% Tasks and algorithmic contribution
We apply the \A approach to two types of alignment minimizing edit distance:
\emph{semi-global}, to map a set of DNA sequences to a pangenome reference, and
\emph{global} to align two sequences end-to-end. In order to handle various
dimensions of the data complexity, we propose several techniques and empirically
study their runtime scaling. A trie index enables sublinear scaling with the
reference size, \emph{seed heuristic} enables near-linear scaling with sequence
length, and inexact seed matching and match chaining enable scaling to high
error rates. Owing to the superior scaling of the \A approach, our prototypical
implementations run orders of magnitude faster than existing optimal approaches
even on long erroneous sequences.

% Future work and limitations
We foresee a multitude of future directions for advancing \A for sequence
alignment, including other types of alignment, generalizing the edit distance
metric, relaxing the optimality guarantee, theoretical analyses of the
performance, superior heuristic functions and algorithms, and more efficient
implementations for production use.

\endgroup

\cleardoublepage%

\begingroup
\let\clearpage\relax
\let\cleardoublepage\relax
\let\cleardoublepage\relax

%\begin{otherlanguage}{ngerman}
\pdfbookmark[1]{Zusammenfassung}{Zusammenfassung}
\chapter*{Zusammenfassung}

Sequenzalignment ist der Prozess der Erkennung von Ähnlichkeiten zwischen
Sequenzen. Seit vor einem halben Jahrhundert zum ersten Mal genetische Sequenzen
sequenziert wurden, ment ist eine grundlegende Aufgabe in der Molekularbiologie
mit Anwendungen in der Evolutionsbiologie, Genomassemblierung,
Variationserkennung und andere. Ein laufender Übergang von uns- Die Umstellung
von Genomen auf die Verwendung von Pangenomen motiviert zum Überdenken der
klassischen Ausrichtung Algorithmen. Die Vielfalt der Anwendungen kombiniert mit
der wachsenden Menge an genetische Daten motivieren die Entwicklung schneller
und genauer Ausrichtungsalgorithmen.

Existierende Ausrichtungsalgorithmen sind entweder optimal, aber quadratisch
oder schnell, aber geeignet. nahe. Diese Arbeit prosiert einen eleganten Ansatz
zur Ausrichtung basierend auf dem \A Algorithmus, der sowohl heuristisch schnell
als auch nachweislich optimal ist. Es wurde gezeigt Diese Ausrichtung ist in
stark subquadratischer Zeit im Allgemeinen wahrscheinlich nicht lösbar Fall. Das
Ziel, das wir in dieser Arbeit verfolgen, ist die Anwendung des \A Ansatz als
möglichst viele Arten von Daten und bleibt dabei schnell und optimal. Wir
betrachten zwei Arten von Alignment: semi-global, um einen DNA-Satz zu kartieren
Sequenzen zu einer Pangenom-Referenz; und global zur Berechnung der
Bearbeitungsentfernung

Abstand zwischen zwei Folgen. Um verschiedene Datendimensionen handhaben zu
können, verwenden wir schlagen mehrere Techniken vor und untersuchen empirisch
ihre Laufzeitskalierung: ein Versuch Index ermöglicht sublineare Skalierung mit
der Referenzgröße, Seed-Heuristik ermöglicht nahezu lineare Skalierung mit
Sequenzlänge und ungenaue Seed-Matching und Match-Verkettung Skalierung auf hohe
Fehlerraten ermöglichen. Aufgrund der überlegenen Skalierung des \A sich nähern,
Unsere prototypischen Implementierungen laufen um Größenordnungen schneller als
bestehende optimale Anflüge auch bei langen Fehlsequenzen.

Wir sehen eine Vielzahl zukünftiger Richtungen für die Weiterentwicklung von \A
für Sequenz Ausrichtung, einschließlich anderer Ausrichtungsarten,
Verallgemeinerung der Bearbeitungsentfernungsmetrik, Lockerung der
Optimalitätsgarantie, theoretische Analysen der Leistung und Effizientere
Implementierungen für den Produktionseinsatz.
%\end{otherlanguage}

\endgroup
\vfill

% Problem
%Biological sequences do not generally
%align perfectly due to biological differences and technical errors. Given two
%sequences, the desired alignment is a position-to-position correspondence
%between two sequences which minimizes the edit costs (substitutions, insertions
%or deletions).
%This task is closely related to calculating \emph{edit distance}.

% Existing algorithms
% by using heuristic
%information to speed up the alignment without sacrificing correctness.
%to to making it both heuristically fast and provably optimal. 

%Can we use the A* algorithm to find provably optimal alignment heuristically
%fast?

%Practical alignment algorithms are desired to \textcircled{1} find accurate
%alignments, \textcircled{2} apply to a wide range of data, and \textcircled{3}
%use little time and memory.

% Shortest path formulation
%We consider a principled alignment formulation based on shortest paths and
%demonstrate that the A* shortest path algorithm can be used to outperform
%current methods. Unlike existing methods, A* enables an \emph{informed search}
%based on information from the unaligned sequence suffixes, thus radically

%Runtime, memory usage, scaling. Data variation A good algorithm should be
%optimal, complete, performant (fast and low on memory)

%polynomial speed ups on real data. It \textcircled{1} provides optimality
%guarantees according to edit distance, \textcircled{2} scales to long and noisy
%sequences, and \textcircled{3} scales subquadratically with sequence length.

%To scale to large reference sequences, we extend the graph with a trie index. To
%scale to long queries, we introduce design an admissible \emph{seed heuristic},
%which is provably-optimal also efficient to compute. To scale to high error
%rates, we design  

%improving the empyrical runtime scaling (up to linear) in the average case while
%providing optimality guarantees.

%mapping on pangenomes, graph references, 
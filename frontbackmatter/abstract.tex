%*******************************************************
% Abstract
%*******************************************************
%\renewcommand{\abstractname}{Abstract}
\pdfbookmark[1]{Abstract}{Abstract}
\begingroup
\let\clearpage\relax
\let\cleardoublepage\relax
\let\cleardoublepage\relax

\chapter*{Abstract}
Sequence alignment is the task of finding similarities between sequences. It has
been a central building block in molecular biology since DNA, RNA and protein
sequences were first obtained half a century ago. Sequence alignment is applied
to research and medicine, with applications in evolutionary biology, genome
assembly, oncology and many others. The analyses of the growing amounts of
genomic data require algorithms with high accuracy and speed. Moreover, the
ongoing transition from single genome references to pangenomes (representative
for a whole population) motivates novel alignment algorithms.

We consider the two problems: \emph{semi-global} alignment of a set of DNA reads
to a pangenome graph reference (possibly containing cycles), and \emph{global}
(end-to-end) alignment of two sequences. Recent theoretical results conclude
that strongly subquadratic algorithms are unlikely to exist for the worst case.
Moreover, the runtime and memory of existing optimal algorithms scale
quadratically even for similar sequences. An open problem is to develop an
algorithm with linear-like empirical scaling on inputs where the errors are
linear in $n$. In the present thesis we introduce an approach that aims to solve
this problem in order to develop practical optimal alignment algorithms.

Modern optimal aligners perform uninformed search -- they neglect information
from the not-yet-aligned parts of the sequences. We exploit this information in
a principled framework where an alignment with minimal edit distance is
equivalent to a shortest path in an alignment graph, and the remaining
information about the sequences is captured in a heuristic function that
estimates the length of a remaining shortest path. The classic shortest path
algorithm \A uses such a heuristic to direct the search, and yet it finds
provably shortest paths given that the heuristic is \emph{admissible}, i.e. it
always return a lower bound on the edit distance of the remaining suffixes.
Curiously, previous attempts to apply \A to multiple sequence alignment (MSA),
did not result in practical algorithms even for pairwise alignment. In the
present thesis we investigate how to do that. In the shortest path formulation,
we (i) suggest a novel highly-informed admissible heuristic, (ii) design
efficient algorithms and data structures for computing the heuristic, (iii)
prove their optimality, (iv) implement the presented algorithms, and (v) compare
their performance and scaling to other optimal algorithms. Our approach is
provably optimal according to edit distance, its runtime empirically scales
subquadratically (and sometimes even near-linearly) with the output size, which
translates to orders of magnitude of speedup compared to state-of-the-art
optimal algorithms.

Throughout this thesis, we demonstrate how to encompass various dimensions of
the input complexity while preserving the speed and optimality. First, we
demonstrate how to apply a trie index to scale the alignment runtime sublinearly
with the reference size. Then, we introduce a novel seed heuristic for \A which
enables aligning of long sequences (up to 100 Mbp) near-linearly with their
length. Last, we extend the seed heuristic to a general chaining seed heuristic
that encompases inexact matching, match chaining, and gap costs to raise the
tolerated error rate (to 30\% for synthetic data and 10\% for real data).
Interestingly, our seed heuristic challenges the omnipresent seed-(chain)-extend
paradigm to sequence alignment which aims to connect long and high-quality seed
matches. We, instead, use the information of the lack of matches to dismiss
suboptimal alignments, which leads to optimizing seeds for being short and not
having many matches.

These first steps to scalable optimal alignment using \A give rise to a number
of research directions: approaching other alignment types, more general
optimization metrics, relaxed optimality guarantees, performance analyses,
improved heuristics, applications outside of biology, and more performant
algorithms and implementations.

\endgroup

\cleardoublepage%

\begingroup
\let\clearpage\relax
\let\cleardoublepage\relax
\let\cleardoublepage\relax

%\begin{otherlanguage}{ngerman}
\pdfbookmark[1]{Zusammenfassung}{Zusammenfassung}
\chapter*{Zusammenfassung}

Beim Sequenzabgleich geht es darum, Ähnlichkeiten zwischen Sequenzen zu finden. Sie ist
Sie ist ein zentraler Baustein der Molekularbiologie, seit vor einem halben Jahrhundert erstmals DNA-, RNA- und Protein
Sequenzen vor einem halben Jahrhundert erstmals gewonnen wurden. Sequenzabgleich wird angewandt
der Forschung und der Medizin, mit Anwendungen in der Evolutionsbiologie, der Genom
Genomzusammensetzung, Onkologie und vielen anderen Bereichen. Die Analyse der wachsenden Mengen an
genomischer Daten erfordert Algorithmen mit hoher Genauigkeit und Geschwindigkeit. Außerdem ist der
Übergang von Einzelgenomreferenzen zu Pangenomen (repräsentativ für eine ganze
für eine ganze Population) motiviert neuartige Algorithmen zum Alignment.

Wir betrachten die beiden Probleme: \emph{semi-global} Alignment eines Satzes von DNA-Reads
an eine Pangenom-Graphenreferenz (die möglicherweise Zyklen enthält), und \emph{global}
(Ende-zu-Ende) Abgleich zweier Sequenzen. Jüngste theoretische Ergebnisse zeigen
dass stark subquadratische Algorithmen für den schlimmsten Fall unwahrscheinlich sind.
Außerdem skalieren die Laufzeit und der Speicherplatz der vorhandenen optimalen Algorithmen
selbst für ähnliche Sequenzen quadratisch. Ein offenes Problem ist die Entwicklung eines
Algorithmus mit linear-ähnlicher empirischer Skalierung für Eingaben zu entwickeln, bei denen die Fehler
linear in $n$ sind. In der vorliegenden Arbeit stellen wir einen Ansatz vor, der darauf abzielt
dieses Problem zu lösen, um praktische optimale Alignment-Algorithmen zu entwickeln.

Moderne optimale Aligner führen eine uninformierte Suche durch - sie vernachlässigen Informationen
aus den noch nicht alignierten Teilen der Sequenzen. Wir nutzen diese Informationen in einem
einem prinzipiellen Rahmen, in dem ein Alignment mit minimalem Editierabstand
einem kürzesten Pfad in einem Alignment-Graphen entspricht, und die verbleibende
Informationen über die Sequenzen werden in einer heuristischen Funktion erfasst, die
die die Länge eines verbleibenden kürzesten Pfades schätzt. Der klassische kürzeste Weg
Algorithmus \A verwendet eine solche Heuristik, um die Suche zu lenken, und er findet dennoch
nachweislich kürzeste Pfade, wenn die Heuristik \emph{zulässig} ist, d.h. er
immer eine untere Schranke für die Edit-Distanz der verbleibenden Suffixe liefert.
Seltsamerweise haben frühere Versuche, \A auf das Multiple Sequence Alignment (MSA) anzuwenden,
nicht einmal für paarweises Alignment zu praktischen Algorithmen geführt. In dieser
vorliegenden Arbeit untersuchen wir, wie dies möglich ist. In der Formulierung des kürzesten Weges,
schlagen wir (i) eine neue, hochinformierte zulässige Heuristik vor, (ii) entwerfen
effiziente Algorithmen und Datenstrukturen für die Berechnung der Heuristik, (iii)
beweisen ihre Optimalität, (iv) implementieren die vorgestellten Algorithmen und (v) vergleichen
ihre Leistung und Skalierung mit anderen optimalen Algorithmen. Unser Ansatz ist
nachweislich optimal nach der Edit-Distanz, seine Laufzeit skaliert empirisch
subquadratisch (und manchmal sogar nahezu linear) mit der Ausgabegröße, was
was zu einer Beschleunigung um Größenordnungen im Vergleich zu
optimalen Algorithmen.

In dieser Arbeit zeigen wir, wie man verschiedene Dimensionen der
der Eingabekomplexität unter Beibehaltung der Geschwindigkeit und Optimalität. Zunächst zeigen wir
demonstrieren wir, wie man einen Trie-Index anwendet, um die Laufzeit des Alignments sublinear
mit der Referenzgröße skaliert. Dann führen wir eine neue Seed-Heuristik für \A ein, die
das Alignment von langen Sequenzen (bis zu 100 Mbp) nahezu linear mit deren
Länge ermöglicht. Schließlich erweitern wir die Seed-Heuristik zu einer allgemeinen Chaining-Seed-Heuristik
die ungenaue Übereinstimmung, Verkettung von Übereinstimmungen und Lückenkosten einschließt, um die
tolerierte Fehlerrate zu erhöhen (auf 30\% für synthetische Daten und 10\% für reale Daten).
Interessanterweise stellt unsere Seed-Heuristik das allgegenwärtige Seed-(Chain)-Extend
Paradigma des Sequenzalignments heraus, das darauf abzielt, lange und hochwertige Seed
Übereinstimmungen. Wir nutzen stattdessen die Information, dass es keine Übereinstimmungen gibt, um
um suboptimale Alignments auszuschließen, was dazu führt, dass die Seeds daraufhin optimiert werden, kurz zu sein und nicht
viele Übereinstimmungen haben.

Diese ersten Schritte zu einem skalierbaren optimalen Alignment mit \A geben Anlass zu einer Reihe von
Forschungsrichtungen: Annäherung an andere Alignment-Typen, allgemeinere
Optimierungsmetriken, entspannte Optimalitätsgarantien, Leistungsanalysen,
verbesserte Heuristiken, Anwendungen außerhalb der Biologie und leistungsfähigere
Algorithmen und Implementierungen.

Sequenzalignment ist der Prozess der Erkennung von Ähnlichkeiten zwischen
Sequenzen. Seit vor einem halben Jahrhundert zum ersten Mal genetische Sequenzen
sequenziert wurden, ment ist eine grundlegende Aufgabe in der Molekularbiologie
mit Anwendungen in der Evolutionsbiologie, Genomassemblierung,
Variationserkennung und andere. Ein laufender Übergang von uns- Die Umstellung
von Genomen auf die Verwendung von Pangenomen motiviert zum Überdenken der
klassischen Ausrichtung Algorithmen. Die Vielfalt der Anwendungen kombiniert mit
der wachsenden Menge an genetische Daten motivieren die Entwicklung schneller
und genauer Ausrichtungsalgorithmen.

Existierende Ausrichtungsalgorithmen sind entweder optimal, aber quadratisch
oder schnell, aber geeignet. nahe. Diese Arbeit prosiert einen eleganten Ansatz
zur Ausrichtung basierend auf dem \A Algorithmus, der sowohl heuristisch schnell
als auch nachweislich optimal ist. Es wurde gezeigt Diese Ausrichtung ist in
stark subquadratischer Zeit im Allgemeinen wahrscheinlich nicht lösbar Fall. Das
Ziel, das wir in dieser Arbeit verfolgen, ist die Anwendung des \A Ansatz als
möglichst viele Arten von Daten und bleibt dabei schnell und optimal. Wir
betrachten zwei Arten von Alignment: semi-global, um einen DNA-Satz zu kartieren
Sequenzen zu einer Pangenom-Referenz; und global zur Berechnung der
Bearbeitungsentfernung

Abstand zwischen zwei Folgen. Um verschiedene Datendimensionen handhaben zu
können, verwenden wir schlagen mehrere Techniken vor und untersuchen empirisch
ihre Laufzeitskalierung: ein Versuch Index ermöglicht sublineare Skalierung mit
der Referenzgröße, Seed-Heuristik ermöglicht nahezu lineare Skalierung mit
Sequenzlänge und ungenaue Seed-Matching und Match-Verkettung Skalierung auf hohe
Fehlerraten ermöglichen. Aufgrund der überlegenen Skalierung des \A sich nähern,
Unsere prototypischen Implementierungen laufen um Größenordnungen schneller als
bestehende optimale Anflüge auch bei langen Fehlsequenzen.

Wir sehen eine Vielzahl zukünftiger Richtungen für die Weiterentwicklung von \A
für Sequenz Ausrichtung, einschließlich anderer Ausrichtungsarten,
Verallgemeinerung der Bearbeitungsentfernungsmetrik, Lockerung der
Optimalitätsgarantie, theoretische Analysen der Leistung und Effizientere
Implementierungen für den Produktionseinsatz.
%\end{otherlanguage}

\endgroup
\vfill

% Problem
%Biological sequences do not generally
%align perfectly due to biological differences and technical errors. Given two
%sequences, the desired alignment is a position-to-position correspondence
%between two sequences which minimizes the edit costs (substitutions, insertions
%or deletions).
%This task is closely related to calculating \emph{edit distance}.

% Existing algorithms
% by using heuristic
%information to speed up the alignment without sacrificing correctness.
%to to making it both heuristically fast and provably optimal. 

%Can we use the \A algorithm to find provably optimal alignment heuristically
%fast?

%Practical alignment algorithms are desired to \textcircled{1} find accurate
%alignments, \textcircled{2} apply to a wide range of data, and \textcircled{3}
%use little time and memory.

% Shortest path formulation
%We consider a principled alignment formulation based on shortest paths and
%demonstrate that the \A shortest path algorithm can be used to outperform
%current methods. Unlike existing methods, \A enables an \emph{informed search}
%based on information from the unaligned sequence suffixes, thus radically

%Runtime, memory usage, scaling. Data variation A good algorithm should be
%optimal, complete, performant (fast and low on memory)

%polynomial speed ups on real data. It \textcircled{1} provides optimality
%guarantees according to edit distance, \textcircled{2} scales to long and noisy
%sequences, and \textcircled{3} scales subquadratically with sequence length.

%To scale to large reference sequences, we extend the graph with a trie index. To
%scale to long queries, we introduce design an admissible \emph{seed heuristic},
%which is provably-optimal also efficient to compute. To scale to high error
%rates, we design  

%improving the empyrical runtime scaling (up to linear) in the average case while
%providing optimality guarantees.

%mapping on pangenomes, graph references, 
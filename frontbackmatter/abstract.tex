%*******************************************************
% Abstract
%*******************************************************
%\renewcommand{\abstractname}{Abstract}
\pdfbookmark[1]{Abstract}{Abstract}
\begingroup
\let\clearpage\relax
\let\cleardoublepage\relax
\let\cleardoublepage\relax

\chapter*{Abstract}

% Context
Sequence alignment is the process of detecting similarities between sequences.
For the last half a century, sequence alignment has been of central importance
for molecular biology since it became possible to sequence DNA. Applications
include evolutionary biology, genome assembly, read mapping, variation
detection. Computational methods are crucial for the correctness of the analyses
of the vast amounts of biological data.

% Problem
This thesis explores two types of alignment: \emph{semi-global} and
\emph{global}. Biological sequences do not generally align perfectly due to
biological differences and technical errors. Given two sequences, the desired
alignment is a position-to-position correspondence between two sequences which
minimizes the edit costs (substitutions, insertions or deletions). This task is
closely related to calculating \emph{edit distance}. Practical alignment
algorithms are desired to \textcircled{1} find accurate alignments,
\textcircled{2} apply to a wide range of data, and \textcircled{3} use little
time and memory.

% Existing algorithms
Existing aligning algorithms are either optimal but quadratic or fast but
appximate. Even though the resulting alignment is linear. This gap has motivated
the development of various faster but approximate algorithms. Moreover,
theoretical results suggest that it is that in general, alignment is not
solvable in strongly subquadratic time.

Can we use the A* algorithm to find provably optimal alignment heuristically
fast?

% Shortest path formulation
We consider a principled alignment formulation based on shortest paths and
demonstrate that the A* shortest path algorithm can be used to outperform
current methods. Unlike existing methods, A* enables an \emph{informed search}
based on information from the unaligned sequence suffixes, thus radically
improving the empyrical runtime scaling (up to linear) in the average case while
providing optimality guarantees. On real data, this approach reaches orders of
magnitude of speedup compared to existing approaches.

% A* context 
An optimal alignment can naturally be represented as a shortest path in an
alignment graph (equivalent to the DP table). In order to find such a shortest
path with minimal exploration, we instantiate the A* algorithm with a novel
problem-specific heuristic function based on the unaligned parts of the
sequences. This additional information is a problem-specific heuristic function
and it heavily determines the efficiency of the search. For any explored state
by A*, this heuristic function should compute a lower bound on the remaining
path length, or more specifically, the minimal cost of edit operations needed to
align the remaining sequences.

% Algorithmic contribution
Seed heuristic. In practice, while achieving polynomial speed ups on real data.
It \textcircled{1} provides optimality guarantees according to edit distance,
\textcircled{2} scales to long and noisy sequences, and \textcircled{3} scales
subquadratically with sequence length. To scale to large reference sequences, we
extend the graph with a trie index. To scale to long queries, we introduce
design an admissible \emph{seed heuristic}, which is provably-optimal also
efficient to compute. To scale to high error rates, we design  

% Tools and Performance
Many tools do not have a well-stated problem they optimize. Prototype tools.

% Future work and limitations
We open domain of research direction on \A for alignment, including approximate
solutions, more general metrics, other alignment types local, production
implementations, theoretical analysis of the scaling and performance. Hopefully
one day multiple sequence alignment (MSA) will also be in reach.

mapping on pangenomes, graph references, 

scaling

%Runtime, memory usage, scaling. Data variation A good algorithm should be
%optimal, complete, performant (fast and low on memory), and reference
%complexity (topology and enthropy).

\endgroup

\cleardoublepage%

\begingroup
\let\clearpage\relax
\let\cleardoublepage\relax
\let\cleardoublepage\relax

%\begin{otherlanguage}{ngerman}
\pdfbookmark[1]{Zusammenfassung}{Zusammenfassung}
\chapter*{Zusammenfassung}

Deutsche Zusammenfassung hier.

%\end{otherlanguage}

\endgroup

\vfill
%*******************************************************
% Abstract
%*******************************************************
%\renewcommand{\abstractname}{Abstract}
\pdfbookmark[1]{Abstract}{Abstract}
\begingroup
\let\clearpage\relax
\let\cleardoublepage\relax
\let\cleardoublepage\relax

\chapter*{Abstract}

% Context
Sequence alignment is the process of detecting similarities between biological
sequences, such as DNA, RNA and proteins. For the last half a century,
variations of this problem have been of central importance for molecular
biology. Alignment is usually a building block for a wide range of applications,
include evolutional biology, genome assembly, read mapping, variation detection
aComputational methods are crucial for the correctness of the analyses and the
practical applicability to the vast amounts of biological data.

% Problem
The usual way to represent the similarities between sequences is by a
base-to-base correspondence between them which that minimizes general edit
distance. Biological sequences do not generally align perfectly due to
biological differences and technical errors. Instead, desired are the best
imperfect alignments according to a similarity metric, such as \emph{edit
distance}.

% Goal
A good alignment algorithm should \textcircled{1} find accurate alignments of
\textcircled{2} a wide range of data, in \textcircled{3} little time and memory.
We consider a principled shortest path formulation of alignment and argue that
the A* algorithm fullfils the desired properties better than current methods. In
particular it can \textcircled{1} provide optimality guarantees according to
edit distance, \textcircled{2} tolerate long and noisy sequences, and
\textcircled{3} scale subquadratically with sequence length.

%(approximate vs probabilistic vs exact/optimal algorithm), Runtime, memory
%usage, scaling.
% Data variation
%A good algorithm should be optimal, complete, performant (fast and low on
%memory) on short and long sequences, various error rates and error models,
%and reference complexity (topology and enthropy).

% Algorithms and Asymptotics
The classical dynamic programming (DP) algorithms run in quadratic time, even
though the resulting alignment is linear. This gap has motivated the development
of various faster but approximate algorithms. Moreover, theoretical results
suggest that it is impossible to solve any reasonable alignment problem in
strongly subquadratic time in the worst case.

In this thesis we argue that the
A* informed search algorithm radically improving the runtime scaling (up to
linear) in the average case while can provide optimality guarantees while. On
real data, this approach reaches orders of magnitude of speedup compared to
existing approaches.

% Contribution overview
In this thesis, we consider the alignment as a shortest path problem that we
approach via A*. The A* algorithm naturally encompasses the existing approaches
but extends them with the possibility for informed search which is capable of
using information from the whole sequences to direct the search. In practice,
while achieving polynomial speed ups on real data. Our A* approach is applicable
to various settings of the alignment problem.

% List of contributions
To scale to large reference sequences, we extend the graph with a
trie index. To scale to long queries, we introduce design an admissible \emph{seed
heuristic}, which is provably-optimal also efficient to compute. To scale to
high error rates, we design  

% A*
Informed search. Admissibility. Seed heuristic.

% Tools and Performance
Many tools do not have a well-stated problem they optimize.

% Future work and limitations
Probabilistic approach.
Focus on the metric.
Extend to MSA, local, affine.
Prorotype implementations
No asymptotics.

\endgroup

\cleardoublepage%

\begingroup
\let\clearpage\relax
\let\cleardoublepage\relax
\let\cleardoublepage\relax

\begin{otherlanguage}{ngerman}
\pdfbookmark[1]{Zusammenfassung}{Zusammenfassung}
\chapter*{Zusammenfassung}

Deutsche Zusammenfassung hier.

\end{otherlanguage}

\endgroup

\vfill
%*******************************************************
% Acknowledgments
%*******************************************************
\pdfbookmark[1]{Acknowledgements}{acknowledgements}

\bigskip

\begingroup
\let\clearpage\relax
\let\cleardoublepage\relax
\let\cleardoublepage\relax
\chapter*{Acknowledgements}

\def\thanks#1{%
\begingroup
\leftskip1em
\noindent #1
\par
\endgroup
}

% family
I am lucky to have such supportive parents as Ivo and Svetla who fully dedicated
themselves to my development, introduced me to their professions of engineering
and medicine, and supported my immense freedom in life, and sow my childhood
dream for science.

% teachers, mentors, summer camps, schools
I am greatful for the opportunities to study, teach and socialize with teachers,
mentors fellow students in numerous schools and Summer camps, most notably the
A\&B informatics school in Shumen~(Bulgaria) by Biserka Yovcheva, Summer
computer school in Russia by Victor Matyukhin, physics preparations for IYPT in
Shumen by Svilen Rusev, Yandex ShAD bioinformatics evening school in Moscow by
Mikhail Gelfand~(mentored by Konstantin Severinov), Algorithmic Biology Lab in
St.Petersburg by Pavel Pevzner~(mentored by Sergey Nurk).

% SRL
I owe my first academic steps the~(then called) Software Reliability Lab in
ETH~Zurich by Martin Vechev who supported my academic freedom for combining
biology and informatics and guided the principled approach to formal
correctness. I am especially thankful to my colleagues Benjamin Bichsel~(Beni)
and Dimitar K. Dimitrov~(Mitko) for their pedantic attitude towards correctness
and consistency, so much needed in the biological sciences.

% other labs
I am grateful to my coauthors Harun Mustafa, André Kahles and Gunnar Rätsch for
directing my \A work towards practical results. I am also greatful to Yun-Tsan
Chang, Desislava Ignatova and Emmanuella Guenova from the Dermatology Clinic in
University of Zurich for collaborating on a risky cross-disciplinary project on
single-cell RNA data from lymphoma patients, despite the lack of publishable
results.

% A* for alignment
I am pleased to see that Ragnar Groot Koerkamp accepted my invitation to join
the research direction of \A for sequence alignment. It is a pleasure for me to
cooperate on a shared passion with a like-minded person. I am also greatful to
Iskren Chernev and Mykola Akulov who dedicated their time and effort to
developing tools and visualizations towards further developing the \A for
alignment.

% soft-skills
I also thank Pavol Bielik for helping me accept certain sides of academia,
Gagandeep Singh for giving me a good example for a future idea of using machine
learning for performance optimization while retaining formal correctness,
Fiorella Meyer for supporting the administrative side of my doctoral program,
and Arshavir Ter-Gabrielyan for his peer support towards preparing this thesis.
I thank my friend Fabio Pak who was with me during all ups and downs during my
PhD, Yavor Milanov who introduced me to his disrespect of art and made the
beautiful thesis cover, Pencho Yordanov who convinced me that new and expensive
data is not the only path towards good science, and my virtual friends who
helped me develop skills for successful collaboration. I am greatful to the
practice of Vipassana meditation for the calm concentration resulting in
imagining the \emph{seed heuristic} that is the base for this thesis. I thank
the anonymous paper reviewers for their valuable feedback, confirming the sanity
of my ideas, and for being accepting to new academicians like me. 

% rest
The countless friends, teachers, students, academics, past scientists and
philosophers, who patiently listened and discussed, shared and taught me their
perspectives, allowing me to better understand and present my research.
Additionally, I acknoledge the impact on me of my virtual friends, fellow
travellers and couchsurfers, hippies, vegans, meditators, paragliding pilots,
wikipedia contributors, psychonauts, and lovers, who shared their passions with
me, hosted me, supported me, and contributed creativity to my environment.

\endgroup

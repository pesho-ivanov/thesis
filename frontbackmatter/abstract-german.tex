Sequenzalignment ist die Aufgabe, Ähnlichkeiten zwischen Sequenzen zu finden.
Seit vor einem halben Jahrhundert die ersten DNA-, RNA- und Proteinsequenzen
gewonnen wurden, ist Sequenzalignment ein zentraler Baustein in der
Molekularbiologie. Seine Anwendungen umfassen Evolutionsbiologie,
Genomassemblierung, Variationserkennung und andere. Die Vielfalt der Anwendungen
und die wachsende Menge genetischer Daten erfordern von den
Alignment-Algorithmen eine hohe Genauigkeit und Geschwindigkeit. Darüber hinaus
motiviert der fortschreitende Übergang von der Verwendung einzelner
Genomreferenzen zur Verwendung von Pangenomen zu einem Überdenken der
klassischen Alignment-Algorithmen.

Wir betrachten zwei Arten von paarweisen Alignment-Problemen: semi-globales
Mapping eines Satzes von DNA-Lesevorgängen auf eine allgemeine Graphreferenz,
die ein Pangenom darstellt, und globales (oder Ende-zu-Ende) Alignment von zwei
Sequenzen. Frühere Algorithmen führen eine uninformierte Suche durch – ihre
Alignmentprozesse vernachlässigen Informationen aus den nicht ausgerichteten
Teilen der Sequenzen. Dadurch steigen ihre Laufzeit und ihr Speicherbedarf
nahezu quadratisch mit der Sequenzlänge, was im allgemeinen Fall ebenfalls eine
grundsätzliche Einschränkung darstellen dürfte. Wir beschleunigen das Alignment
verwandter Sequenzen, indem wir Informationen aus den gesamten Sequenzen nutzen.

A* ist ein klassischer Kürzester-Pfad-Algorithmus, der explizit Heuristiken
nutzt, um die Suche zu lenken und zu beschleunigen, und dennoch Garantien für
das Finden eines kürzesten Pfads bietet. Seltsamerweise wurde A* bisher nicht
für paarweises Sequenzalignment übernommen, und in der vorliegenden Arbeit
untersuchen wir, wie dies zu bewerkstelligen ist. Wir (i) formulieren das
Alignmentproblem als Kürzester-Pfad-Problem, (ii) schlagen eine Heuristik vor,
um die A*-Suche zu steuern, (iii) entwerfen effiziente Algorithmen und
Datenstrukturen zur Berechnung der Heuristik, (iv) beweisen ihre Optimalität,
(v) implementieren die vorgestellten Algorithmen und (vi) vergleichen ihre
Leistung und Skalierung mit anderen optimalen Algorithmen. Unser Ansatz ist
nachweislich optimal in Bezug auf die Bearbeitungsentfernung, seine Laufzeit
skaliert empirisch stark subquadratisch (und manchmal sogar nahezu linear), was
zu einer Beschleunigung um Größenordnungen im Vergleich zu modernen optimalen
Algorithmen führt.

In der gesamten Arbeit demonstrieren wir, wie verschiedene Dimensionen der
Eingabekomplexität erfasst werden können, während die Geschwindigkeit und
Optimalität erhalten bleiben. Zuerst demonstrieren wir, wie man einen Trie-Index
anwendet, um die Mapping-Laufzeit sublinear mit der Referenzgröße zu skalieren.
Dann führen wir eine neuartige Seed-Heuristik für A* ein, die zusätzlich eine
nahezu lineare Ausrichtung langer Sequenzen (bis zu Millionen von Basenpaaren)
mit ihrer Länge ermöglicht. Zuletzt erweitern wir die Seed-Heuristik mit
ungenauem Matching und Match-Chaining, um die tolerierte Fehlerrate zu erhöhen
(bis zu 30 \%). Zusammenfassend zeigen wir, wie man lange und fehlerhafte
Sequenzen optimal ausrichtet.

Der A*-Ansatz zum Sequenzalignment steckt noch in den Kinderschuhen, aber eine
Reihe von Forschungsrichtungen gehen bereits davon aus: andere Alignment-Typen,
allgemeinere Optimierungsmetriken, gelockerte Optimalitätsgarantien,
Leistungsanalysen, verbesserte Heuristiken, Anwendungen außerhalb der Biologie
und leistungsfähigere Algorithmen und Implementierungen. Vielleicht wird der
A*-Ansatz für die Ausrichtung eines Tages ausgereift genug sein, um die Leistung
suboptimaler Algorithmen zu erreichen.

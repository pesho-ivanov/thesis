Beim Sequenzabgleich geht es darum, Ähnlichkeiten zwischen Sequenzen zu finden. Sie ist
Sie ist ein zentraler Baustein der Molekularbiologie, seit vor einem halben Jahrhundert erstmals DNA-, RNA- und Protein
Sequenzen vor einem halben Jahrhundert erstmals gewonnen wurden. Sequenzabgleich wird angewandt
der Forschung und der Medizin, mit Anwendungen in der Evolutionsbiologie, der Genom
Genomzusammensetzung, Onkologie und vielen anderen Bereichen. Die Analyse der wachsenden Mengen an
genomischer Daten erfordert Algorithmen mit hoher Genauigkeit und Geschwindigkeit. Außerdem ist der
Übergang von Einzelgenomreferenzen zu Pangenomen (repräsentativ für eine ganze
für eine ganze Population) motiviert neuartige Algorithmen zum Alignment.

Wir betrachten die beiden Probleme: \emph{semi-global} Alignment eines Satzes von DNA-Reads
an eine Pangenom-Graphenreferenz (die möglicherweise Zyklen enthält), und \emph{global}
(Ende-zu-Ende) Abgleich zweier Sequenzen. Jüngste theoretische Ergebnisse zeigen
dass stark subquadratische Algorithmen für den schlimmsten Fall unwahrscheinlich sind.
Außerdem skalieren die Laufzeit und der Speicherplatz der vorhandenen optimalen Algorithmen
selbst für ähnliche Sequenzen quadratisch. Ein offenes Problem ist die Entwicklung eines
Algorithmus mit linear-ähnlicher empirischer Skalierung für Eingaben zu entwickeln, bei denen die Fehler
linear in $n$ sind. In der vorliegenden Arbeit stellen wir einen Ansatz vor, der darauf abzielt
dieses Problem zu lösen, um praktische optimale Alignment-Algorithmen zu entwickeln.

Moderne optimale Aligner führen eine uninformierte Suche durch - sie vernachlässigen Informationen
aus den noch nicht alignierten Teilen der Sequenzen. Wir nutzen diese Informationen in einem
einem prinzipiellen Rahmen, in dem ein Alignment mit minimalem Editierabstand
einem kürzesten Pfad in einem Alignment-Graphen entspricht, und die verbleibende
Informationen über die Sequenzen werden in einer heuristischen Funktion erfasst, die
die die Länge eines verbleibenden kürzesten Pfades schätzt. Der klassische kürzeste Weg
Algorithmus \A verwendet eine solche Heuristik, um die Suche zu lenken, und er findet dennoch
nachweislich kürzeste Pfade, wenn die Heuristik \emph{zulässig} ist, d.h. er
immer eine untere Schranke für die Edit-Distanz der verbleibenden Suffixe liefert.
Seltsamerweise haben frühere Versuche, \A auf das Multiple Sequence Alignment (MSA) anzuwenden,
nicht einmal für paarweises Alignment zu praktischen Algorithmen geführt. In dieser
vorliegenden Arbeit untersuchen wir, wie dies möglich ist. In der Formulierung des kürzesten Weges,
schlagen wir (i) eine neue, hochinformierte zulässige Heuristik vor, (ii) entwerfen
effiziente Algorithmen und Datenstrukturen für die Berechnung der Heuristik, (iii)
beweisen ihre Optimalität, (iv) implementieren die vorgestellten Algorithmen und (v) vergleichen
ihre Leistung und Skalierung mit anderen optimalen Algorithmen. Unser Ansatz ist
nachweislich optimal nach der Edit-Distanz, seine Laufzeit skaliert empirisch
subquadratisch (und manchmal sogar nahezu linear) mit der Ausgabegröße, was
was zu einer Beschleunigung um Größenordnungen im Vergleich zu
optimalen Algorithmen.

In dieser Arbeit zeigen wir, wie man verschiedene Dimensionen der
der Eingabekomplexität unter Beibehaltung der Geschwindigkeit und Optimalität. Zunächst zeigen wir
demonstrieren wir, wie man einen Trie-Index anwendet, um die Laufzeit des Alignments sublinear
mit der Referenzgröße skaliert. Dann führen wir eine neue Seed-Heuristik für \A ein, die
das Alignment von langen Sequenzen (bis zu 100 Mbp) nahezu linear mit deren
Länge ermöglicht. Schließlich erweitern wir die Seed-Heuristik zu einer allgemeinen Chaining-Seed-Heuristik
die ungenaue Übereinstimmung, Verkettung von Übereinstimmungen und Lückenkosten einschließt, um die
tolerierte Fehlerrate zu erhöhen (auf 30\% für synthetische Daten und 10\% für reale Daten).
Interessanterweise stellt unsere Seed-Heuristik das allgegenwärtige Seed-(Chain)-Extend
Paradigma des Sequenzalignments heraus, das darauf abzielt, lange und hochwertige Seed
Übereinstimmungen. Wir nutzen stattdessen die Information, dass es keine Übereinstimmungen gibt, um
um suboptimale Alignments auszuschließen, was dazu führt, dass die Seeds daraufhin optimiert werden, kurz zu sein und nicht
viele Übereinstimmungen haben.

Diese ersten Schritte zu einem skalierbaren optimalen Alignment mit \A geben Anlass zu einer Reihe von
Forschungsrichtungen: Annäherung an andere Alignment-Typen, allgemeinere
Optimierungsmetriken, entspannte Optimalitätsgarantien, Leistungsanalysen,
verbesserte Heuristiken, Anwendungen außerhalb der Biologie und leistungsfähigere
Algorithmen und Implementierungen.
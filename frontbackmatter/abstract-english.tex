Sequence alignment is the task of finding similarities between sequences. It has
been a central building block in molecular biology since DNA, RNA and protein
sequences were first obtained half a century ago. Sequence alignment is applied
to research and medicine, with applications in evolutionary biology, genome
assembly, oncology and many others. The analyses of the growing amounts of
genomic data require algorithms with high accuracy and speed. Moreover, the
ongoing transition from single genome references to pangenomes (representative
for a whole population) motivates novel alignment algorithms.

We consider the two problems: \emph{semi-global} alignment of a set of DNA reads
to a pangenome graph reference (possibly containing cycles), and \emph{global}
(end-to-end) alignment of two sequences. Recent theoretical results conclude
that strongly subquadratic algorithms are unlikely to exist for the worst case.
Moreover, the runtime and memory of existing optimal algorithms scale
quadratically even for similar sequences. An open problem is to develop an
algorithm with linear-like empirical scaling on inputs where the errors are
linear in $n$. In the present thesis we introduce an approach that aims to solve
this problem in order to develop practical optimal alignment algorithms.

Modern optimal aligners perform uninformed search -- they neglect information
from the not-yet-aligned parts of the sequences. We exploit this information in
a principled framework where an alignment with minimal edit distance is
equivalent to a shortest path in an alignment graph, and the remaining
information about the sequences is captured in a heuristic function that
estimates the length of a remaining shortest path. The classic shortest path
algorithm \A uses such a heuristic to direct the search, and yet it finds
provably shortest paths given that the heuristic is \emph{admissible}, i.e. it
always return a lower bound on the edit distance of the remaining suffixes.
Curiously, previous attempts to apply \A to multiple sequence alignment (MSA),
did not result in practical algorithms even for pairwise alignment. In the
present thesis we investigate how to do that. In the shortest path formulation,
we (i) suggest a novel highly-informed admissible heuristic, (ii) design
efficient algorithms and data structures for computing the heuristic, (iii)
prove their optimality, (iv) implement the presented algorithms, and (v) compare
their performance and scaling to other optimal algorithms. Our approach is
provably optimal according to edit distance, its runtime empirically scales
subquadratically (and sometimes even near-linearly) with the output size, which
translates to orders of magnitude of speedup compared to state-of-the-art
optimal algorithms.

Throughout this thesis, we demonstrate how to encompass various dimensions of
the input complexity while preserving the speed and optimality. First, we
demonstrate how to apply a trie index to scale the alignment runtime sublinearly
with the reference size. Then, we introduce a novel seed heuristic for \A which
enables aligning of long sequences (up to 100 Mbp) near-linearly with their
length. Last, we extend the seed heuristic to a general chaining seed heuristic
that encompasses inexact matching, match chaining, and gap costs to raise the
tolerated error rate (to 30\% for synthetic data and 10\% for real data).
Interestingly, our seed heuristic challenges the omnipresent seed-(chain)-extend
paradigm to sequence alignment which aims to connect long and high-quality seed
matches. We, instead, use the information of the lack of matches to dismiss
suboptimal alignments, which leads to optimizing seeds for being short and not
having many matches.

These first steps to scalable optimal alignment using \A give rise to a number
of research directions: approaching other alignment types, more general
optimization metrics, relaxed optimality guarantees, performance analyses,
improved heuristics, applications outside of biology, and more performant
algorithms and implementations.
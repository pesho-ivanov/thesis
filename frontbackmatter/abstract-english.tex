Sequence alignment is the task of finding similarities between sequences. Since
the first DNA, RNA and protein sequences were obtained half a century ago,
sequence alignment has been a central building block in molecular biology. Its
applications include evolutionary biology, genome assembly, variation detection,
and others. The variety of applications and the growing amount of genetic data
require high accuracy and speed from the alignment algorithms. Moreover, the
ongoing transition from using single genome references to using pangenomes
motivates a rethinking of the classic alignment algorithms.

We consider two types of pairwise alignment problems: semi-global mapping of a
set of DNA reads to a general graph reference representing a pangenome, and
global (or end-to-end) alignment of two sequences. Prior algorithms perform
uninformed search – their aligning processes neglect  information from the
unaligned parts of the sequences. As a result, their runtime and memory
footprint increase near-quadratically in the sequence length, which is also
likely to be a fundamental limitation in the general case. We speed up the
alignment of related sequences exploiting information from the whole sequences.

\A is a classic shortest path algorithm, which explicitly leverages heuristics
to direct and speed up the search, and yet it provides guarantees for finding a
shortest path. Curiously, so far \A has not been adopted for pairwise sequence
alignment, and in the present thesis we investigate how to do that. We (i)
phrase the alignment problem as a shortest path problem, (ii) suggest a
heuristic to direct the \A search, (iii) design efficient algorithms and data
structures for computing the heuristic, (iv) prove their optimality, (v)
implement the presented algorithms, and (vi) compare their performance and
scaling to other optimal algorithms. Our approach is provably optimal according
to edit distance, its runtime empirically scales strongly subquadratic (and even
sometimes near-linearly), which translates to orders of magnitude of speedup
compared to state-of-the-art optimal algorithms.

Throughout the thesis, we demonstrate how to encompass various dimensions of the
input complexity while preserving the speed and optimality. First, we
demonstrate how to apply a trie index to scale the mapping runtime sublinearly
with the reference size. Then, we introduce a novel seed heuristic for \A which
additionally enables aligning of long sequences (up to millions of base pairs)
near-linearly with their length. Last, we extend the seed heuristic with inexact
matching and match chaining to raise the tolerated error rate (up to 30\%). In
summary, we demonstrate how to optimally align long and erroneous sequences.

The \A approach to sequence alignment is in its infancy but a number of research
directions already stem from it: other alignment types, more general
optimization metrics, relaxed optimality guarantees, performance analyses,
improved heuristics, applications outside of biology, and more performant
algorithms and implementations. Perhaps, one day the \A approach to alignment
will mature enough to match the performance of suboptimal algorithms.

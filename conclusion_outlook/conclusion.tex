\section{Conclusion}

%2.5. Relevance and impact
The potential adoption of the \A approach to sequencing (incl. the usage of
AStarix mapper and the A*PA global aligner) may hugely alleviate the
decades-long conflict between alignment optimality and computational costs. This
basic task is an extremely common element in the upstream pipelines and other
high-level tools like assemblers. Possibly, many alignment tools can be
outcompeted (similar to the Parasail aligner) and become outdated. The general
impact of guaranteed optimality is increased trust in the tools by
bioinformaticians, ability to explain any alignment, reduced number of
misalignments, applicability to both linear and graph references. A risky
high-stake problem we plan to explore is the multiple sequence alignment (MSA)
since it has many resemblances to pairwise alignment but suffers from the
combinatorial explosion with the number of sequences. Even with relaxed
optimality-guarantees, a novel approach to this general problem may have a
direct impact on comparative genomics, including protein folding. Our experience
with presenting the \A approach convinced us that it is not a common knowledge
in the computation biology community, even though most conference attendants
with technical background are familiar with the name and the general usage. Our
conviction is that \A is a powerful instrument that has fallen out of focus in
our community but that may prove useful for other algorithmic biology problems
as well. Potentially, it could become a standard topic in the university
bioinformatics courses.

% paper:global
Our graph-based approach to alignment differs considerably from dynamic
programming approaches, mainly because of the ability to use information from the
entire sequences. This additional information enables radically more focused
path-finding at the cost of more complex algorithms.

\paragraph{Limitations}
Our presented method has several limitations:
\begin{enumerate}
  \item \emph{Complex regions trigger quadratic search.} Since it is unlikely
        that edit distance in general can be solved in strongly subquadratic
        time, it is inevitable that there are inputs for which our algorithm
        requires quadratic time.  In particular, regions with high error rate,
        long indels, and too many matches~(\cref{GLOBALsec:limitations}) are
        challenging and trigger quadratic exploration.
  \item \emph{High constant in runtime complexity.} Despite the near-linear
        scaling of the number of expanded states~(\cref{GLOBALsec:expanded}),
        \astarpa only outperforms \edlib and \wfa for sufficiently long sequences
        (~\cref{GLOBALfig:scaling-n}) due to the relatively high computational constant
        that the \A search induces.
  \item \emph{Complex parameter tuning.} The performance of our algorithm
        depends heavily on the sequences to be aligned and the corresponding choice of
        parameters (whether to use chaining, the seed length $k$, and whether to use
        inexact matches $\spot$). The parameter tuning (currently
        very simple~(\cref{GLOBALsec:evals-setup}) may require a more comprehensive
        framework when introducing additional optimizations.
  \item \emph{Real data.} The efficiency of the presented algorithm has high
        variability on real data~(\cref{GLOBALsec:evals-comparison-hg}) due to high
        error rates, long indels, and multiple repeats (demonstrated
        in~\cref{GLOBALfig:limitations}). Further optimizations are needed to align
        complex data.
\end{enumerate}

\paragraph{Future work}
We foresee a multitude of extensions and optimizations that may lead to
efficient global aligning for production usage.

\begin{enumerate}
    \item \emph{Performance.} The practical performance of our \A approach could
        be improved using multiple existing ideas from the alignment domain:
        diagonal transition method, variable seed lengths, overlapping seeds,
        combining heuristics with different seed lengths, gap costs between
        matches in a chain~\citep{ukkonen1985algorithms,wilbur1984context}, more
        aggressive pruning, and better parameter tuning. More efficient
        implementations may be possible by using computational
        domains~\citep{spouge1989speeding}, bit-parallelization~\citep{myers1999fast},
        and SIMD~\citep{marco2021fast}.
    \item \emph{Generalizations.} Our method can be generalized to more
        expressive cost models (non-unit costs, affine costs) and different alignment
        types (semi-global, ends-free, and possibly local alignment).
    \item \emph{Relaxations.} Abandoning the optimality guarantee
        enables various performance optimizations. Another relaxation
        of our algorithm would be to validate the optimality of a given alignment more
        efficiently than finding an optimal alignment from scratch.
    \item \emph{Analysis.} The near-linear scaling behaviour requires a thorough
        theoretical analysis~\citep{medvedev2022limitations}. The fundamental
        question that remains to be answered is: \emph{What sequences and what
        errors can be tolerated while still scaling near-linearly with the
        sequence length?} We expect both theoretical and practical contributions
        to this question.
\end{enumerate}

\section{Conclusion} \label{GLOBALsec:conclusion}

% Summary of the paper
We presented an algorithm with an implementation in \astarpa solving pairwise alignment between
two sequences. The algorithm is based on \A with a \sh, inexact matching, match
chaining, and match pruning, which we proved to find an exact solution according
to edit distance. For random sequences with up to $15\%$ uniform errors, the
runtime of \astarpa scales near-linearly to very long sequences ($10^7\bp$)
and outperforms other exact aligners. We demonstrate that on real ONT reads from
a human genome, \astarpa is faster than other aligners on only a limited portion of the
reads.

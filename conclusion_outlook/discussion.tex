\section{Discussion}

\subsection{Stagnation of the field}
% Why nobody did this already given that there is nothing new

\subsection{Scaling}
% asymptotics

\subsection{Using seeds for optimal alignment}

\emph{Seed-extend} (and its variant \emph{seed-chain-extend} for long reads) is
currently the main paradigm that drives the aligning algorithms.

Usually, approximate aligners seek similar patches between the aligned sequences
with the intuition that they hint towards accurate alignment. These patches can
be found in a variety of ways (hashing, kmers, minimizers, etc.) but they In the They 

\begin{observation}[Seeds without matches]
    To efficiently find an optimal alignment using \A with the seed heuristic,
    seeds are not required to match (even on the resulting alignment).
\end{observation}

Nevertheless, each seed can penalize potential alignment by not more than its
\emph{potential} (\ie the number of plus $1$, for the case of exact matching
with unit costs). Any additional errors will require more states to be expanded.

This is an interesting observation was made by Ragnar while playing with the
seed heursitic. It looks Indeed, of finding a good alignment but to prove that all
alternative alignments are no better, the seed heuristic for \A search does not
really need matches to be efficient.

\subsection{On the optimization metric}
\dictum[Freeman Dyson]{%
   It is better to be wrong than to be vague.}
\vskip 1em
\begin{table}[t]
	\centering
	\ra{0.7}
	\caption[Performance comparison for the seed heuristic
	% (short and long reads; linear and graph references)
	]{Runtime and memory comparison of
	optimal aligners. Simulated Illumina and HiFi reads are aligned to linear
	\textit{E.~coli} and graph MHC references. The runtime of the \seedh is
	expressed as absolute time per aligned kbp, while the other aligners are
	compared to the \seedh at a fold change. Additionally, the fraction of
	explored states is shown for the \seedh and the \prefixh.}
	\label{SEEDtab:illumina table}
	\sffamily

	\renewrobustcmd{\bfseries}{\fontseries{b}\selectfont}
	\renewrobustcmd{\boldmath}{}
	
%\quad&&\color{mygrey}{.0004\%} &\color{mygrey}{99.9982\%} &&\color{mygrey}{99.9989} &\color{mygrey}{99.9984} &\,\color{mygrey}{fraction of explored states} \\
%\quad&&\color{mygrey}{99.96} &\color{mygrey}{99.95} && & & \,\color{mygrey}{\% skipped states}\\

\begin{tabular}{@{}lrrrrrl@{}}
\addlinespace[-\aboverulesep]
\cmidrule[\heavyrulewidth]{1-6}
      & \multicolumn{2}{ c }{\textbf{Illumina reads}}& \phantom{a} & \multicolumn{2}{ c }{\textbf{HiFi reads}} &\\
        \cmidrule{2-3} \cmidrule{5-6}
		\textbf{Tool} & \textit{E.~coli} & MHC && \textit{E.~coli} & MHC &\\
\cmidrule{1-6}

\rowcolor{gray!10} \mbox{\textbf{Seeds heuristic}}\xspace &\bfseries 0.019 &\bfseries 0.041 &&\bfseries 0.001 &\bfseries 0.002 & \cellcolor{white} s/kbp\\
\rowcolor{gray!10} &\color{mygrey}{2.4} &\color{mygrey}{2.6} &&\color{mygrey}{2.4} &\color{mygrey}{1.7} & \cellcolor{white} \color{mygrey}{GB (max used)}\\
%\quad&&\color{mygrey}{1:250k} &\color{mygrey}{1:53k} &&\color{mygrey}{1:91k} &\color{mygrey}{1:61k} &\,\color{mygrey}{explored states} \\
\rowcolor{gray!10} \quad&\color{mygrey}{99.9996} &\color{mygrey}{99.9981} &&\color{mygrey}{99.9989} &\color{mygrey}{99.9984} &\cellcolor{white} \color{mygrey}{\% skipped states} \\

\mbox{Prefix heuristic}\xspace &\numprint{269}x &\numprint{180}x && n/a& n/a& x slowdown\\
&\color{mygrey}{7.7} &\color{mygrey}{9.6} &&\color{mygrey}{>20} &\color{mygrey}{>20} & \\
%\quad&&\color{mygrey}{1:2449} &\color{mygrey}{1:1936} &&\color{mygrey}{n/a} &\color{mygrey}{n/a} & \\
\quad&\color{mygrey}{99.9501} &\color{mygrey}{99.9501} &&\color{mygrey}{n/a} &\color{mygrey}{n/a} & \\

\rowcolor{gray!10} \mbox{\textsc{GraphAligner}}\xspace &\numprint{424}x &\numprint{212}x &&\numprint{118}x &\numprint{64}x & \cellcolor{white} \\
\rowcolor{gray!10} &\color{mygrey}{0.2} &\color{mygrey}{0.2} &&\color{mygrey}{3.6} &\color{mygrey}{3.4} & \cellcolor{white} \\

\mbox{\textsc{Vargas}}\xspace &\numprint{133}x &\numprint{67}x &&\numprint{1413}x &\numprint{705}x & \\
&\color{mygrey}{<0.1}&\color{mygrey}{<0.1}&&\color{mygrey}{7.3} &\color{mygrey}{7.3} & \\
\arrayrulecolor{black!30}

\rowcolor{gray!10} \mbox{\textsc{PaSGAL}}\xspace &\numprint{263}x &\numprint{130}x &&\numprint{1367}x &\numprint{736}x & \cellcolor{white} \\
\rowcolor{gray!10} &\color{mygrey}{0.6} &\color{mygrey}{0.6} &&\color{mygrey}{0.6} &\color{mygrey}{0.6} & \cellcolor{white} \\

%\arrayrulecolor{black}
\cmidrule[\heavyrulewidth]{1-6}
\end{tabular}

\end{table}

\subsection{Q1: Speedup of the seed heuristic}
%
\cref{SEEDtab:illumina table} shows that the \seedh achieves a speedup of at least
60 times compared to all considered aligners, across all regimes of operation:
both Illumina and HiFi reads aligned on \textit{E.~coli} and MHC references.

In the Illumina experiments, the \seedh is given 100k reads, while the other
tools are given 1000 reads. In the HiFi experiments, the \seedh is given reads
that cover the reference 10 times, and the other tools are given reads of
coverage 0.1.

The key reason for the speedup of the \seedh is that on all four experiments, it
skips $\geq 99.99\%$ of the $Nm$ states computed by the DP approaches of
\graphaligner, \pasgal, and \vargas. This fraction accounts for both the
explored states during the \A algorithm, and the number of crumbs added to nodes
during precomputation for each read.

The \prefixh exceeded the available memory on HiFi reads, as it is not designed
for long reads.

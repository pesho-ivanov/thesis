\begin{algorithm}[H]
	\caption{For each seed: find all its exact matches in the reference graph
	and put corresponding crumbs backwards from the match. \todo{make the code not exponential like SeedStarts} \todo{SeedStarts in $O(n^2)$}}\label{alg:crumbs}
	\begin{algorithmic}[1]
		\State \makebox[0.4in][r]{$\RG\colon$} Reference graph \label{lin:reference}
        \State \makebox[0.4in][r]{$l\colon$} Length of each seed
		\State \makebox[0.4in][r]{$C(s)\colon$} Set of nodes with a crumb for seed $s$
        \State \makebox[0.4in][r]{$\maxdel\colon$} Maximum number of deletions\label{lin:d}
		\Comment{defined in \cref{eq:maximum-deletions}}
		\Statex
		\Function{PlaceAllCrumbs}{$q\colon$ Read}
			\State $S \gets \{\, q[kl {:} kl+l] \mid k \in \mathbb{N}, kl+l \leq \lvert q \rvert \}$
            \Comment{Split $q$ into seeds} \label{lin:seeds}
			\ForAll{$s \in S$} \label{lin:seeds-loop}
				\State $C(s) \gets \emptyset$ \label{lin:clear-crumbs}
				\ForAll{$u \in \Call{SeedStarts}{s}$} \label{lin:seed-starts-call}
					\State $i \gets \mli{start}(s)$
					\Comment{Starting position of $s$ in $q$} \label{lin:start}
					\State $\Call{PlaceCrumbsBackwards}{s, u, i + \maxdel}$
					\Comment{Crumbs on nodes before $u$} \label{lin:place-crumbs-backwards-call}
				\EndFor
			\EndFor
		\EndFunction
		\Statex
		\Function{SeedStarts}{$s\colon$ Seed}
			\State $T \gets \RGV$
			\Comment{Start from all nodes}
			\label{lin:match-init}
        	\ForAll{$k \in [0, \dots, l-1 ]$}
			\Comment{Match seed letters forward}
			\label{lin:match-forward-start}
				\State $T \gets \{w \mid v \in T, (v, w, s[k]) \in \RGE\}$
				\Comment{Follow edges matching $s[k]$ (forward)}
			\EndFor \label{lin:match-forward-end}
        		\ForAll{$k \in [l-1, \dots, 0 ]$}
				\Comment{Match seed letters backward}
				\label{lin:match-backward-start}
					\State $T \gets \{u \mid v \in T, (u,v,s[k]) \in \RGE \}$
					\Comment{Follow edges matching $s[k]$ (backward)}
			\EndFor \label{lin:match-backward-end}
			\State $M(s) \gets T$
			\State \Return{$M(s)$}
		\EndFunction
		\Statex
		\Function{PlaceCrumbsBackwards}{$s \colon$Seed, $u \colon $Node, $k \colon $Number of
		edges} \label{lin:place-crumbs-backwards-start}
			\ForAll{$t \in \Call{BackwardsBFS}{u, k}$}
				\Comment Nodes $t$ reaching $u$ within $k$ edges
				\label{lin:backwards-bfs}
				\State $C(s) \gets C(s) \cup \{t\}$
				\Comment Add a crumb for seed $s$ to node $t$ \label{lin:place-crumbs-backwards-end}
			\EndFor
		\EndFunction
	\end{algorithmic}
\end{algorithm}

\subsection{Computing the Seed heuristic} \label{sec:algo}
%
In order to efficiently evaluate the \seedh $h\st{v}{i}$, we must be
able to compute the crumbs that should be associated with each vertex $v \in
\RGV$.
%
To this end, \textsc{PlaceAllCrumbs} in \cref{alg:crumbs} shows an efficient
algorithm for precomputing the crumbs $C(s_j)$.

\para{Place Crumbs}
\crefrange{lin:seeds}{lin:seeds-loop} loop over all seeds $s$.
\cref{lin:clear-crumbs} erases the crumbs placed for seed $s$ during the
alignment of previous reads, thus ensuring that the crumbs of previous
alignments do not affect the current run. Then, \cref{lin:seed-starts-call}
locates all exact matches of $s$ in $\RG$ by calling \textsc{SeedStarts}.
\textsc{SeedStarts} proceeds in two phases.
\crefrange{lin:match-init}{lin:match-forward-end} locate all nodes that can be
the end point of an exact match. Then,
\crefrange{lin:match-backward-start}{lin:match-backward-end} backtrack through
the graph, locating all nodes that can be the starting point of an exact match.
We note that \crefrange{lin:match-forward-start}{lin:match-forward-end} could be
omitted without jeopardizing the correctness of \textsc{SeedStarts}, but they
will be crucial when we introduce our trie optimization in \cref{sec:trie}.

For each node $u$ that can be the start of an exact match of $s$,
\cref{lin:place-crumbs-backwards-call} calls \textsc{PlaceCrumbsBackwards},
which places crumbs on all nodes before $u$, up to a distance of $i{+}d$. Here,
$i$ is the index of $s$ in $q$ (\cref{lin:start}), and $\maxdel$ is the maximum
number of deletions we tolerate (\cref{lin:d}).
%
In order to avoid repeatedly placing crumbs on the same nodes,
\textsc{PlaceCrumbsBackwards} internally uses a breath first search (BFS, see
\cref{lin:backwards-bfs}) starting from $u$ and proceeding through reverse edges
until depth $i + \maxdel$.
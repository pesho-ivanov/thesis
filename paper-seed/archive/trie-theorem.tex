The following theorem is providing an uppper limit on the expected runtime
asymptotics of the \dijkstra algorithm that agrees with the experimental results
but is not tight.

\begin{theorem}[Trie speedup]
    Let a random linear reference $G$ of length $N$ be supplemented with a trie
    $T$ of sufficient depth at least $m$. Then \dijkstra is expected to explore
    $\Oh(N^{0.42})$ states to align a random read $q$ of length $m$.
\end{theorem}
\begin{proof}
    Lets the read $q$ is best aligned to $G$ for cost $c$.

    Let $e(N)$ be the expected number of explored states aligning $q$ to a
    random linear referece of length $N$.

    By induction, we will show that for all $N>N_0$ for some $N_0$, $e(2N) \le
    e(N)*(1+C)$, where $C=1/3$ is an upper limit on the number of new states
    that a kmer from the second part of the $2N$ reference is expected to
    produce.

    %$C_0=2*4^c$ (an upper limit of the
    %number of nodes in a full trie of depth $c$), and
    
    Then the closed form of $e(N) \in \Oh((4/3)^{\log_2 N}) \equiv
    \Oh(N^{\log_2(4/3)}) \equiv \Oh(N^{0.42})$.

    Let $E$ be the subgraph of $G+T$ that includes only the states that
    \dijkstra explores for aligning $q$ for cost $c$. Notice that every path $p$
    from the root to a leaf of $E$ has cost at most $c$ so $E$ contains all
    nodes from $T$ at depth not greater than $c$.

    Assuming that every kmer in the second part of the $2N$ reference reaches a
    leaf in $E$, the following letter has a $1/4$ chance to add the same letter
    that is following in the read; with a $3/4*1/4$ chance to add two following
    letters, and so on. This geometric progression is bound by $\frac{1}{1-1/4} = 4/3$.

    
    %We will show that $e(4N) = e(N)$.
\end{proof}

%\begin{theorem}
%\end{theorem}
%\begin{proof}
%\end{proof}
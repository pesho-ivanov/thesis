The following theorems establish that \seedh is expected to be exponentially
faster than \seedh with increasing read length or error rate.

\begin{definition}[Heuristic potential]
    The \textit{potential} $U(h)$ of a \seedh function $h$ is the maximal
    possible heuristic value it can have. 
\end{definition}

Note that the potential does not depend on the reference but only on the read
seeds. In the general case for exact matches: $U(h) = \min(\sum_{s \in S}
\max(\csubst, \cins, \cdel), d)$. For unit edit costs, constant seed length $l$,
and high enough $d$, $U(h_q) = \lvert S \rvert = \lfloor \lvert q \rvert / l
\rfloor$.

\begin{theorem}[Heuristic speedup]
    Let a random linear reference $G$ of length $N$ be supplemented with a trie
    $T$ of depth $d=\lfloor \log_4(N) \rfloor$. Then \dijkstra is expected to
    explore at least $3^{s_1}$ times more states than \A with \seedh $h$ where
    $s_1=s{(\frac{s-1}{s})}^c$ is the expected number of seeds without errors. $s=\lfloor m/l
    \rfloor$ seeds of length $l=\lfloor \log_4(N) \rfloor$ to align a random
    read $q$ of length $m$ that best aligns with cost $c$. Assuming unit edit
    costs.
\end{theorem}
\begin{proof}
\end{proof}

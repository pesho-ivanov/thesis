% Description: This file enables using \cref to references Figures, Sections,
% Equations, etc.
%
% Usage: % Description: This file enables using \cref to references Figures, Sections,
% Equations, etc.
%
% Usage: % Description: This file enables using \cref to references Figures, Sections,
% Equations, etc.
%
% Usage: % Description: This file enables using \cref to references Figures, Sections,
% Equations, etc.
%
% Usage: \input{headers/lqa/references}

%%%%%%%%%%%%%%
% REFERENCES %
%%%%%%%%%%%%%%

% Package documentation:
% http://ftp.math.purdue.edu/mirrors/ctan.org/macros/latex/contrib/cleveref/cleveref.pdf

% PACKAGE OPTIONS:
%
% capitalize: always capitalize cross-reference names, regardless of where they
% appear in the sentence, writing Theorem 1 and Equation 3 (as opposed to
% theorem 1 and equation 3)
%
% noabbrev:  avoid abbreviations (\eg use Figure instead of Fig.). Note: To
% avoid all abbreviations, you must also check all manually defined reference
% names in this file.

%\usepackage[capitalize]{cleveref}

% required for proper hyperrefs into algorithm lines
% use "renewcommand" instead of "newcommand" if \eg the document class already defines this
\makeatletter
\newcommand\theHALG@line{\thealgorithm.\arabic{ALG@line}}
\makeatother

% OVERRIDE CREF FORMAT:
%
% Override the cref format, \eg, for sections to: §1.2
%
% #1: formatted version of the label counter
%
% #2, #3: beginning and end of the part of the cross-reference that forms the
% hyperlink
%
% Example: Override the cref format for equations to, \eg, Eq.~(1)
%
% \crefformat{equation}{Eq.~(#2#1#3)}

\crefformat{section}{\S#2#1#3}

% referencing sections without labels (\eg, from citations)
\newcommand{\secref}[1]{\S#1}

% OVERRIDE CREF FORMAT FOR RANGES:
%
% override the cref format for ranges of sections, \eg: §1.2 to §1.3
%
% #1, #2: formatted versions of the two label counters defining the reference
% range
%
% #3, #4: denote the beginning and end of the hyperlink for the first reference
%
% #5, #6: denote the beginning and end of the hyperlink for the second reference
\crefrangeformat{section}{\S#3#1#4\crefrangeconjunction\S#5#2#6}

% OVERRIDE CREF FORMAT FOR LISTS:
%
% override the cref format for multiple sections, \eg,:
%
% Argument 1: the cross-reference type
%
% Argument 2: the format for the first cross-reference in a list
%
% Argument 3: the format for the second cross-reference in a list of two
%
% Argument 4: the format for the middle cross-references in a list of more than two
%
% Argument 5: the format for the last cross-reference in a list of more than two
\crefmultiformat{section}{\S#2#1#3}{\crefpairconjunction\S#2#1#3}{\crefmiddleconjunction\S#2#1#3}{\creflastconjunction\S#2#1#3}

% ADAPT CONJUNCTION
%
% Adapt the conjunction used in a reference range, to, \eg: Figs. 1-2
\newcommand{\crefrangeconjunction}{--}

% CUSTOMIZE/ADD REFERENCE NAME
%
% Customize the cross-reference name for a given cross-reference type
%
% Argument 1: the cross-reference type
% 
% Argument 2: singular form of name
%
% Argument 3: plural form of name
% 
% Examples:
% 
% \crefname{section}{Sec.}{Sections}
\crefname{theorem}{Thm.}{Thms.}
\crefname{thm}{Thm.}{Thms.}
% \crefname{lem}{Lem.}{Lemmas}
% \crefname{lstlisting}{Listing}{listings}
% \crefname{algorithm}{Alg.}{Algs.}
% \crefname{example}{Ex.}{Exs.}
% \crefname{table}{Tab.}{Tabs.}
\crefname{listing}{Lst.}{listings}
\crefname{line}{Lin.}{Lin.}
\crefname{appendix}{App.}{App.}

% references without labels (\eg, from citations)
% \newcommand{\thmref}[1]{Thm.~#1}
% \newcommand{\lemref}[1]{Lem.~#1}
% \newcommand{\appref}[1]{App.~#1}
% \newcommand{\algoref}[1]{Alg.~#1}
% \newcommand{\exref}[1]{Ex.~#1}
% \newcommand{\tabref}[1]{Tab.~#1}
% \newcommand{\propref}[1]{Prop.~#1}
% \newcommand{\figref}[1]{Fig.~#1}

%%%%%%%%%%%%
% APPENDIX %
%%%%%%%%%%%%

\newcommand{\appref}[1]{%
	\ifbool{includeappendix}{\cref{SEED#1}}{the appendix}%
}
\newcommand{\Appref}[1]{%
	\ifbool{includeappendix}{\cref{SEED#1}}{The appendix}%
}

%%%%%%%%%%%%%%%%%%%%%%%%%%%
% OPTIONAL CUSTOMIZATIONS %
%%%%%%%%%%%%%%%%%%%%%%%%%%%

% Alias a counter to a different cross-reference type.
%
% Example: Write Fig.~5 for \cref{SEEDsec:abc}
%
% \crefalias{section}{figure}


%%%%%%%%%%%%%%
% REFERENCES %
%%%%%%%%%%%%%%

% Package documentation:
% http://ftp.math.purdue.edu/mirrors/ctan.org/macros/latex/contrib/cleveref/cleveref.pdf

% PACKAGE OPTIONS:
%
% capitalize: always capitalize cross-reference names, regardless of where they
% appear in the sentence, writing Theorem 1 and Equation 3 (as opposed to
% theorem 1 and equation 3)
%
% noabbrev:  avoid abbreviations (\eg use Figure instead of Fig.). Note: To
% avoid all abbreviations, you must also check all manually defined reference
% names in this file.

%\usepackage[capitalize]{cleveref}

% required for proper hyperrefs into algorithm lines
% use "renewcommand" instead of "newcommand" if \eg the document class already defines this
\makeatletter
\newcommand\theHALG@line{\thealgorithm.\arabic{ALG@line}}
\makeatother

% OVERRIDE CREF FORMAT:
%
% Override the cref format, \eg, for sections to: §1.2
%
% #1: formatted version of the label counter
%
% #2, #3: beginning and end of the part of the cross-reference that forms the
% hyperlink
%
% Example: Override the cref format for equations to, \eg, Eq.~(1)
%
% \crefformat{equation}{Eq.~(#2#1#3)}

\crefformat{section}{\S#2#1#3}

% referencing sections without labels (\eg, from citations)
\newcommand{\secref}[1]{\S#1}

% OVERRIDE CREF FORMAT FOR RANGES:
%
% override the cref format for ranges of sections, \eg: §1.2 to §1.3
%
% #1, #2: formatted versions of the two label counters defining the reference
% range
%
% #3, #4: denote the beginning and end of the hyperlink for the first reference
%
% #5, #6: denote the beginning and end of the hyperlink for the second reference
\crefrangeformat{section}{\S#3#1#4\crefrangeconjunction\S#5#2#6}

% OVERRIDE CREF FORMAT FOR LISTS:
%
% override the cref format for multiple sections, \eg,:
%
% Argument 1: the cross-reference type
%
% Argument 2: the format for the first cross-reference in a list
%
% Argument 3: the format for the second cross-reference in a list of two
%
% Argument 4: the format for the middle cross-references in a list of more than two
%
% Argument 5: the format for the last cross-reference in a list of more than two
\crefmultiformat{section}{\S#2#1#3}{\crefpairconjunction\S#2#1#3}{\crefmiddleconjunction\S#2#1#3}{\creflastconjunction\S#2#1#3}

% ADAPT CONJUNCTION
%
% Adapt the conjunction used in a reference range, to, \eg: Figs. 1-2
\newcommand{\crefrangeconjunction}{--}

% CUSTOMIZE/ADD REFERENCE NAME
%
% Customize the cross-reference name for a given cross-reference type
%
% Argument 1: the cross-reference type
% 
% Argument 2: singular form of name
%
% Argument 3: plural form of name
% 
% Examples:
% 
% \crefname{section}{Sec.}{Sections}
\crefname{theorem}{Thm.}{Thms.}
\crefname{thm}{Thm.}{Thms.}
% \crefname{lem}{Lem.}{Lemmas}
% \crefname{lstlisting}{Listing}{listings}
% \crefname{algorithm}{Alg.}{Algs.}
% \crefname{example}{Ex.}{Exs.}
% \crefname{table}{Tab.}{Tabs.}
\crefname{listing}{Lst.}{listings}
\crefname{line}{Lin.}{Lin.}
\crefname{appendix}{App.}{App.}

% references without labels (\eg, from citations)
% \newcommand{\thmref}[1]{Thm.~#1}
% \newcommand{\lemref}[1]{Lem.~#1}
% \newcommand{\appref}[1]{App.~#1}
% \newcommand{\algoref}[1]{Alg.~#1}
% \newcommand{\exref}[1]{Ex.~#1}
% \newcommand{\tabref}[1]{Tab.~#1}
% \newcommand{\propref}[1]{Prop.~#1}
% \newcommand{\figref}[1]{Fig.~#1}

%%%%%%%%%%%%
% APPENDIX %
%%%%%%%%%%%%

\newcommand{\appref}[1]{%
	\ifbool{includeappendix}{\cref{SEED#1}}{the appendix}%
}
\newcommand{\Appref}[1]{%
	\ifbool{includeappendix}{\cref{SEED#1}}{The appendix}%
}

%%%%%%%%%%%%%%%%%%%%%%%%%%%
% OPTIONAL CUSTOMIZATIONS %
%%%%%%%%%%%%%%%%%%%%%%%%%%%

% Alias a counter to a different cross-reference type.
%
% Example: Write Fig.~5 for \cref{SEEDsec:abc}
%
% \crefalias{section}{figure}


%%%%%%%%%%%%%%
% REFERENCES %
%%%%%%%%%%%%%%

% Package documentation:
% http://ftp.math.purdue.edu/mirrors/ctan.org/macros/latex/contrib/cleveref/cleveref.pdf

% PACKAGE OPTIONS:
%
% capitalize: always capitalize cross-reference names, regardless of where they
% appear in the sentence, writing Theorem 1 and Equation 3 (as opposed to
% theorem 1 and equation 3)
%
% noabbrev:  avoid abbreviations (\eg use Figure instead of Fig.). Note: To
% avoid all abbreviations, you must also check all manually defined reference
% names in this file.

%\usepackage[capitalize]{cleveref}

% required for proper hyperrefs into algorithm lines
% use "renewcommand" instead of "newcommand" if \eg the document class already defines this
\makeatletter
\newcommand\theHALG@line{\thealgorithm.\arabic{ALG@line}}
\makeatother

% OVERRIDE CREF FORMAT:
%
% Override the cref format, \eg, for sections to: §1.2
%
% #1: formatted version of the label counter
%
% #2, #3: beginning and end of the part of the cross-reference that forms the
% hyperlink
%
% Example: Override the cref format for equations to, \eg, Eq.~(1)
%
% \crefformat{equation}{Eq.~(#2#1#3)}

\crefformat{section}{\S#2#1#3}

% referencing sections without labels (\eg, from citations)
\newcommand{\secref}[1]{\S#1}

% OVERRIDE CREF FORMAT FOR RANGES:
%
% override the cref format for ranges of sections, \eg: §1.2 to §1.3
%
% #1, #2: formatted versions of the two label counters defining the reference
% range
%
% #3, #4: denote the beginning and end of the hyperlink for the first reference
%
% #5, #6: denote the beginning and end of the hyperlink for the second reference
\crefrangeformat{section}{\S#3#1#4\crefrangeconjunction\S#5#2#6}

% OVERRIDE CREF FORMAT FOR LISTS:
%
% override the cref format for multiple sections, \eg,:
%
% Argument 1: the cross-reference type
%
% Argument 2: the format for the first cross-reference in a list
%
% Argument 3: the format for the second cross-reference in a list of two
%
% Argument 4: the format for the middle cross-references in a list of more than two
%
% Argument 5: the format for the last cross-reference in a list of more than two
\crefmultiformat{section}{\S#2#1#3}{\crefpairconjunction\S#2#1#3}{\crefmiddleconjunction\S#2#1#3}{\creflastconjunction\S#2#1#3}

% ADAPT CONJUNCTION
%
% Adapt the conjunction used in a reference range, to, \eg: Figs. 1-2
\newcommand{\crefrangeconjunction}{--}

% CUSTOMIZE/ADD REFERENCE NAME
%
% Customize the cross-reference name for a given cross-reference type
%
% Argument 1: the cross-reference type
% 
% Argument 2: singular form of name
%
% Argument 3: plural form of name
% 
% Examples:
% 
% \crefname{section}{Sec.}{Sections}
\crefname{theorem}{Thm.}{Thms.}
\crefname{thm}{Thm.}{Thms.}
% \crefname{lem}{Lem.}{Lemmas}
% \crefname{lstlisting}{Listing}{listings}
% \crefname{algorithm}{Alg.}{Algs.}
% \crefname{example}{Ex.}{Exs.}
% \crefname{table}{Tab.}{Tabs.}
\crefname{listing}{Lst.}{listings}
\crefname{line}{Lin.}{Lin.}
\crefname{appendix}{App.}{App.}

% references without labels (\eg, from citations)
% \newcommand{\thmref}[1]{Thm.~#1}
% \newcommand{\lemref}[1]{Lem.~#1}
% \newcommand{\appref}[1]{App.~#1}
% \newcommand{\algoref}[1]{Alg.~#1}
% \newcommand{\exref}[1]{Ex.~#1}
% \newcommand{\tabref}[1]{Tab.~#1}
% \newcommand{\propref}[1]{Prop.~#1}
% \newcommand{\figref}[1]{Fig.~#1}

%%%%%%%%%%%%
% APPENDIX %
%%%%%%%%%%%%

\newcommand{\appref}[1]{%
	\ifbool{includeappendix}{\cref{SEED#1}}{the appendix}%
}
\newcommand{\Appref}[1]{%
	\ifbool{includeappendix}{\cref{SEED#1}}{The appendix}%
}

%%%%%%%%%%%%%%%%%%%%%%%%%%%
% OPTIONAL CUSTOMIZATIONS %
%%%%%%%%%%%%%%%%%%%%%%%%%%%

% Alias a counter to a different cross-reference type.
%
% Example: Write Fig.~5 for \cref{SEEDsec:abc}
%
% \crefalias{section}{figure}


%%%%%%%%%%%%%%
% REFERENCES %
%%%%%%%%%%%%%%

% Package documentation:
% http://ftp.math.purdue.edu/mirrors/ctan.org/macros/latex/contrib/cleveref/cleveref.pdf

% PACKAGE OPTIONS:
%
% capitalize: always capitalize cross-reference names, regardless of where they
% appear in the sentence, writing Theorem 1 and Equation 3 (as opposed to
% theorem 1 and equation 3)
%
% noabbrev:  avoid abbreviations (\eg use Figure instead of Fig.). Note: To
% avoid all abbreviations, you must also check all manually defined reference
% names in this file.

%\usepackage[capitalize]{cleveref}

% required for proper hyperrefs into algorithm lines
% use "renewcommand" instead of "newcommand" if \eg the document class already defines this
\makeatletter
\newcommand\theHALG@line{\thealgorithm.\arabic{ALG@line}}
\makeatother

% OVERRIDE CREF FORMAT:
%
% Override the cref format, \eg, for sections to: §1.2
%
% #1: formatted version of the label counter
%
% #2, #3: beginning and end of the part of the cross-reference that forms the
% hyperlink
%
% Example: Override the cref format for equations to, \eg, Eq.~(1)
%
% \crefformat{equation}{Eq.~(#2#1#3)}

\crefformat{section}{\S#2#1#3}

% referencing sections without labels (\eg, from citations)
\newcommand{\secref}[1]{\S#1}

% OVERRIDE CREF FORMAT FOR RANGES:
%
% override the cref format for ranges of sections, \eg: §1.2 to §1.3
%
% #1, #2: formatted versions of the two label counters defining the reference
% range
%
% #3, #4: denote the beginning and end of the hyperlink for the first reference
%
% #5, #6: denote the beginning and end of the hyperlink for the second reference
\crefrangeformat{section}{\S#3#1#4\crefrangeconjunction\S#5#2#6}

% OVERRIDE CREF FORMAT FOR LISTS:
%
% override the cref format for multiple sections, \eg,:
%
% Argument 1: the cross-reference type
%
% Argument 2: the format for the first cross-reference in a list
%
% Argument 3: the format for the second cross-reference in a list of two
%
% Argument 4: the format for the middle cross-references in a list of more than two
%
% Argument 5: the format for the last cross-reference in a list of more than two
\crefmultiformat{section}{\S#2#1#3}{\crefpairconjunction\S#2#1#3}{\crefmiddleconjunction\S#2#1#3}{\creflastconjunction\S#2#1#3}

% ADAPT CONJUNCTION
%
% Adapt the conjunction used in a reference range, to, \eg: Figs. 1-2
\newcommand{\crefrangeconjunction}{--}

% CUSTOMIZE/ADD REFERENCE NAME
%
% Customize the cross-reference name for a given cross-reference type
%
% Argument 1: the cross-reference type
% 
% Argument 2: singular form of name
%
% Argument 3: plural form of name
% 
% Examples:
% 
% \crefname{section}{Sec.}{Sections}
\crefname{theorem}{Thm.}{Thms.}
\crefname{thm}{Thm.}{Thms.}
% \crefname{lem}{Lem.}{Lemmas}
% \crefname{lstlisting}{Listing}{listings}
% \crefname{algorithm}{Alg.}{Algs.}
% \crefname{example}{Ex.}{Exs.}
% \crefname{table}{Tab.}{Tabs.}
\crefname{listing}{Lst.}{listings}
\crefname{line}{Lin.}{Lin.}
\crefname{appendix}{App.}{App.}

% references without labels (\eg, from citations)
% \newcommand{\thmref}[1]{Thm.~#1}
% \newcommand{\lemref}[1]{Lem.~#1}
% \newcommand{\appref}[1]{App.~#1}
% \newcommand{\algoref}[1]{Alg.~#1}
% \newcommand{\exref}[1]{Ex.~#1}
% \newcommand{\tabref}[1]{Tab.~#1}
% \newcommand{\propref}[1]{Prop.~#1}
% \newcommand{\figref}[1]{Fig.~#1}

%%%%%%%%%%%%
% APPENDIX %
%%%%%%%%%%%%

\newcommand{\appref}[1]{%
	\ifbool{includeappendix}{\cref{SEED#1}}{the appendix}%
}
\newcommand{\Appref}[1]{%
	\ifbool{includeappendix}{\cref{SEED#1}}{The appendix}%
}

%%%%%%%%%%%%%%%%%%%%%%%%%%%
% OPTIONAL CUSTOMIZATIONS %
%%%%%%%%%%%%%%%%%%%%%%%%%%%

% Alias a counter to a different cross-reference type.
%
% Example: Write Fig.~5 for \cref{SEEDsec:abc}
%
% \crefalias{section}{figure}

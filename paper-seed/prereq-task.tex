\subsection{Problem statement: Alignment as shortest path} \label{sec:task}
%
In the following, we formalize the task of optimally aligning a read to a
reference graph in terms of finding a shortest path in an \emph{alignment
graph}. Our discussion closely follows~\citep{ivanov2020astarix} and is in line
with~\citep{rautiainen_aligning_2017}.

\para{Reference graph}
%
A reference graph $\RG=(\RGV,\RGE)$ encodes a collection of references to be
considered when aligning a read. Its directed edges $\RGE \subseteq \RGV \times
\RGV \times \Sigma$ are labeled by nucleotide letters from $\Sigma =
\{\texttt{A},\texttt{C},\texttt{G},\texttt{T}\}$, hence any walk
$\reference{\pi}$ in $\RG$ spells a string $\sigma(\reference{\pi}) \in
\Sigma^*$.

An alignment of a read $q \in \Sigma^*$ to a reference graph $\RG$ consists of
(i)~a walk $\reference{\pi}$ in $\RG$ and (ii)~a sequence of edits (matches,
substitutions, deletions, and insertions) that transform
$\sigma(\reference{\pi})$ to $q$. An alignment is \emph{optimal} if it minimizes
the sum of edit costs for a given real-valued cost model $\cedits = (\cmatch,
\csubst,\cdel, \cins)$.
%
Throughout this work, we assume that edit costs are non-negative---a
pre-requisite for the correctness of \A. Further, we assume that $\cmatch \leq
\csubst, \cins, \cdel$---a prerequisite for the correctness of our heuristic.

We note that our approach naturally works for cyclic reference graphs.

\begin{figure}[t]
	\begin{alignat*}{20}
		(
			&\langle&& u &,& i   &\rangle&,
			&\langle&  v &,& i+1 &\rangle&,
			&&q[i],
			&&\cmatch
		&)&\in \AGE
		&& \quad \text{ if } (u,v,\ell) \in \RGE, \ell = q[i] & \qquad \text{(match)}\\
		%
		(
			&\langle&& u &,& i   &\rangle&,
			&\langle&  v &,& i+1 &\rangle&,
			&&q[i],
			&&\csubst
		&) &\in \AGE
		&& \quad \text{ if } (u,v,\ell) \in \RGE, \ell \neq q[i] & \qquad \text{(substitution)}\\
		%
		(
			&\langle&& u &,& i &\rangle&,
			&\langle&  v &,& i &\rangle&,
			&&\varepsilon,
			&&\cdel
		&) &\in \AGE
		&& \quad \text{ if } (u,v,\ell) \in \RGE & \qquad \text{(deletion)}\\
		%
		(
			&\langle&& u &,& i   &\rangle&,
			&\langle&  u &,& i+1 &\rangle&,
			&&q[i],
			&&\cins
		&) &\in \AGE
		&& \quad & \qquad \text{(insertion)},
	\end{alignat*}
	\caption{Formal definition of alignment graph edges $\AGE \subseteq \AGV[q]
	\times \AGV[q] \times \Sigma_\varepsilon \times \mathbb{R}_{\geq 0}$. Here,
	$u,v \in \RGV$, $0 \leq i < |q|$, $\ell \in \Sigma$, and $\varepsilon$
	represents the empty string, indicating that letter $\ell$ was deleted.}
	\label{fig:graph-edges}
\end{figure}

\para{Alignment graph}
%
In order to formalize optimal alignment as a shortest path finding problem, we
rely on an \emph{alignment graph} $\AG[q]=(\AGV[q],\AGE[q])$.
%
Its nodes $\AGV[q]$ are \emph{states} of the form $\langle v, i \rangle$, where
$v \in \RGV$ is a node in the reference graph and $i \in \{0, \dots, |q|\}$
corresponds to a position in the read $q$.
%
Its edges $\AGE[q]$ are selected such that any path $\alignment{}{\pi}$ in
$\AG[q]$ from $\langle u, 0 \rangle$ to $\langle v, i \rangle$ corresponds to an
alignment of the first $i$ letters of $q$ to $\RG$.
%
Further, the edges are weighted, which allows us to define an \emph{optimal
alignment} of a read $q \in \Sigma^*$ as a shortest path $\alignment{}{\pi}$ in
$\AG[q]$ from $\langle u, 0 \rangle$ to $\langle v, |q| \rangle$, for any $u, v
\in \RGV$.
%
\cref{fig:graph-edges} formally defines the edges $\AGE$.


\subsection{Setting}
%
All experiments were executed on a commodity machine with an Intel Core i7-6700
CPU @ 3.40GHz processor, using a memory limit of 20 GB and a single thread.
%
We note that while multiple tools support parallelization when aligning a single
read, all tools can be trivially parallelized to align multiple reads in
parallel.

\para{Compared aligners}
%
We compare the novel \seedh to \prefixh (both heuristics are implemented in
\astarix), \graphaligner, \pasgal, and \vargas. We provide the versions and
commands used to run all aligners and read simulators in~\cref{sec:commands}. 
%
We note that we do not compare to \vg~\citep{garrison_variation_2018} and
\hga~\citep{feng2021accelerating} since the optimal alignment functionality of
\vg is meant for debugging purposes and has been shown to be inferior to other
aligners~\citep[Tab.~4]{feng2021accelerating}, and \hga makes use of
parallelization and a GPU but has been shown to be superseded in the single CPU
regime~\citep[Fig.~9]{feng2021accelerating}. \pasgal and \vargas are compiled
with AVX2 support. We execute the \prefixh with the default lookup depth $d=5$.

\para{Seed heuristic parameters}
%
The choice of parameters for the \seedh influences its performance.
%
Increasing the trie depth increases its number of nodes, but decreases the
average out-degree of its leaves. We set the trie depth for all experiments to
$D=14 \approx \log_4 N$.

Shorter seeds are more likely to be matched perfectly by an optimal alignment,
as they contain less letters that could be subject to edits. Thus, shorter seeds
can tolerate higher error rate, but at the cost of slower precomputation due to
a higher total number of matches, and slower alignment due to more off-track
matches.
%
In our experiments, we use seed lengths of $k=25$ for Illumina reads and $k=150$
for HiFi reads.

\para{Data}
%
We aligned reads to two different reference graphs: a linear \textit{E.~coli}
genome (strain: K-12 substr. MG1655, ASM584v2)~\citep{howe2019ensembl}, with
length of \numprint{4641652}bp (approx. 4.7Mbp), and a variant graph with the
Major Histocompatibility Complex (MHC), of size \numprint{5318019}bp (approx.
5Mbp), taken from the evaluations of \pasgal~\citep{jain_accelerating_2019}.
%\footnote{\url{\url{https://alurulab.cc.gatech.edu/sites/all/projectFiles/PaSGAL_Chirag}}}
%
Additionally, we extracted a path from MHC in order to create a linear reference
MHC-linear of length \numprint{4956648}bp which covers approx. 93\% of the
original graph. Because of input graph format restrictions, we execute
\graphaligner, \vargas and \pasgal only on linear references in FASTA format
(\textit{E.~coli} and the MHC-linear), while we execute the \seedh and the
\prefixh on the original references (\textit{E.~coli} and MHC). This yields an
underestimation of the speedup of the \seedh, as we expect the performance on
MHC-linear to be strictly better than on the whole MHC graph.

To generate both short Illumina and long HiFi reads, we relied on two tools. We
generated short single-end 200bp Illumina MSv3 reads using \art
simulator~\citep{huang2011art}. We generated long HiFi reads using the script
\randomreads\footnote{\url{https://github.com/BioInfoTools/BBMap/blob/master/sh/randomreads.sh}}
with sequencing lengths \mbox{5--25kbp} and error rates 0.3\%, which are typical
for HiFi reads.

\para{Edit costs}
We execute \astarix with edit costs typical to the corresponding sequencing
technology: $\cedits=(0,1,5,5)$ for Illumina reads and $\cedits=(0,1,1,1)$ for
HiFi reads. As the performance of DP-based tools is independent of edit costs,
we are using the respective default edits costs when executing \graphaligner,
\pasgal and \vargas.


\para{Metrics}
%
We report the performance of aligners in terms of runtime per one thousand
aligned base pairs~[s/kbp]. Since we measured runtime end-to-end (including
loading reference graphs and reads from disk, and building the trie index for
\astarix), we ensured that alignment time dominates the total runtime by
providing sufficiently many reads to each aligner. In order to prevent excessive
runtimes for slower tools, we used a different number of reads for each tool
and explicitly report them for each experiment.

Since shortest path approaches skip considerable parts of the computation
performed by aligners based on dynamic programming, the commonly used Giga Cell
Updates Per Second (GCUPS) metric is not adequate for measuring performance in
our context. 

We measured used memory by \texttt{max\_rss} (Maximum Resident Set Size) from
Snakemake\footnote{\url{https://snakemake.readthedocs.io/en/stable/}}.

We do not report accuracy or number of unaligned reads, as all evaluated
tools align all reads with guaranteed optimality according to edit distance. 
%
We note that \vargas reports a warning that some of its alignments are not
optimal---we ignore this warning and focus on its performance.

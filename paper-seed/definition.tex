\subsection{Formal definition} \label{sec:definition}
%
Next, we formally define the \seedh function $h\st{v}{i}$. Overall, we want to
ensure that $h\st{v}{i}$ is admissible, \ie, that it is a lower bound on the
cost of a shortest path from $\st{v}{i}$ to some $\st{w}{|q|}$ in $\AG$.

\para{Seeds}
%
We split read $q \in \Sigma^*$ into a set $\seeds$ of non-overlapping seeds
$s_0, \dots, s_{|\seeds|-1} \in \Sigma^*$.
%
For simplicity, in this work we ensure that all seeds have the same length and
are consecutive, \ie, we split $q$ into substrings $s_0 \cdot s_1 \cdots
s_{|\seeds|-1} \cdot t$, where all $s_j$ are seeds of length $k$ and we ignore
the suffix $t$ of $q$, which is shorter than~$k$.
%
We note that our approach could be trivially generalized to seeds of different
lengths or non-consecutive seeds as long as they do not overlap. An interesting
future work item is investigating how different choices of seeds affect the
performance of our approach, and selecting seeds accordingly.

\para{Matches}
%
For each seed $s \in \seeds$, we locate all nodes $u \in M(s)$ in the reference
graph that can be the start of an exact match of $s$:
%
\begin{align*}
	M(s) := \{u \in \RGV \mid \exists \text{walk } \pi \text{ starting from } u \in \RG \text{ and spelling } \sigma(\pi) = s \}.
\end{align*}
%
To compute $M(s)$ efficiently, we leverage the trie introduced in
\cref{sec:trie}.

\para{Crumbs}
For seed $s_j$ starting at position $i$ in $q$, we place crumbs on all nodes $u
\in \RGV$ which can reach a node $v \in M(s_j)$ using less than $i+\maxdel$
edges:
%
\begin{align*}
	C(s) := \{u \in \RGV \mid & \exists v \in M(s)\colon \mli{dist}(u, v) < i + \maxdel \},\\
	\text{where }\mli{dist}(u, v) &\text{ is the length of a shortest walk from } u \text{ to } v.
\end{align*}

Later in this section, we will select $\maxdel$ to ensure that if an alignment
uses more than $\maxdel$ deletions, its cost must be so high that the heuristic
function is trivially admissible.

To compute $C(s)$ efficiently, we can traverse the reference graph backwards
from each $v \in M(s)$ by a backward breadth-first-search (BFS).

% Crumbs should be placed before each exact match so that any alignment that may
% reach the match sees a crumb. For each seed we calculate the maximal distance
% to a match that has to be covered by crumbs. More specifically, we compute
% this distance as $i{+}d$, where $i$ is the position of the seed in the read
% (seeds towards the end need more crumbs so that the beginning of the alignment
% can also see them), and $d$ is the maximal number of deletions (since a
% deletion in the read may require a longer alignment in the reference grap)
% after which the \seedh is anyway capped.\todo{explain better} Note that
% putting crumbs in all nodes that can potentially lead to an exact match is
% crucial for the optimality guarantees of the \seedh, while putting unnecessary
% crumbs may only make the heuristic less precise so more state may be explored.
% So in~\cref{the:admissible} we prove that all necessary crumbs are placed and
% the heuristic is admissible.

\para{Heuristic}
%
Let $\seeds_{\ge i}$ be the set of seeds that start at or after position $i$ of
the read, formally defined by $\seeds_{\ge i} := \{s_{j} \mid \lceil i / k
\rceil \leq j < |\seeds|  \}$.
%
This allows us to define the number of expected but missing crumbs in state
$\st{v}{i}$ as $\mli{misses}\st{v}{i} := \big\lvert \left\{  s
\in \seeds_{\ge i} \mid v \notin C(s) \right \} \big\rvert$.
%
Finally, we define the \seedh as
\begin{align} \label{eq:heuristic}
	h\st{v}{i} &= (\lvert q \rvert - i) \cdot \cmatch + \mli{misses}\st{v}{i} \cdot \delta_\mli{min},\\
	\text{for } &\delta_\mli{min} = \min(\csubst - \cmatch,\cdel,\cins - \cmatch), \label{eq:delta}
\end{align}
%
Intuitively, \cref{eq:heuristic} reflects that the cost of aligning each
remaining letter from $q[i{:}]$ is at least $\cmatch$. In addition, every
inexact alignment of a seed induces an additional cost of at least
$\delta_\mli{min}$. 
%
Specifically, every substitution costs $\csubst$ but requires one less match;
every deletion costs $\cdel$; and every insertion costs $\cins$ but also
requires one less match.

We note that $h\st{v}{i}$ implicitly also depends on the reference graph $\RG$,
the read $q$, the set of seeds, and the edit costs $\cedits$.

In order for an alignment with at least $\maxdel$ deletions to have a cost so
high that the heuristic function is trivially admissible, we ensure $\maxdel
\cdot \cdel \geq h\st{v}{i}$ by defining
\begin{align} \label{eq:maximum-deletions}
	\maxdel := \left\lceil \frac{\lvert q \rvert \cdot \cmatch + |\seeds| \cdot \delta_{\mli{min}}}{\cdel} \right\rceil.
\end{align}
%
In \cref{the:admissible}, we show that $h\st{v}{i}$ is admissible, ensuring that
our heuristic yields optimal alignments.


\begin{restatable}[Admissibility]{thm}{admissibility}
	\label{the:admissible}
	The \seedh $h\st{v}{i}$ is admissible.
\end{restatable}
\begin{proof}
	We provide a proof for \cref{the:admissible} in \cref{app:proof}.
	%
	\qedwhite
\end{proof}

\section{Introduction}

% General: aligning, edit distance
Alignment of reads to a reference genome is an essential and early step in most
bioinformatics pipelines. While linear references have been used traditionally,
an increasing interest is directed towards graph references capable of
representing biological variation~\citep{garrison_variation_2018}.
%
Specifically, a \emph{sequence-to-graph} alignment is a base-to-base
correspondence between a given read and a walk in the graph. As sequencing
errors and biological variation result in inexact read alignments, edit distance
is the most common metric that alignment algorithms optimize in order to find
the most probable read origin in the reference.

% We note that in contrast to linear references, reference graphs capture
% genomic variation and therefore enable more accurate
% alignments~\citep{garrison_variation_2018}.

\para{Suboptimal alignment}
%
In the last decades, approximate and alignment-free methods satisfied the demand
for faster algorithms which process huge volumes of genetic
data~\citep{kucherov2019evolution}. 
%
\emph{Seed-and-extend} is arguably the most popular paradigm in read
alignment~\citep{altschul_basic_1990,langmead_fast_2012,li_fast_2009}. First,
substrings (called \emph{seeds} or \emph{kmers}) of the read are extracted, then
aligned to the reference, and finally prospective matching locations are
\emph{extended} on both sides to align the full read.

While such a heuristic may produce acceptable alignments in many cases, it
fundamentally does not provide quality guarantees, resulting in suboptimal
alignment accuracy.
%
In contrast, we demonstrate in this work that seeds can benefit optimal
alignment as well.

\para{Key challenges in optimal alignment}
%
Finding optimal alignments is desirable but expensive in the worst case,
requiring $\Oh(Nm)$ time~\citep{equi2019complexity}, for graph size $N$ and read
length $m$.
%
Unfortunately, most optimal sequence-to-graph aligners rely on dynamic
programming (DP) and always reach this worst-case asymptotic runtime. Such
aligners include \vargas~\citep{darby2020vargas},
\pasgal~\citep{jain_accelerating_2019},
\graphaligner~\citep{rautiainen_bitparallel_2019},
\hga~\citep{feng2021accelerating}, and \vg~\citep{garrison_variation_2018},
which use bit-level optimizations and parallelization to increase their
throughput.

In contrast, \astarix~\citep{ivanov2020astarix} follows the promising direction
of using a heuristic to avoid worst-case runtime on realistic data. To this end,
\astarix rephrases the task of alignment as a shortest-path problem in an
\emph{alignment graph} extended by a \emph{trie index}, and solves it using the
\A~algorithm instantiated with a problem-specific \prefixh. Importantly, its
choice of heuristic only affects performance, not optimality.
%
Unlike DP-based algorithms, this \prefixh allows scaling sublinearly with the
reference size, substantially increasing performance on large genomes. However,
it can only efficiently align reads of limited length.

\para{This work: Optimal alignment for short and long reads}
%
In this work, we address the key challenge of scaling to long HiFi reads, while
retaining the superior scaling of \astarix in the size of the reference graph.
%
To this end, we instantiate the \A algorithm with a novel \seedh, which
outperforms existing optimal aligners on both short and long HiFi reads.
%
Specifically, the \seedh utilizes information from the whole read to narrowly
direct the \A search by placing \emph{crumbs} on graph nodes which lead up to a
\emph{seed match}, \ie, an exact match of a substring of the read.

Overall, the contributions presented in this work are:
%
\begin{enumerate}
    \item A novel \A~\seedh that exploits information from the whole read to
    quickly align it to a general graphs reference.
    \item An optimality proof showing that the \seedh always finds an alignment
    with minimal edit distance.
	\item An implementation of the \seedh as part of the \astarix aligner.
    \item An extensive evaluation of our approach, showing that we align both
    short Illumina reads and long HiFi reads to both linear and graph references
    $\geq 60 \times$ faster than existing optimal aligners.
    \item A demonstration of superior empirical runtime scaling in the reference
    size $N$: $N^{0.46}$ on Illumina reads and $N^{0.11}$ on HiFi reads.
\end{enumerate}

\subsection{\A~algorithm for finding a shortest path} \label{sec:astar}
%
The \A~algorithm is a shortest path algorithm that generalizes Dijkstra's
algorithm by directing the search towards the target.
Given a weighted graph $G=(V,E)$, the \A~algorithm finds a shortest path from
sources $S \subseteq V$ to targets $T \subseteq V$.
%
To prioritize paths that lead to a target, it relies on an admissible heuristic
function $h \colon V \to \mathbb{R}_{\geq 0}$, where $h(v)$ estimates the
remaining length of the shortest path from a given node $v \in V$ to a target
$t \in T$.


\paragraph{Algorithm}
% 
In a nutshell, the \A~algorithm maintains a set of \emph{explored} nodes,
initialized by all possible starting nodes $S$. It then iteratively
\emph{expands} the explored state $v$ with lowest estimated total cost $f(v)$ by
exploring all its neighbors. Here, $f(v) := g(v) + h(v)$, where $g(v)$ is the
distance from $s \in S$ to $v$, and $h(v)$ is the estimated distance from $v$ to
$t \in T$.
%
When the \A~algorithm expands a target node $t \in T$, it reconstructs the path
leading to $t$ and returns it.
%
%For completeness, \cref{app:astar} provides an implementation of \A.

\paragraph{Admissibility}
%
The \A~algorithm is guaranteed to find a shortest path if its heuristic $h$
provides a lower bound on the distance to the closest target, often referred to
as $h$ being \emph{admissible} or optimistic.

Further, the performance of the \A~algorithm relies critically on the choice of
$h$. Specifically, it is crucial to have low estimates for the optimal paths but
also to have high estimates for suboptimal paths.

\paragraph{Discussion}
%
To summarize, we use the \A~algorithm to find a shortest path from $\st{u}{0}$
to $\st{v}{|q|}$ in $\AG$. To guarantee optimality, its heuristic function
$h\st{v}{i}$ must provide a lower bound on the shortest distance from state
$\st{v}{i}$ to a terminal state of the form $\st{w}{\lvert q \rvert}$.
%
Equivalently, $h\st{v}{i}$ should lower bound the minimal cost of aligning
$q[i{:}]$ to $\RG$ starting from $v$, where $q[i{:}]$ denotes the suffix of $q$
starting at position $i$ ($0$-indexed).
%
The key challenge is thus finding a heuristic that is not only admissible but
also yields favorable performance.